\section*{Tableaux de rencontre des donjons}\label{tableaux-de-rencontre-des-donjons}

Cette section fournit des tableaux de rencontres par niveau de donjon,
en utilisant les monstres disponibles dans ce livre. Les arbitres se
servant de monstres supplémentaires (ou alternatifs) doivent, soit
adapter les tableaux afin de les inclure, soit créer leurs propres
tableaux de rencontre. Des tableaux spéciaux peuvent également être
créés pour refléter l'équilibre différent de monstres habitant un donjon
spécifique.

\subsection*{Comment tirer une rencontre}\label{comment-tirer-une-rencontre}

Lancez 1d20 et cherchez le résultat dans la colonne du tableau
ci-dessous correspondant au niveau du donjon exploré. Le résultat
indique le monstre rencontré, ainsi que le nombre qui apparaît entre
parenthèses.

\subsection*{Remarques}\label{remarques}

\paragraph{PNJ Aventuriers :} Les rencontres avec des PNJ aventuriers sont
répertoriées dans les tableaux comme « Aventurier basique » ou «
Aventurier expert ». On trouvera les étapes pour définir ces PNJ
aventuriers dans \href{Groupe_d’aventuriers.md}{Groupes d'aventurier}.

\paragraph{Nombre apparaissant :} Le nombre de monstres indiqué pour
certaines créatures dans le tableau ne correspond pas aux valeurs
répertoriées dans leur description. S'il le souhaite et par souci de
cohérence, l'arbitre peut à la place utiliser la valeur apparaissant
dans la description du monstre.

\subsection*{Rencontre selon le niveau du donjon (Niveaux 1 à 3)}\label{rencontre-selon-le-niveau-du-donjon-niveaux-1-uxe0-3}

\begin{table}[H]
 \centering
\begin{tabular}[]{llll}
\titlecell D20 & \titlecell Niveau 1 & \titlecell Niveau 2 & \titlecell Niveau 3 \\
1 & Abeille tueuse (1d10) & Araignée Veuve noire (1d3) & Araignée
Tarentelle (1d3) \\
2 & Acolyte (1d8) & Babouin des rochers (2d6) & Aventurier basique
(1d4+4) \\
3 & Araignée crabe (1d4) & Berserker (1d6) & Cube gélatineux (1) \\
4 & Bandit (1d8) & Elfe (1d4) & Doppelganger (1d6) \\
5 & Commerçant (1d8) & Félin, Lion des montagnes (1d6) & Fourmi géante
(2d4) \\
6 & Esprit follet (3d6) & Gnoll (1d6) & Gargouille (1d6) \\
7 & Gnome (1d6) & Goule (1d6) & Gelée ocre (1) \\
8 & Gobelin (2d4) & Hobgobelin (1d6) & Goblours (2d4) \\
9 & Kobold (4d4) & Homme-lézard (2d4) & Gorille blanc (1d6) \\
10 & Limon vert (1d4) & Lézard, Draco (1d4) & Harpie (1d6) \\
11 & Loup (2d6) & Mouche voleuse (1d6) & Rat-garou (1d8) \\
12 & Lézard Gecko (1d3) & Noble (2d6) & Médium (1d4) \\
13 & Musaraigne géante (1d10) & Néandertalien (1d10) & Méduse (1d3) \\
14 & Nain (1d6) & Pixie (2d4) & Nécrophage (1d6) \\
15 & Orque (2d4) & Scarabée à huile (1d8) & Ogre (1d6) \\
16 & Scarabée de feu (1d8) & Serpent, Crotale (1d8) & Ombre (1d8) \\
17 & Serpent, Cobra cracheur (1d6) & Troglodyte (1d8) & Scarabée tigré
(1d6) \\
18 & Squelette (3d4) & Vase grise (1) & Statue vivante de cristal
(1d6) \\
19 & Strige (1d10) & Vétéran (2d4) & Thoul (1d6) \\
20 & Tinigens (3d6) & Zombie (2d4) & Ver charognard (1d3) \\
\end{tabular}
\end{table}