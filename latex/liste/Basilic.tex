\begin{monster}
\monstercarac{name}{Basilic}
\monstercarac{description}{
  Lézard de forme serpentine, long de 3 m, et dépourvu d'intelligence,
mais hautement magique. Il niche dans les cavernes et les ronces
épineuses.

}
\monsterdetail{Surprise}{Les personnages surpris par le basilic subissent une attaque de son
regard pétrifiant.
}
\monsterdetail{Toucher pétrifiant}{Toute créature touchée par le basilic est transformée en pierre (jet de
sauvegarde contre la pétrification).
}
\monsterdetail{Regard pétrifiant}{Toute créature rencontrant le regard du basilic est transformée en
pierre (jet de sauvegarde contre la pétrification). À moins d'éviter son
regard ou d'utiliser un miroir, les personnes en mêlée contre le basilic
sont affectées chaque round.
}
\monsterdetail{Éviter son regard}{Une personne qui cherche à combattre le basilic en évitant son regard
subit un malus de --4 pour toucher, tandis que le basilic bénéficie d'un
bonus de +2.
}
\monsterdetail{Utiliser un miroir}{Le reflet du basilic est inoffensif. Se battre en regardant dans un
miroir inflige un malus de --1 à l'attaque. Si le basilic voit son
propre reflet (2 chances sur 6), il doit effectuer un jet de sauvegarde
ou être pétrifié.
}
\monstercarac{ca}{4 [15]}
\monstercarac{hd}{6+1** (28 pv)}
\monstercarac{taco}{13 [+6]}
\monstercarac{moral}{9}
\monstercarac{alignement}{Neutre}
\monstercarac{xp}{950}
\monstercarac{nombre_donjon}{1d6}
\monstercarac{nombre_exterieur}{1d6}
\monstercarac{tresor}{F}
\monstercarac{save_mort_poison}{10}
\monstercarac{save_baguettes}{11}
\monstercarac{save_paralysie_petrification}{12}
\monstercarac{save_souffles}{13}
\monstercarac{save_sorts_sceptres_batons}{14}
\monsterattack{1 × morsure }{1d10 + pétrification)}
\monsterattack{1 × regard }{pétrification)}
\end{monster}
