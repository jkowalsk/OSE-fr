\begin{monster}
\monstercarac{name}{Vampire}
\monstercarac{description}{
  Monstre mort-vivant très redouté qui vit en buvant le sang des mortels.
Habite dans les ruines, les tombes et les lieux désertés.

}
\monsterdetail{Mort-vivant}{Ne fait aucun bruit jusqu'à ce qu'il attaque. Immunisé contre les effets
affectant les créatures vivantes (par exemple, le poison). Immunisé
contre les sorts affectant ou lisant l'esprit (par exemple, charme,
paralysie, sommeil).
}
\monsterdetail{Immunité aux dégâts normaux}{Ne peut être blessé que par des attaques magiques.
}
\monsterdetail{Absorption d’énergie}{Une cible touchée avec succès perd deux niveaux d'expérience (ou Dés de
vie) de manière permanente. Ceci entraîne la perte de points de vie
correspondant aux DV de perdus, ainsi que les autres capacités dues aux
niveaux perdus (sorts, jets de sauvegarde, etc.). L'XP du personnage est
réduite au minimum du nouveau niveau. Une personne qui perd tous ses
niveaux devient un vampire en 3 jours.
}
\monsterdetail{Regard (charme)}{Quiconque croise le regard du vampire doit effectuer un jet de
sauvegarde contre les sorts avec un malus de --2. En cas d'échec, la
victime est charmée et doit : se déplacer en direction du vampire (en
résistant à ceux qui essaient de l'en empêcher) ; le défendre ; et obéir
à ses ordres (à condition de les comprendre). Incapable de lancer de
sorts, ou d'utiliser d'objets magiques, ou s'en prendre au vampire. La
mort du vampire rompt le charme.
}
\monsterdetail{Régénération}{Un vampire ayant subi des dégâts regagne 3 pv au début de chaque round,
tant qu'il est vivant.
}
\monsterdetail{À 0 pv}{Il passe en forme gazeuse et fuit en direction de son cercueil.
}
\monsterdetail{Changeur de forme}{À volonté ; prend 1 round :

\begin{enumerate}
\def\labelenumi{\arabic{enumi}.}
\tightlist
\item
  \textbf{Humanoïde :} Forme standard.
\item
  \textbf{Loup géant :} Att 1 × morsure (2d4), DP 45 m (15 m), CA, DV,
  Ml : Comme ceux du vampire.
\item
  \textbf{Chauve-souris géante :} Att 1 × morsure (1d4), DP 9 m (3 m) /
  54 m (18 m) vol, CA, DV, Ml : Comme ceux du vampire.
\item
  \textbf{Nuage gazeux :} DP 54 m (18 m) vol.~Immunisé aux dégâts
  normaux. Ne peut pas attaquer.
\end{enumerate}
}
\monsterdetail{Convocation de bêtes}{Sous sa forme humaine uniquement, le vampire peut faire venir à lui des
créatures des environs :

\begin{itemize}
\tightlist
\item
  1d10 × 10 \href{Rat.md\#Rat-normal}{rats},
\item
  5d4 \href{Rat.md\#Rat-géant}{rats géants},
\item
  1d10 × 10
  \href{Chauve-souris.md\#Chauve-souris-normale}{chauves-souris},
\item
  3d6 \href{Chauve-souris.md\#Chauve-souris-géante}{chauves-souris
  géantes},
\item
  3d6 \href{Loup.md\#Loup-normal}{loups communs}, ou
\item
  2d4 \href{Loup.md\#Loup-géant}{loups géants}.
\end{itemize}
}
\monsterdetail{Cercueils}{Doit se reposer durant la journée dans un cercueil ou perdre 2d6 pv
(uniquement régénérés en se reposant une journée complète). Ne peut pas
se reposer dans un cercueil béni. Garde toujours plusieurs cercueils
dans des endroits cachés.
}
\monsterdetail{Vulnérabilités}{\begin{enumerate}
\def\labelenumi{\arabic{enumi}.}
\tightlist
\item
  \textbf{Ail :} L'odeur le repousse : jet de sauvegarde contre le
  poison ou incapacité à attaquer ce round.
\item
  \textbf{Symboles sacrés :} S'ils sont présentés, ils gardent le
  vampire à distance (3 m). Peut attaquer le porteur à partir d'une
  autre direction.
\item
  \textbf{Eau vive :} Ne peut la traverser (sous aucune forme), sauf au
  moyen d'un pont ou s'il est transporté à l'intérieur d'un cercueil.
\item
  \textbf{Miroirs :} Les évite ; on n'y voit pas son reflet.
\item
  \textbf{Lumière éternelle :} Partiellement aveuglé par la lumière de
  ce sort (malus de --4 aux attaques).
\end{enumerate}
}
\monsterdetail{Destruction}{\begin{enumerate}
\def\labelenumi{\arabic{enumi}.}
\tightlist
\item
  \textbf{Lumière du soleil :} Doit effectuer un jet de sauvegarde
  contre la mort à chaque round où il y est exposé ou est désintégré.
\item
  \textbf{Pieu dans le cœur :} Tue le vampire définitivement.
\item
  \textbf{Immersion dans l'eau :} Pendant 1 tour, le tue définitivement.
\item
  \textbf{Détruire les cercueils :} Le vampire est tué de façon
  permanente si tous ses points de vie sont perdus et qu'il est
  incapable de se reposer (voir cercueils).
\end{enumerate}
}
\monstercarac{ca}{2 [17]}
\monstercarac{hd}{7 à 9** (31/36/40 pv)}
\monstercarac{taco}{13 [+6]/12 [+7]/12 [+7]}
\monstercarac{moral}{11}
\monstercarac{alignement}{Chaotique}
\monstercarac{xp}{1 250/1 750/2 300}
\monstercarac{nombre_donjon}{1d4}
\monstercarac{nombre_exterieur}{1d6}
\monstercarac{tresor}{F}
\monstercarac{save_mort_poison}{8}
\monstercarac{save_baguettes}{9}
\monstercarac{save_paralysie_petrification}{10}
\monstercarac{save_souffles}{10}
\monstercarac{save_sorts_sceptres_batons}{12}
\monsterattack{1 × toucher }{1d10 + absorption d’énergie) ou 1 × regard}
\end{monster}
