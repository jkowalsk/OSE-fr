\begin{monster}
\monstercarac{name}{Dragon blanc}
\monstercarac{description}{
  Une race fière et ancienne de reptiles gigantesques, carnivores et
ailés. Il existe de nombreuses sous-espèces de dragons, dont beaucoup se
distinguent par la couleur de leurs écailles. Tous les dragons couvent
leurs œufs et amassent un trésor dans leurs repaires, loin des zones de
la civilisation humaine.

  On le trouve dans les régions froides.

}
\monsterdetail{Comportement}{Les dragons chaotiques essaient généralement de dévorer les humains,
mais ils peuvent parfois simplement les capturer. Les dragons neutres
peuvent attaquer ou ignorer les humains. Les dragons loyaux peuvent
aider les groupes qu'ils jugent dignes de cet honneur.
}
\monsterdetail{Fierté}{Les dragons sont des créatures extrêmement fières et auront toujours une
oreille attentive à la flatterie.
}
\monsterdetail{Schéma d’attaque}{Un dragon attaque toujours en premier avec son souffle, puis il souffle
à nouveau ou effectue des attaques de mêlée (à probabilité égale).
}
\monsterdetail{Souffle - général}{Peut être utilisé jusqu'à trois fois par jour. Toutes les créatures se
trouvant dans la zone concernée subissent des dégâts égaux aux points de
vie actuels du dragon (jet de sauvegarde contre les souffles divise par
deux). Formes de souffle :

\begin{itemize}
\tightlist
\item
  \textbf{Nuage :} 15 m de long, 12 m de large, 6 m de haut.
\item
  \textbf{Cône :} 60 cm de large à l'entrée, 9 m à l'extrémité.
\item
  \textbf{Ligne :} 1,50 m de large sur toute la longueur.
\end{itemize}
}
\monsterdetail{Immunité naturelle}{Les dragons sont immunisés aux dégâts de leur propre souffle ou à des
versions inférieures du même type. Ils réussissent automatiquement leurs
jets de sauvegarde contre les formes d'attaques similaires. Par exemple,
un dragon rouge est immunisé contre l'huile enflammée et ne subit que la
moitié des dégâts du sort boule de feu.
}
\monsterdetail{Langue et sorts}{Certains dragons sont capables de parler (leur propre langue, ainsi que
le Commun). La probabilité est répertoriée par sous-espèces. Ceux qui
sont doués de parole peuvent également lancer des sorts de magicien
choisis au hasard (le nombre et le niveau des sorts répertoriés).
}
\monsterdetail{Dormir}{Les chances qu'un dragon soit endormi lorsqu'on le rencontre sont
répertoriées par sous-espèces. On peut attaquer un dragon endormi
pendant un round avec un bonus de +2 pour toucher. Mais attention : les
dragons peuvent parfois faire semblant de dormir !
}
\monsterdetail{Soumettre}{Le dragon se rendra s'il est réduit à 0 pv par des attaques non létales
(voir Soumettre dans \href{/Combat_:_autres_détails}{Combat : autres
détails}), admettant qu'il a été vaincu. Notez que les dégâts de
soumission ne réduisent pas les dégâts infligés par le souffle. Un
dragon soumis tentera de s'échapper ou d'attaquer ses ravisseurs, si
l'occasion se présente ou s'il reçoit un ordre suicidaire. Un dragon
soumis peut être vendu jusqu'à 1 000 po par pv.
}
\monsterdetail{Âge}{Les statistiques suivantes définissent des dragons de taille moyenne.
Les jeunes dragons peuvent avoir jusqu'à 3 DV de moins et ne posséder
que ¼ à ½ trésors. Les dragons plus âgés peuvent avoir jusqu'à 3 DV de
plus et deux fois plus de trésors.
}
\monsterdetail{Repaire}{Le trésor d'un dragon est toujours conservé dans son repaire bien caché
et est rarement laissé sans surveillance.
}
\monsterdetail{Souffle}{Cône de froid de 24 m de long.
}
\monsterdetail{Langue et sorts}{10 \% ; 3 × 1er niveau.
}
\monsterdetail{Sommeil}{50 \%.
}
\monstercarac{ca}{3 [16]}
\monstercarac{hd}{6** (27 pv)}
\monstercarac{taco}{14 [+5]}
\monstercarac{moral}{8}
\monstercarac{alignement}{Neutre}
\monstercarac{xp}{725}
\monstercarac{nombre_donjon}{1d4}
\monstercarac{nombre_exterieur}{1d4}
\monstercarac{tresor}{H}
\monstercarac{mvt}{27 m (9 m) / 72 m (24 m) vol}
\monstercarac{save_mort_poison}{10}
\monstercarac{save_baguettes}{11}
\monstercarac{save_paralysie_petrification}{12}
\monstercarac{save_souffles}{13}
\monstercarac{save_sorts_sceptres_batons}{14}
\monsterattack{2 × griffes }{1d4}
\monsterattack{1 × morsure }{2d8}
\monsterattack{souffle}{}
\end{monster}
