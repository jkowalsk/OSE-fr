\begin{monster}
\monstercarac{name}{Calmar géant}
\monstercarac{description}{
  Céphalopode géant doté de dix tentacules. Peut attaquer les navires à
l'aide de ses deux plus grands. Habite dans les profondeurs de la mer et
ne remonte à la surface que pour trouver des proies.

}
\monsterdetail{Écraser les bateaux}{25 \% de chances d'enrouler ses 2 grands tentacules autour d'un bateau,
infligeant 1d10 points de dégâts de structure par tentacule. Lorsque les
grands tentacules attaquent un bateau, le rostre inflige automatiquement
lors des rounds suivants 2 points de dégâts de structure.
}
\monsterdetail{Saisir l’équipage}{Le calmar géant a 75 \% de chances de saisir un membre d'équipage sur le
pont et de le faire tomber à l'eau pour le dévorer.
}
\monsterdetail{Constriction}{Après une attaque réussie, les tentacules saisissent la victime et se
resserrent sur elle, lui infligeant automatiquement 1d4 points de dégâts
par round.
}
\monsterdetail{Sectionner les tentacules}{Nécessite de porter un coup infligeant 6 points de dégâts ou plus (pour
les 8 petits tentacules) ou 10 dégâts ou plus (pour les 2 grands
tentacules).
}
\monsterdetail{Nuage d’encre}{Lorsque le calmar géant s'échappe, il peut émettre un nuage d'encre
noire (18 m de diamètre) et s'enfuir à une vitesse 3 fois supérieure à
son déplacement normal. Il ne peut le faire que deux fois par jour au
maximum.
}
\monsterdetail{Spécimens gigantesques}{Il arrive de rencontrer des spécimens deux ou trois fois plus gros.
}
\monstercarac{ca}{7 [12]}
\monstercarac{hd}{6 (27 pv)}
\monstercarac{taco}{14 [+5]}
\monstercarac{moral}{7}
\monstercarac{alignement}{Neutre}
\monstercarac{xp}{275}
\monstercarac{nombre_donjon}{0}
\monstercarac{nombre_exterieur}{1d4}
\monstercarac{tresor}{V}
\monstercarac{save_mort_poison}{12}
\monstercarac{save_baguettes}{13}
\monstercarac{save_paralysie_petrification}{14}
\monstercarac{save_souffles}{15}
\monstercarac{save_sorts_sceptres_batons}{16}
\monsterattack{8 × petit tentacule }{1d4 + constriction)}
\monsterattack{2 × grand tentacule }{1d4 + constriction ou 1d10 points de dégâts de structure)}
\monsterattack{1 × rostre }{1d10 ou 2 points de dégâts de structure)}
\end{monster}
