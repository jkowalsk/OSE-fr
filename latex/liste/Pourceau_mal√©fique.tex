\begin{monster}
\monstercarac{name}{Pourceau maléfique}
\monstercarac{description}{
  Changeurs de forme capables d'alterner entre leur apparence humaine et
animale.

  Cet humain corpulent peut se transformer en gros pourceau. Il aime
manger de la chair humaine et rôde dans les villages humains isolés à
proximité des forêts ou des marais.

}
\monsterdetail{Forme humaine}{Possède des caractéristiques physiques qui rappellent sa nature animale.
}
\monsterdetail{Immunité aux dégâts normaux}{Sous sa forme animale, ne peut être blessé que par les armes en argent
ou la magie.
}
\monsterdetail{Langues}{Sous sa forme humaine, il parle normalement, mais sous forme animale, il
ne peut communiquer qu'avec les animaux du même type.
}
\monsterdetail{Armure}{Les armures gênent la transformation d'un lycanthrope ; il n'en porte
jamais.
}
\monsterdetail{Appel d’animaux}{Peut convoquer 1 ou 2 animaux de son espèce présents dans les alentours
(les rats-garous appellent des rats géants --- voir p.~XX). Ils arrivent
en 1d4 rounds.
}
\monsterdetail{Herbe au loup}{Un lycanthrope touché par de l'herbe au loup doit effectuer un jet de
sauvegarde contre le poison ou s'enfuir, terrorisé.
}
\monsterdetail{Retour}{À sa mort, le lycanthrope reprend sa forme humaine.
}
\monsterdetail{Odeur}{Les chevaux et certains autres animaux sentent les lycanthropes et
peuvent prendre peur.
}
\monsterdetail{Infection}{Un personnage qui perd plus de la moitié de ses points de vie à cause
des attaques naturelles d'un lycanthrope (morsure, griffes\ldots)
contracte la lycanthropie. Les humains deviennent des créatures-garous
du même type et passent sous le contrôle de l'arbitre ; les non-humains
meurent. La maladie se déclare après 2d12 jours, des signes d'infection
étant déjà visibles à la moitié de ce temps.
}
\monsterdetail{Changement de forme}{Uniquement la nuit.
}
\monsterdetail{Embuscade}{Le pourceau maléfique préfère attaquer par surprise.
}
\monsterdetail{Charme-personne}{3 fois par jour. Jet de sauvegarde contre les sorts avec un malus de --2
ou la cible est charmée : elle se déplace en direction du pourceau
maléfique (en résistant à ceux qui essaient de l'en empêcher) ; obéit à
ces ordres (à condition de les comprendre) ; le défend et est incapable
de lancer des sorts ou d'utiliser des objets magiques. De plus, elle est
incapable de s'en prendre à lui. Tuer le pourceau brise le charme.
}
\monsterdetail{Victimes charmées}{1d4--1 accompagnent un pourceau maléfique.
}
\monstercarac{ca}{}
\monstercarac{hd}{}
\monstercarac{taco}{}
\monstercarac{moral}{10}
\monstercarac{alignement}{Chaotique}
\monstercarac{xp}{1 600}
\monstercarac{nombre_donjon}{}
\monstercarac{nombre_exterieur}{}
\monstercarac{tresor}{}
\monstercarac{mvt}{}
\monstercarac{save_mort_poison}{}
\monstercarac{save_baguettes}{}
\monstercarac{save_paralysie_petrification}{}
\monstercarac{save_souffles}{}
\monstercarac{save_sorts_sceptres_batons}{}
\monsterattack{1 × défense }{2d6}
\monsterattack{1 × arme }{1d6 }
\monsterattack{selon l’arme) }{}
\monsterattack{1 × magie }{charme}
\end{monster}
