\begin{monster}
\monstercarac{name}{Grand troupeau}
\monstercarac{description}{
  Ces animaux sauvages vivent dans les pâturages en grands troupeaux. Le
type exact dépend du terrain.

  Par exemple : wapiti ou élan.

}
\monsterdetail{Débandade}{Les troupeaux de 20 individus ou plus peuvent piétiner ceux qui se
trouvent sur leur passage. 3-sur-4 chances à chaque round. Bonus de +4
pour toucher les créatures de taille humaine ou plus petites. 1d20
points de dégâts.
}
\monsterdetail{Mâles}{Dans les groupes de 3 individus ou plus, un quart seulement sont des
mâles. Ils ont 1d4 points de vie supplémentaires et protègent le
troupeau.
}
\monsterdetail{Femelles et petits}{Ils fuient le danger. Les femelles n'ont pas d'attaque de percussion.
Les petits n'ont que la moitié des points de vie.
}
\monsterdetail{Piétinement}{Les troupeaux de 20 individus ou plus peuvent piétiner ceux qui se
trouvent sur leur passage. 3-sur-4 chances à chaque round. Bonus de +4
pour toucher les créatures de taille humaine ou plus petites. 1d20
points de dégâts.
}
\monstercarac{ca}{7 [12]}
\monstercarac{hd}{4 (18 pv)}
\monstercarac{taco}{16 [+3]}
\monstercarac{moral}{5}
\monstercarac{alignement}{Neutre}
\monstercarac{xp}{75}
\monstercarac{nombre_donjon}{0}
\monstercarac{nombre_exterieur}{3d10}
\monstercarac{tresor}{Aucun}
\monstercarac{mvt}{72 m (24 m)}
\monstercarac{save_mort_poison}{12}
\monstercarac{save_baguettes}{13}
\monstercarac{save_paralysie_petrification}{14}
\monstercarac{save_souffles}{15}
\monstercarac{save_sorts_sceptres_batons}{16}
\monsterattack{1 × percussion }{1d8}
\end{monster}
