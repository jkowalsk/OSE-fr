\begin{monster}
\monstercarac{name}{Pirate}
\monstercarac{description}{
  Ces marins gagnent leur vie en attaquant les villages côtiers, en volant
d'autres navires et en faisant commerce illégal d'esclaves. On les
croise généralement en haute mer. Réputés pour leur conduite mauvaise,
ils sont impitoyables.

}
\monsterdetail{Navires et équipage}{Ils dépendent de l'endroit où on rencontre les pirates. Rivières ou lacs
: 1d8 bateaux fluviaux avec chacun 1d2 × 10 pirates ; eaux côtières :
1d6 petites galères avec chacune 1d3+1 × 10 pirates ; tous : drakkars
1d4 avec chacun 1d3+2 × 10 pirates ; océan : 1d3 petits navires de
guerre avec chacun 1d5+3 × 10 pirates.
}
\monsterdetail{Armes}{50 \% du groupe est équipé avec : une armure de cuir et une épée ; 35 \%
est équipé avec : une armure de cuir, une épée et une arbalète ; 15 \%
est équipé avec : une cotte de mailles et une épée.
}
\monsterdetail{Chefs}{Par groupe de 30 pirates, on trouve un guerrier de 4e niveau. Par groupe
de 50 pirates et pour chaque navire, un guerrier de 5e niveau. Par
groupe de 100 pirates et par flotte, un guerrier de 8e niveau.
}
\monsterdetail{Commandant de flotte}{Les flottes de 300 pirates ou plus sont dirigées par un seigneur pirate
(guerrier de 11eniveau). 75 \% de chances qu'il y ait un magicien de
niveau 1d2+8.
}
\monsterdetail{Traître}{Les pirates n'hésitent pas à attaquer d'autres pirates, s'ils y trouvent
leur profit.
}
\monsterdetail{Prisonniers}{Il y a 25 \% de chances de demandes de rançons pour 1d3 de leurs
prisonniers.
}
\monsterdetail{Trésor}{Réparti entre les vaisseaux de la flotte. Au lieu de l'avoir à bord, ils
peuvent avoir une carte de l'endroit où il est enterré.
}
\monsterdetail{Havres}{Les villes côtières fortifiées où aucune loi n'a cours peuvent
constituer un refuge pour les pirates.
}
\monstercarac{ca}{7 [12] or 5}
\monstercarac{hd}{1 (4 pv)}
\monstercarac{taco}{19 [0]}
\monstercarac{moral}{7}
\monstercarac{alignement}{Chaotique}
\monstercarac{xp}{10}
\monstercarac{nombre_donjon}{0}
\monstercarac{nombre_exterieur}{voir ci-dessous}
\monstercarac{tresor}{A}
\monstercarac{save_mort_poison}{12}
\monstercarac{save_baguettes}{13}
\monstercarac{save_paralysie_petrification}{14}
\monstercarac{save_souffles}{15}
\monstercarac{save_sorts_sceptres_batons}{16}
\monsterattack{1 × arme }{1d6 ou selon l’arme)}
\end{monster}
