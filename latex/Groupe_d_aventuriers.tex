\chapter{Groupe d'aventuriers}\label{groupe-daventuriers}
Cette procédure génère des groupes de PNJ aventuriers. Comme la
procédure est assez complexe, l'arbitre peut souhaiter prétirer quelques
groupes de PNJ afin de les utiliser lors de rencontres aléatoires. Les
détails généraux suivants s'appliquent à tous les types de PNJ décrits :

\begin{itemize}
\tightlist
\item
  \textbf{Sorts :} Si des lanceurs de sorts sont présents, choisissez ou
  tirez aléatoirement leurs sorts mémorisés.
\item
  \textbf{Équipement :} L'équipement classique pour partir à l'aventure.
\item
  \textbf{Trésor :} Trésor de type U+V, réparti parmi le groupe.
\item
  \textbf{Ordre de marche :} Décidé par l'arbitre.
\end{itemize}

Si les classes présentées dans cet ouvrage ne sont pas utilisées,
l'arbitre doit remplacer les classes listées par des équivalents
utilisés dans sa campagne.

\section*{Aventuriers basiques}\label{aventuriers-basiques}

\begin{itemize}
\tightlist
\item
  \textbf{Composition :} 1d4+4 personnages de classe et niveau choisis
  au hasard (voir ci-dessous).
\item
  \textbf{Alignement :} Lancez pour définir l'alignement de chaque PNJ
  ou une seule fois pour tout le groupe.
\end{itemize}

\section*{Aventuriers experts}\label{aventuriers-experts}

\begin{itemize}
\tightlist
\item
  \textbf{Composition :} 1d6+3 personnages de classe et niveau choisis
  au hasard (voir ci-dessous).
\item
  \textbf{Alignement :} Lancez pour définir l'alignement de chaque PNJ
  ou une seule fois pour tout le groupe.
\item
  \textbf{Montures :} Dans les contrées sauvages, il y a 75 \% de
  chances qu'elles soient montées.
\item
  \textbf{Objets magiques :} Par individu : Il est possible que le PNJ
  ait un objet magique mentionné dans chaque sous-table d'objets
  magiques appropriée.
  La probabilité par sous-table est de 5 \% par niveau du PNJ. Les
  objets tirés qui ne peuvent pas être utilisés par ce PNJ doivent être
  ignorés (pas de relance).
\end{itemize}

\section*{Clerc de haut-niveau}\label{clerc-de-haut-niveau}

Un clerc de haut niveau et ses fidèles (montures et objets magiques
comme des Aventuriers experts).

\begin{itemize}
\tightlist
\item
  \textbf{Composition :} Chef (Clerc de niveau 1d6+6), 1d4 Clercs
  (niveau 1d4+1), 1d3 Guerriers (niveau 1d6).
\item
  \textbf{Alignement :} Tirer pour l'ensemble du groupe.
\end{itemize}

\section*{Guerrier de haut-niveau}\label{guerrier-de-haut-niveau}

Un guerrier de haut niveau et son groupe de suivants, souvent en route
vers une guerre ou en revenant (montures et objets magiques comme des
Aventuriers experts).

\begin{itemize}
\tightlist
\item
  \textbf{Composition :} Chef (guerrier de niveau 1d4+6), 2d4 Suivants
  (niveau 1d4+2, n'importe quelle classe).
\item
  \textbf{Alignement :} Tirer pour l'ensemble du groupe.
\end{itemize}

\section*{Magicien de haut-niveau}\label{magicien-de-haut-niveau}

Un magicien de haut niveau, accompagné de ses apprentis et d'un groupe
de gardes embauchés, souvent à la recherche de savoirs mystérieux
(montures et objets magiques comme des Aventuriers experts).

\begin{itemize}
\tightlist
\item
  \textbf{Composition :} Chef (Magicien de niveau 1d4+6), 1d4 Apprentis
  (Magiciens de niveau 1d3), 1d4 Mercenaires (Guerriers de niveau
  1d4+1).
\item
  \textbf{Alignement :} Tirer pour définir l'alignement du chef. Les
  apprentis ont le même alignement que le chef, tandis que les
  mercenaires peuvent être d'un alignement différent.
\end{itemize}

\section*{Classe, Niveau et alignement des PNJ Aventuriers}\label{classe-et-niveau-des-pnj-aventuriers}

\begin{table}[h]
\parbox{.45\linewidth}{
\centering
\begin{tabular}[]{@{}lll@{}}
\titlecell Niveau (d8) & \titlecell Classe & \titlecell Basique \\
1 & Clerc & 1d3 \\
2 & Nain & 1d3 \\
3 & Elfe & 1d3 \\
4 & Guerrier & 1d3 \\
5 & Guerrier & 1d3 \\
6 & Tinigens & 1d3 \\
7 & Magicien & 1d3 \\
8 & Voleur & 1d3 \\
\end{tabular}
\caption{Classe et Niveau des PNJ Aventuriers}
} 
\parbox{.45\linewidth}{
\centering
\begin{tabular}[]{@{}ll@{}}
\titlecell d6 & \titlecell Alignement \\
1--2 & Loyal \\
3--4 & Neutre \\
5--6 & Chaotique \\
\end{tabular}\\
\caption{Alignement des PNJ Aventuriers}
}

\end{table}
