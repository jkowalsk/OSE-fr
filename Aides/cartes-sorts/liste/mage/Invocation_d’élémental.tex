\begin{spell}
\spellcarac{name}{Invocation d’élémental}
\spellcarac{level}{5}
\spellcarac{duree}{Permanente}
\spellcarac{portee}{72 m}
\spellcarac{description}{
Un élémental -- être formé de matière élémentaire pure -- possédant 16
DV est convoqué à partir d'un plan élémentaire choisi par le lanceur
(air, terre, feu, eau) afin d'exécuter ses ordres.

\begin{itemize}
\item
  \textbf{Matériaux :} La convocation nécessite un grand volume de
  l'élément approprié.
\item
  \textbf{Concentration :} Elle est nécessaire pour diriger un
  élémental.
\item
  \textbf{Annulation :} Tant que dure le contrôle sur l'élémental, le
  lanceur peut dissiper le sort à n'importe quel moment, renvoyant alors
  l'élémental sur son plan d'origine.
\item
  \textbf{Interruption :} Si le lanceur se déplace à plus de la moitié
  de sa vitesse ou si sa concentration est perturbée, le contrôle qu'il
  a sur l'élémental prend fin. Ce dernier retrouve son libre arbitre et
  tente immédiatement de tuer le lanceur et tous ceux qui se mettent en
  travers de son chemin.
\item
  \textbf{Dissipation :} Un élémental convoqué peut être chassé en
  utilisant dissipation de la magie ou dissipation du mal.
\item 
  \textbf{Restrictions :} Le lanceur peut convoquer tout au plus un
  élémental de chaque plan sur une journée donnée.
\end{itemize}


}
\end{spell}
