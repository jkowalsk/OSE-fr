\section{Recherches magiques}\label{recherches-magiques}

Les classes de lanceurs de sorts peuvent effectuer des recherches afin
de créer de nouveaux sorts et objets magiques. Ce processus demande du
temps, de l'argent et parfois des ingrédients rares et difficiles à
trouver.

\subsection{Risques d'échec}\label{risques-duxe9chec}

Le succès de ces recherches n'est jamais garanti. La probabilité minimum
d'échec de toute recherche magique est de 15\%. Si la recherche échoue,
l'investissement en temps et en argent est perdu.

\subsection{Création de nouveaux sorts}\label{cruxe9ation-de-nouveaux-sorts}

Le joueur décrit en détail le sort qu'il souhaite créer, ainsi que ses
effets. L'arbitre décide alors s'il est possible de créer le sort et,
dans l'affirmative, de son niveau.

\subsubsection{Restrictions}\label{restrictions}

Le personnage doit être capable de lancer des sorts du niveau de celui
qu'il veut créer.

\subsubsection{Temps et coût}\label{temps-et-couxfbt}

Effectuer des recherches pour un nouveau sort demande deux semaines de
travail ainsi que 1 000 po par niveau du sort.

\subsection{Création d'objets magiques}\label{cruxe9ation-dobjets-magiques}

Le joueur décrit en détail l'objet qu'il envisage de concevoir, ainsi
que ses effets. L'arbitre décide si la création est possible et, si oui,
détermine les matériaux requis.

\subsubsection{Restrictions}\label{restrictions-1}

\textbf{Lanceurs de sorts divins~:} Ne peuvent créer que des objets
utilisables par leur classe.

\textbf{Lanceurs de sorts arcaniques~:} Peuvent créer tous types
d'objets, à l'exception de ceux réservés aux lanceurs de sorts divins.

\subsubsection{Matériaux}\label{matuxe9riaux}

Créer un objet magique requiert souvent des composants rares, comme des
gemmes de grande valeur ou bien des ingrédients prélevés sur des animaux
peu communs ou des monstres puissants. Il arrive fréquemment que le
concepteur de l'objet doive partir à l'aventure pour acquérir ces
matériaux.

\subsubsection{Temps et coût}\label{temps-et-couxfbt-1}

\textbf{Copier les effets d'un sort~:} De nombreux objets magiques
dupliquent les effets d'un sort existant. Concevoir un tel objet
requiert une semaine de temps de jeu et 500 po par niveau du sort copié.

\textbf{Objets à usages multiples~:} Pour un objet capable de répliquer
l'effet d'un sort plus d'une fois (par exemple, une baguette à charges
multiples), le coût en temps et en argent est multiplié par le nombre
d'utilisations.

\textbf{Autres objets~:} Pour les objets ne dupliquant pas exactement
les effets d'un sort, l'arbitre doit faire preuve de bon sens. Plus un
objet est puissant, plus il devrait être difficile à concevoir. En règle
générale, la création d'un objet magique devrait coûter entre 10 000 et
100 000 po et nécessiter entre un mois et un an de temps de jeu.
Quelques exemples~: 20 flèches +1 (10 000 po, 1 mois), une armure de
plates +1 (10 000 po, 6 mois), une boule de cristal (30 000 po, 6 mois),
un anneau de rayons X (100 000 po, 1 an).

\subsection{Autres recherches magiques}\label{autres-recherches-magiques}

Un personnage peut également tenter de créer des effets magiques qui ne
soient ni des sorts ni des objets. Par exemple, un clerc qui voudrait
consacrer un site religieux, ou bien un magicien souhaitant créer un
piège magique, une créature artificielle ou un portail.

Comme pour un objet magique, c'est à l'arbitre de décider du temps et
des moyens financiers nécessaires. Il peut également demander les choses
suivantes~:

\begin{itemize}
  \item Le lancement de certains sorts spécifiques
  \item Des ingrédients rares
  \item Que le rituel soit répété à intervalles donnés pour
  maintenir l'effet magique
\end{itemize}