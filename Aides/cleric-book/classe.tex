
\section{Classe de Clerc}\label{clerc}
\begin{itemize}
  \item \textbf{Caractéristique requise~:} Aucune
  \item \textbf{Caractéristique principale~:} SAG
  \item \textbf{Dés de vie~:} 1d6
  \item \textbf{Niveau maximum~:} 14
  \item \textbf{Armures~:} toutes, boucliers compris
  \item \textbf{Armes~:} toutes les armes contondantes
  \item \textbf{Langues~:} langue d'alignement, commun
\end{itemize}

Les clercs sont des humains ayant juré de servir une divinité.
Ils sont entraînés au combat et peuvent canaliser la puissance de leur dieu.

\subsection{Combat}\label{cls:combat}

Les clercs peuvent utiliser tous les types d'armures. Les doctrines
sacrées interdisent aux clercs l'usage des armes d'estoc et de taille.
Ils peuvent utiliser les armes suivantes~: gourdin, masse, fronde,
bâton, marteau de guerre.

\subsection{Magie divine}\label{magie-divine}

\textbf{Symbole sacré~:} Un clerc doit porter avec lui un symbole sacré.

\textbf{Disgrâce~:} Si un clerc est mis à l'index pour avoir agi
contrairement aux croyances liées à son alignement, à sa religion ou aux
règles imposées par son clergé, ses pouvoirs peuvent se retrouver
limités.

\textbf{Recherches magiques~:} Quel que soit son niveau, un clerc peut
consacrer du temps et de l'argent à mener des recherches afin de créer
de nouveaux sorts ou d'inventer d'autres effets magiques associés à sa
divinité. Au 9e niveau, il peut créer des objets magiques.

\textbf{Lancer de sorts~:} Une fois que le clerc a prouvé sa foi, il
peut à partir du 2e niveau prier pour recevoir des sorts de sa divinité.
Le nombre et la puissance de ses sorts dépendent du niveau d'expérience
du personnage.

\textbf{Utilisation d'objets magiques~:} Les clercs peuvent utiliser les
parchemins correspondant aux sorts de leur liste. Ils peuvent aussi
utiliser les objets magiques réservés aux lanceurs de sorts d'origine
divine (par exemple, certains bâtons).

\subsection{Repousser les morts-vivants (vade  retro)}\label{repousser-les-morts-vivants-vade-retro}

Les clercs peuvent faire appel au pouvoir de leur divinité pour
repousser et détruire les morts-vivants. Pour repousser les
morts-vivants, le joueur lance 2d6. L'arbitre consulte ensuite la table
ci-contre, en comparant le jet aux Dés de vie des monstres morts-vivants
ciblés.

\subsubsection{Vade retro réussi}\label{vade-retro-ruxe9ussi}

Si la tentative est un succès, le joueur lance 2d6 pour déterminer le
nombre de Dés de vie de monstres affectés (repoussés ou détruits).

\textbf{Les morts-vivants repoussés~:} Ils quittent la zone s'ils le
peuvent, et ne cherchent plus à blesser ou même à entrer en contact avec
le clerc.

\textbf{Les morts-vivants détruits (résultat "D")~:} Ils sont
instantanément et irrémédiablement détruits.

\textbf{Excédent~:} Les Dés de vie qui ne sont pas suffisants pour
affecter une créature sont perdus. Toutefois, au moins une créature sera
toujours affectée sur un Vade retro réussi.

\textbf{Groupes mixtes~:} Si un Vade retro réussit contre un groupe de
créatures variées, les premières affectées sont celles ayant le moins de
Dés de vie.

\subsubsection{Vade retro}\label{vade-retro}

\begin{osrtable}{X[1.5, c]X[c]X[c]X[c]X[c]X[c]X[c]X[c]X[1.5,l]}{2}
  & \SetCell[c=8]{c} Dés de vie du monstre & & & & & & & \\
  Niv. & 1 & 2 & 2* & 3 & 4 & 5 & 6 & 7-9 \\
  1 & 7 & 9 & 11 & -- & -- & -- & -- & -- \\
  2 & T & 7 & 9 & 11 & -- & -- & -- & -- \\
  3 & R & R & 7 & 9 & 11 & -- & -- & -- \\
  4 & D & R & R & 7 & 9 & 11 & -- & -- \\
  5 & D & D & R & R & 7 & 9 & 11 & -- \\
  6 & D & D & D & R & R & 7 & 9 & 11 \\
  7 & D & D & D & D & R & R & 7 & 9 \\
  8 & D & D & D & D & D & R & R & 7 \\
  9 & D & D & D & D & D & D & R & R \\
  10 & D & D & D & D & D & D & D & R \\
  11+ & D & D & D & D & D & D & D & D \\
\end{osrtable}
* Monstres ayant 2 DV et un pouvoir spécial.

%\subsubsection{Résultats du Vade Retro}\label{ruxe9sultats-du-vade-retro}
\begin{itemize}
  \item[\textbf{--~:}] la tentative échoue.
  \item[\textbf{Nombre~:}] si la somme des 2d6 est supérieure ou égale au nombre
  indiqué, la tentative est un succès.
  \item[\textbf{R~:}] la tentative réussit.
  \item[\textbf{D~:}] la tentative réussit, et les monstres sont détruits plutôt
  que d'être simplement repoussés.
\end{itemize}

\subsection{Au 9e niveau}\label{au-9e-niveau}

Un clerc peut construire ou établir une place forte. Tant qu'il a la
faveur de sa divinité, celle-ci intervient et la place forte peut être
achetée ou construite à moitié prix.

Une fois la place forte établie, le clerc attire des adeptes (5d6 x 10
guerriers de niveau 1 ou 2). Ces troupes sont entièrement dévouées au
clerc et n'effectuent jamais de test de moral. L'arbitre décide de la
proportion d'adeptes niveau 1 et 2, ainsi que de la proportion de
fantassins, d'archers, etc.
\columnbreak
\pagebreak
\goOneColumns

\subsection{Progression du Clerc}\label{progression-du-clerc}
\begin{osrtable}{X[c]X[c,3]X[c, 2]X[c,2]|X[c]X[c]X[c]X[c]X[c]|XXXXX}{2}
  & & & &
  \SetCell[c=5]{c} Jets de sauvegarde & & & & &
  \SetCell[c=5]{c} Sorts & & & & \\
  \textbf{Niv.} & \textbf{XP} & \textbf{DV} & \textbf{TAC0} &
  \textbf{MP} & \textbf{B} & \textbf{PP} & \textbf{S} & \textbf{SSB} &
  \textbf{1} & \textbf{2} & \textbf{3} & \textbf{4} & \textbf{5} \\
  1 & 0 & 1d6 & 19[0] & 11 & 12 & 14 & 16 & 15 & -- & -- & -- & -- &
  -- \\
  2 & 1 500 & 2d6 & 19[0] & 11 & 12 & 14 & 16 & 15 & 1 & -- & -- & --
  & -- \\
  3 & 3 000 & 3d6 & 19[0] & 11 & 12 & 14 & 16 & 15 & 2 & -- & -- & --
  & -- \\
  4 & 6 000 & 4d6 & 19[0] & 11 & 12 & 14 & 16 & 15 & 2 & 1 & -- & --
  & -- \\
  5 & 12 000 & 5d6 & 17[+2] & 9 & 10 & 12 & 14 & 12 & 2 & 2 & -- & --
  & -- \\
  6 & 25 000 & 6d6 & 17[+2] & 9 & 10 & 12 & 14 & 12 & 2 & 2 & 1 & 1 &
  -- \\
  7 & 50 000 & 7d6 & 17[+2] & 9 & 10 & 12 & 14 & 12 & 2 & 2 & 2 & 1 &
  1 \\
  8 & 100 000 & 8d6 & 17[+2] & 9 & 10 & 12 & 14 & 12 & 3 & 3 & 2 & 2
  & 1 \\
  9 & 200 000 & 9d6 & 14[+5] & 6 & 7 & 9 & 11 & 9 & 3 & 3 & 3 & 2 &
  2 \\
  10 & 300 000 & 9d6+1* & 14[+5] & 6 & 7 & 9 & 11 & 9 & 4 & 4 & 3 & 3
  & 2 \\
  11 & 400 000 & 9d6+2* & 14[+5] & 6 & 7 & 9 & 11 & 9 & 4 & 4 & 4 & 3
  & 3 \\
  12 & 500 000 & 9d6+3* & 14[+5] & 6 & 7 & 9 & 11 & 9 & 5 & 5 & 4 & 4
  & 3 \\
  13 & 600 000 & 9d6+4* & 12[+7] & 3 & 5 & 7 & 8 & 7 & 5 & 5 & 5 & 4
  & 4 \\
  14 & 700 000 & 9d6+5* & 12[+7] & 3 & 5 & 7 & 8 & 7 & 6 & 5 & 5 & 5
  & 4 \\
\end{osrtable}

* Les modificateurs de CON ne s'appliquent plus à partir du niveau 10.\\
%\textbf{Définitions des jets de sauvegarde :}
\textbf{MP :} Mort, Poison /
\textbf{ B :} Baguettes /
\textbf{PP :} Paralysie, Pétrification /
\textbf{S :} Souffles /
\textbf{SSB :} Sorts, Sceptres, Bâtons.

\goTwoColumns