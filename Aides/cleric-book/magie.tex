\section{La magie Divine}
\subsection{Mémoriser ses sorts}\label{muxe9moriser-ses-sorts}

\textbf{Du temps et du repos~:} Un personnage mémorise de nouveaux sorts
après une nuit de sommeil ininterrompue. La mémorisation de tous les
sorts que le personnage est capable de retenir est un processus qui
prend une heure.

\textbf{Sorts en double~:} Il est possible de mémoriser le même sort
plus d'une fois, tant que le personnage est capable de retenir plusieurs
sorts de ce niveau.


\subsubsection{Mémorisation des sorts}\label{muxe9morisation-des-sorts}

Les lanceurs de sorts divins mémorisent leurs sorts en priant leur dieu.
Un clerc peut choisir n'importe quel sort dans la liste de sa classe, à
condition qu'il soit d'un niveau suffisant pour pouvoir le lancer.

\subsubsection{Sorts réversibles}\label{sorts-ruxe9versibles}

Les sorts divins peuvent être inversés simplement en prononçant les
incantations à l'envers et en faisant les gestes également dans l'ordre
inverse.

\subsubsection{Disgrâce divine}\label{disgruxe2ce-divine}

Les clercs doivent rester fidèles aux principes de leur alignement, de
leur clergé et de leur religion. Si le personnage perd la faveur de sa
divinité, l'arbitre pourra lui imposer des pénalités~; il peut s'agir
d'un malus à l'attaque (--1), d'une réduction en sorts mémorisables, ou
d'une quête périlleuse à accomplir. Pour retrouver la faveur divine, le
personnage devra accomplir un acte de foi au nom de sa divinité (décidé
par l'arbitre), comme faire un don d'or ou d'objets magiques, construire
un temple, convertir de nouveaux adeptes en masse, vaincre un puissant
ennemi de la divinité, etc.

Un clerc peut s'attirer la disgrâce de sa divinité en lançant des sorts
(ou leur version inversée) dont les effets sont contraires à
l'alignement du dieu.

\begin{itemize}
  \item
  \textbf{Personnages loyaux~:} Ne lancent de sorts inversés que dans
  des circonstances désespérées.
  \item
  \textbf{Personnages neutres~:} Préfèrent les sorts normaux ou inversés
  en fonction de la divinité qu'ils servent (aucune divinité n'autorise
  en même temps les sorts normaux ou inversés).
  \item
  \textbf{Personnages chaotiques~:} Lancent généralement leurs sorts
  inversés, et réservent la version normale de ces sorts aux alliés de
  leur religion.
\end{itemize}

\subsection{Lancer de sorts}\label{lancer-de-sorts}

Un sort mémorisé se lance en reproduisant avec soin une série de gestes
des mains tout en prononçant des incantations mystiques.

\textbf{Utilisation unique~:} Lorsqu'un sort est lancé, il s'efface de
la mémoire du personnage et doit être mémorisé à nouveau.

\textbf{Contraintes au lancer de sorts~:} Un lanceur de sort doit avoir
les mains libres et être capable de parler pour que la magie opère. Par
conséquent, il est impossible de lancer un sort lorsqu'on est ligoté,
bâillonné, ou pris dans une zone de silence magique.

\textbf{Champ de vision~:} Sauf indication contraire dans la description
d'un sort, la cible visée (un monstre, un personnage, un objet ou une
zone d'effet spécifique) doit être visible du lanceur de sorts.

\subsection{Effets de sorts}\label{effets-de-sorts}

\textbf{Sélection des cibles~:} Certains sorts affectent plusieurs
cibles, soit par zone, soit par total de Dés de vie. Si la description
du sort ne précise pas comment les cibles sont sélectionnées, l'arbitre
doit décider si elles le sont aléatoirement, par le lanceur de sorts,
etc.

\textbf{Concentration~:} Certains sorts précisent que le lanceur doit se
concentrer pour maintenir les effets magiques. Sauf description
contraire du sort, entreprendre toute autre action ou être distrait (par
exemple, être attaqué) brise cette concentration.

\textbf{Effets de sorts cumulatifs~:} La même capacité (bonus à l'attaque, Classe d'armure, dégâts,
sauvegardes, etc.) ne peut être améliorée plus d'une fois avec un sort. Les sorts qui affectent des
capacités différentes peuvent en revanche sans problème être combinés.
Les sorts peuvent cumuler leurs effets avec ceux des objets magiques.
