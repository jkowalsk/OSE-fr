\chapter{Introduction}

La création de personnage pour Old School Essentials se fait traditionnellement de la manière suivante~:
\begin{enumerate}
  \item Tirez les caractéristiques
  \item Choisissez une classe
  \item Ajustez les caractéristiques
  \item Notez les modificateurs des caractéristiques
  \item Notez les valeurs d'attaque
  \item Notez scores de sauvegarde et capacités de classe
  \item Tirez vos points de vie
  \item Choisissez l'alignement
  \item Notez les langues connues
  \item Achetez votre équipement
  \item Notez votre Classe d'armure
  \item Notez votre niveau et vos XP
  \item Baptisez votre personnage
\end{enumerate}

Ce livret a pour objectif de remplacer les étapes 1 à 3 par une méthode inspîrée des playbooks du jeu \emph{Beyond the Wall and Other Adventures}\footnote{\url{https://www.flatlandgames.com/btw/}}.

Cette approche doit permettre
\begin{itemize}
  \item De fournir un historique basique au personnage
  \item D'assurer des valeurs caractéristiques qui correspondent à la classe choisie
\end{itemize}

\subsection{Création de personnages basée sur l'historique}
Appliquez successivement les étapes suivantes~:
\begin{enumerate}
  \item Déterminez la valeur initiale des caractéristiques~:
        Lancez 1D4+5 pour chacune d'entre elles.
  \item Déterminez les étapes importantes de votre enfance.
        Pour chacune des 3 tables du chapitre \nameref{chilhoud}~:
        \begin{itemize}
          \item Lancez le dé correspondant
          \item Notez l'élément d'historique obtenu
          \item Notez le bonus de caractéristique obtenu
        \end{itemize}
  \item Choisissez votre classe.
%        Les valeurs de caractéristiques que vous avez obtenu à l'étape précédente peuvent orienter votre choix.
  \item Déterminez les étapes importantes de votre formation.
        Référez vous au chapitre correspondant à votre classe (\nameref{warrior}, \nameref{mage}, \nameref{cleric}, \nameref{thief}, \nameref{dwarf}, \nameref{elf} ou \nameref{halfling}).
        Notez le bonus spécifique à chaque classe, puis,
        pour chacune des 4 tables de la formation~:
        \begin{itemize}
          \item Lancez le dé correspondant
          \item Notez l'élément d'historique obtenu
          \item Notez le bonus de caractéristique obtenu
        \end{itemize}
        \textbf{Note~:} Si la somme des bonus pour une caractéristique dépasse 18, ignorez le bonus ou retirez sur la table.
  \item Calculez vos valeurs de caractéristiques en ajoutant les bonus obtenus.
  \item Notez l'équipement obtenu
\end{enumerate}

\ifmulticolEnd
\section*{Notes de conception}
La méthode traditionnelle distribue toutes les caractériques de la même manière, sur une distribution en cloche centrée autour de la valeur 10,5.

La méthode proposée permet de favoriser selon la classe une caractéristique principale et deux secondaires.

\begin{osrtable}{XXXXXXXXX}{1}
Caractéristique  & Guerrier   & Mage     & Clerc    & Voleur   & Nain     & Elfe     & Halfelin & Moyenne \\
Principale       &  FOR       & INT      & SAG      & DEX      & CON      & INT      & DEX      & 13.15          \\
Secondaire       &  CON; DEX  & DEX; SAG & CON; CHA & INT; CHA & FOR; SAG & DEX; SAG & CON; SAG & 12.15          \\
Autres           &            &          &          &          &          &          &          & 10.15 \\
\end{osrtable}
