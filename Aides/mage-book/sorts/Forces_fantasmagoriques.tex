\begin{spell}
\spellcarac{name}{Forces fantasmagoriques}
\spellcarac{level}{2}
\spellcarac{duree}{Concentration}
\spellcarac{portee}{72 m}
\spellcarac{description}{Le lanceur fait apparaître une illusion de son choix à l'intérieur d'une
zone de 6 m3.

\begin{itemize}
	\item \textbf{Types d'illusions}
\begin{itemize}
\item
  \textbf{Un monstre illusoire :} Le lanceur peut lui ordonner
  d'attaquer. Il a alors une CA de 9 {[}10{]} et disparaît s'il est
  touché en combat.
\item
  \textbf{Une attaque illusoire :} Par exemple, une avalanche, un
  éboulement de plafond, un projectile magique, etc. Les cibles qui
  réussissent un jet de sauvegarde contre les sorts ne sont pas
  affectées.
\item
  \textbf{Une scène :} Celle-ci peut changer l'apparence de la zone
  affectée ou faire apparaître quelque chose de nouveau. Si on la
  touche, la scène s'évapore.
\end{itemize}
\item \textbf{Conditions :}
\begin{itemize}
\item
  \textbf{Concentration :} Elle est requise pour maintenir l'illusion.
  Si le lanceur se déplace ou perd sa concentration, le sort prend fin.
\item
  \textbf{Monstres et attaques illusoires :} Ces illusions peuvent
  sembler dangereuses, mais elles n'infligent pas de dégâts réels. Un
  personnage qui semble mourir est en fait inconscient, un personnage
  transformé en pierre est paralysé, etc. De tels effets durent 1d4
  tours.
\item
  \textbf{Illusions issues de l'imagination :} Si le lanceur crée
  l'illusion de quelque chose qu'il n'a jamais vu, l'arbitre pourra
  accorder à la cible un bonus aux jets de sauvegarde applicables.
\end{itemize}
\end{itemize}
}
\end{spell}
