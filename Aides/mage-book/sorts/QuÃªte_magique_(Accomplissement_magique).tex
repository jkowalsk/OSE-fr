\begin{spell}
\spellcarac{name}{Quête magique}
\spellcarac{level}{6}
\spellcarac{duree}{Permanente}
\spellcarac{portee}{9 m}
\spellcarac{description}{Le lanceur ordonne à une cible d'effectuer ou d'éviter d'effectuer une
action spécifique.

\begin{itemize}
\item
  \textbf{Exemples :} Apporter un objet spécifique au lanceur, manger ou
  boire sans retenue, garder une chose secrète.
\item
  \textbf{Tâches impossibles ou mortelles :} La quête magique prescrite
  ne doit pas être impossible ou directement mortelle -- si une telle
  quête magique est lancée, le sort se affecte le lanceur à la place.
\item
  \textbf{Jet de sauvegarde :} La cible peut effectuer un jet de
  sauvegarde contre les sorts, pour éviter de succomber à l'effet du
  sort.
\item
  \textbf{Si le jet de sauvegarde échoue :} La cible doit suivre la
  ligne de conduite stipulée ou subir des malus croissants (et
  éventuellement fatals) déterminés par l'arbitre. Par exemple : malus à
  l'attaque, réductions de caractéristiques, incapacité à mémoriser les
  sorts, tourments physiques et faiblesse, etc.
\end{itemize}

\subsubsection{Inversé : Accomplissement magique}

Peut dissiper un sort de quête magique actif et tout malus encouru. Si
le niveau du lanceur de la quête magique à annuler est d'un niveau plus
élevé que le personnage qui lance l'inversion, il y a un risque que
cette dernière n'ait aucun effet. La probabilité d'échec est de 5 \% par
niveau de différence entre le lanceur et celui qui essaye d'annuler la
quête.
}
\end{spell}
