\section{La magie}
\subsection{Mémoriser ses sorts}\label{muxe9moriser-ses-sorts}

\textbf{Du temps et du repos~:} Un personnage mémorise de nouveaux sorts
après une nuit de sommeil ininterrompue. La mémorisation de tous les
sorts que le personnage est capable de retenir est un processus qui
prend une heure.

\textbf{Sorts en double~:} Il est possible de mémoriser le même sort
plus d'une fois, tant que le personnage est capable de retenir plusieurs
sorts de ce niveau.


\subsubsection{Mémorisation des sorts}\label{muxe9morisation-des-sorts}

Les lanceurs de sorts arcaniques mémorisent leurs sorts à partir de
livres de sorts (grimoires) et sont donc limités dans leurs choix par le
contenu de leur livre de sorts, qu'ils doivent avoir avec eux.

\subsubsection{Sorts réversibles}\label{sorts-ruxe9versibles}

Le personnage doit décider s'il mémorise un sort dans sa version normale
ou inversée. S'il est capable de mémoriser plus d'un sort de ce niveau,
il peut retenir les deux versions en même temps.

\subsection{Lancer de sorts}\label{lancer-de-sorts}

Un sort mémorisé se lance en reproduisant avec soin une série de gestes
des mains tout en prononçant des incantations mystiques.

\textbf{Utilisation unique~:} Lorsqu'un sort est lancé, il s'efface de
la mémoire du personnage et doit être mémorisé à nouveau.

\textbf{Contraintes au lancer de sorts~:} Un lanceur de sort doit avoir
les mains libres et être capable de parler pour que la magie opère. Par
conséquent, il est impossible de lancer un sort lorsqu'on est ligoté,
bâillonné, ou pris dans une zone de silence magique.

\textbf{Champ de vision~:} Sauf indication contraire dans la description
d'un sort, la cible visée (un monstre, un personnage, un objet ou une
zone d'effet spécifique) doit être visible du lanceur de sorts.

\subsection{Effets de sorts}\label{effets-de-sorts}

\textbf{Sélection des cibles~:} Certains sorts affectent plusieurs
cibles, soit par zone, soit par total de Dés de vie. Si la description
du sort ne précise pas comment les cibles sont sélectionnées, l'arbitre
doit décider si elles le sont aléatoirement, par le lanceur de sorts,
etc.

\textbf{Concentration~:} Certains sorts précisent que le lanceur doit se
concentrer pour maintenir les effets magiques. Sauf description
contraire du sort, entreprendre toute autre action ou être distrait (par
exemple, être attaqué) brise cette concentration.

\textbf{Effets de sorts cumulatifs~:}  La même capacité (bonus à l'attaque, Classe d'armure, dégâts,
sauvegardes, etc.) ne peut être améliorée plus d'une fois avec un sort.
Les sorts qui affectent des capacités différentes peuvent en revanche sans problème être combinés.
Les sorts peuvent cumuler leurs effets avec ceux des objets magiques.


