\begin{spell}
\spellcarac{name}{Rappel à la vie}
\spellcarac{level}{5}
\spellcarac{duree}{Instantanée}
\spellcarac{portee}{36 m}
\spellcarac{description}{Ce sort a deux usages :

\begin{enumerate}
\def\labelenumi{\arabic{enumi}.}
\item
  \textbf{Résurrection :} Cible un humain ou semi-humain récemment
  décédé. Voir description.
\item
  \textbf{Destruction des morts-vivants :} Un unique monstre mort-vivant
  est détruit, s'il rate son jet de sauvegarde contre les sorts.
\end{enumerate}

\textbf{Ressusciter :} Lorsque le sort est utilisé pour une
résurrection, les conditions suivantes s'appliquent :

\begin{itemize}
\item
  \textbf{Limite de temps :} Le lanceur peut rappeler à la vie une
  personne qui est morte depuis un nombre de jours maximum égal à 4
  jours par niveau du lanceur au-dessus du 7e niveau. Par exemple, un
  lanceur de 10e niveau peut rappeler à la vie un personnage mort depuis
  12 jours maximum (3 niveaux au-dessus du 7e x 4 jours).
\item
  '`'Faiblesse :' Revenir d'entre les morts est une épreuve. Tant que la
  cible ne s'est pas reposée durant deux semaines pleines dans un lit,
  elle possède seulement 1 point de vie, se meut à la moitié de son
  déplacement, ne peut porter d'objets lourds, ne peut attaquer, lancer
  de sorts, ou utiliser des capacités de classes. Cette période de
  faiblesse ne peut pas être réduite grâce à des soins magiques.
\end{itemize}

\subsection{Inversé : Doigt de mort}

Dirige un rayon de magie mortelle sur une cible unique. Si cette
dernière rate son jet de sauvegarde contre la mort, elle meurt
instantanément. Notez que lancer le sort doigt de mort est un acte
chaotique, qui ne peut être utilisé par des personnages loyaux qu'en cas
de situation absolument désespérée.
}
\end{spell}
