%   COMMANDS ZUM ZUSAMMENBAUEN DER KARTEN
%   ---------------------------------------

%   TikZ/PGF Settings für die Karten
\pgfmathsetmacro{\cardwidth}{7.5}
\pgfmathsetmacro{\cardheight}{8.88}
\pgfmathsetmacro{\imagewidth}{7.35}
\pgfmathsetmacro{\imageheight}{8.73}
\pgfmathsetmacro{\stripwidth}{0.9}
\pgfmathsetmacro{\strippadding}{0.2}
\pgfmathsetmacro{\textpadding}{0.1}
\pgfmathsetmacro{\titley}{\cardheight-\strippadding-2*\textpadding}
\pgfmathsetmacro{\techy}{\cardheight-\strippadding-5*\textpadding}
\pgfmathsetmacro{\titleheight}{\cardheight-\strippadding-1}

%   Formen der einzelnen Kartenelemente/-bestandteile
\def\shapeCard{(0,0) rectangle (\cardwidth,\cardheight)}
\def\shapeLeftStripLong{(\strippadding,-0.2) rectangle (\strippadding+\stripwidth,\cardheight-\strippadding-\strippadding-1)}
\def\shapeLeftStripShort{(\strippadding,\cardheight-\strippadding-1) rectangle (\strippadding+\stripwidth,\cardheight+0.2)}
\def\shapeRightStripShort{(\cardwidth-\stripwidth-\strippadding,\titleheight) rectangle (\cardwidth-\strippadding,\cardheight+0.2)}
\def\shapeTitleArea{(2*\strippadding+\stripwidth,\cardheight-\strippadding) rectangle (\cardwidth-2*\strippadding-\stripwidth,\titleheight)}
\def\shapeContentArea{(2*\strippadding+\stripwidth,0.5*\cardheight) rectangle (\cardwidth+0.2,-0.2)}


%   Stylings für die Elemente definieren
\tikzset{
  %   runde Ecken für die Karten
  cardcorners/.style={
      rounded corners=0.2cm
    },
  %   runde Ecken für die "Fähnchen"
  elementcorners/.style={
      rounded corners=0.1cm
    },
  %   Schlagschatten für die "Fähnchen"
  stripshadow/.style={
      drop shadow={
          opacity=.25,
          color=black
        }
    },
  %   Style für die "Fähnchen"
  strip/.style={
      elementcorners,
      stripshadow
    },
  %   Bild für das Kartenmotiv
  cardimage/.style={
      path picture={
	      \fill[elementcorners,black] \shapeCard;
          \node at (0.5*\cardwidth,0.5*\cardheight) {
            \includegraphics[width=\imagewidth cm, height=\imageheight cm]{#1}
          };
        }
    },
}

%   TikZ-Raster
\newcommand{\carddebug}{
  \draw [step=1,help lines] (0,0) grid (\cardwidth,\cardheight);
}

%   Rahmen der Karte
\newcommand{\cardborder}{
  \draw[lightgray,cardcorners] \shapeCard;
}

%   Hintergrund der Karte
\newcommand{\cardbackground}[1]{
  \draw[cardcorners, cardimage=#1] \shapeCard;
}

%   Kategorie der Karte
\newcommand{\cardtype}[3]{
  %   First we fill the intersecting area
  %   The \clip command does not allow options, therefore
  %   we have to use a scope to set the even odd rule.
  \begin{scope}[even odd rule]
    %   Define a clipping path. All paths outside shapeCard will
    %   be cut because the even odd rule is set.
    \clip[cardcorners] \shapeCard;
    % Fill shapeLeftStripLong and shapeLeftStripShort.
    %   Since the even odd rule is set, only the card will be filled.
    \fill[strip, #1] \shapeLeftStripLong node[rotate=90,above left=0.9mm,font=\Large\scshape] {
       \color{white}#2
    };
    \fill[strip,#1] \shapeLeftStripShort;
  \end{scope}

  \node at (\strippadding+\stripwidth-0.28,\cardheight-\strippadding-\strippadding-0.37) {\color{white}#3};
}
\newcommand{\cardtypeCharacter}{\cardtype{characterbg}{Charaktereigenschaft}{\hspace{-1mm}\LARGE\lefthand}}
\newcommand{\cardtypeAbility}{\cardtype{abilitybg}{Fähigkeit}{\hspace{-1mm}\Large\floweroneright}}
\newcommand{\cardtypeItem}{\cardtype{itembg}{Gegenstand}{\hspace{-1mm}\LARGE\bomb}}
\newcommand{\cardtypeTest}{\cardtype{testbg}{Testkarte}{\hspace{-1.4mm}\huge\ding{78}}}

%   Titel der Karte
\newcommand{\cardtitle}[1]{
	%\fill[elementcorners,titlebg,opacity=.85] \shapeTitleArea;
	\node[text width=4.75cm] at (0.5*\cardwidth,\titley) {
		\begin{center}
		{\normalfont\scshape\large\bfseries\color{maincolor!60 !black}#1}
		\end{center}
	};
}

\newcommand{\leftpunsh} {
	\node[circle, draw] at (2*\strippadding, \cardheight-2*\strippadding) {};
}

\newcommand{\rightpunsh} {
	\node[circle, draw] at (\cardwidth-2*\strippadding, \cardheight-2*\strippadding) {};
}

\newcommand{\cardtech}[2]{
	\node[text width=4.75cm, fill=black, opacity=0.33, text opacity=1, elementcorners] at (0.5*\cardwidth,\techy) {
		\small
		\begin{itemize}
			\item \textbf{Durée :} #1
			\item \textbf{Portée :} #2
		\end{itemize}
	};
}

%   Inhalt der Karte
\newcommand{\cardcontent}[1]{
  \begin{scope}[even odd rule]
    \clip[cardcorners] \shapeCard;
    %\fill[elementcorners,contentbg] \shapeContentArea;
  \end{scope}
  \node[text width=(\cardwidth-4*\textpadding-1)*1cm] at (0.5*\cardwidth+0.5,0.5*\cardheight-5*\textpadding) {
      \begin{fitbox}{6.1cm}{6.5cm}%
           #1%
      \end{fitbox}%
  	%\parbox[c][6.88cm][c]{6cm}{
	%    \tiny #1
    %}
  };
%  \node[below right, text width=(\cardwidth-2*\strippadding-\stripwidth-2*\textpadding-0.3)*1cm] at (2*\strippadding+\stripwidth+\textpadding,3) {
%    \vrule width \textwidth height 2pt \\[-2pt]
%    \vspace{0.2cm}
%    {\scriptsize #1}
%  };
}

%   Preis der Karte
\newcommand{\cardlevel}[1]{
  \begin{scope}[even odd rule]
    \clip[cardcorners] \shapeCard;
    \fill[strip,\levelBgColor{#1}] \shapeRightStripShort;
  \end{scope}
  \node at (\cardwidth-0.5*\stripwidth-\strippadding,\titley) {\bfseries\scriptsize\levelFgColor{#1}Niveau};
  \node at (\cardwidth-0.5*\stripwidth-\strippadding,\titley-0.3) {\bfseries\levelFgColor{#1}#1};
}

\newcommand{\cardtypeback}{
	\node at (0.5*\cardwidth,0.5*\cardheight) {\resizebox{3cm}{!}{\color{\titlecolor}\classicon} };
}