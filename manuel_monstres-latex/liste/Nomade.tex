\begin{monster}
\monstercarac{img}{Nomad.jpg}
\monstercarac{name}{Nomade}
\monstercarac{description}{
  Tribus superstitieuses qui errent dans les steppes et les régions
désertiques. Elles vivent dans des tentes ou des huttes temporaires.
Leur comportement dépend de la tribu : certaines sont belliqueuses,
d'autres pacifiques.

}
\monsterdetail{Monté}{À \href{Cheval.md\#Cheval-de-selle}{cheval de selle} ou à dos de
\href{Chameau.md}{chameau} (dans le désert).
}
\monsterdetail{Armes (désert)}{50 \% du groupe est équipé avec : une armure de cuir, un bouclier, une
lance ; 30 \% est équipé avec : une cotte de mailles, un bouclier, une
lance ; 20 \% est équipé avec : une armure de cuir, un arc court.
}
\monsterdetail{Armes (steppes)}{50 \% du groupe est équipé avec : une armure de cuir, un arc court ; 20
\% est équipé avec : une armure de cuir, un bouclier, une lance ; 20 \%
est équipé avec : une cotte de mailles, un arc court ; 10 \% est équipé
avec : une cotte de mailles, un bouclier, une lance et peut être monté
sur un cheval de guerre.
}
\monsterdetail{Chefs}{Par groupe de 25 nomades, on trouve un guerrier de 2e niveau. Par groupe
de 40 nomades, un guerrier de 4e niveau.
}
\monsterdetail{Camps}{Les groupes de chasseurs / cueilleurs s'unissent et vivent généralement
dans le cadre d'une tribu comptabilisant jusqu'à 300 guerriers nomades.
}
\monsterdetail{Chefs de camp}{Par groupe de 100 nomades, on trouve un chef de tribu (guerrier de 8e
niveau), plus un guerrier de 5e niveau. 50 \% de chances qu'il y ait un
clerc (9e niveau) ; 25 \% de chances qu'il y ait un magicien (8e
niveau).
}
\monsterdetail{Butin}{Ne possèdent qu'un trésor de type A lorsqu'on les rencontre dans le
camp.
}
\monsterdetail{Commerçants}{Ils acceptent souvent de raconter des récits de pays lointains dont ils
ont entendu parler sur les routes commerciales.
}
\monstercarac{ca}{7 [12] à 4}
\monstercarac{hd}{1 (4 pv)}
\monstercarac{taco}{19 [0]}
\monstercarac{moral}{8}
\monstercarac{alignement}{Tous}
\monstercarac{xp}{10}
\monstercarac{nombre_donjon}{0}
\monstercarac{nombre_exterieur}{1d4 × 10}
\monstercarac{tresor}{A}
\monstercarac{mvt}{36 m (12 m)}
\monstercarac{save_mort_poison}{12}
\monstercarac{save_baguettes}{13}
\monstercarac{save_paralysie_petrification}{14}
\monstercarac{save_souffles}{15}
\monstercarac{save_sorts_sceptres_batons}{16}
\monsterattack{1 × arme }{1d6 selon l’arme}
\end{monster}
