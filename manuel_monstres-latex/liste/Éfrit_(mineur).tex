\begin{monster}
\monstercarac{img}{Efreeti-dnd.png}
\monstercarac{name}{Éfrit (mineur)}
\monstercarac{description}{
}
\monsterdetail{Immunité aux dégâts normaux}{Ne peut être blessé que par des attaques magiques.
}
\monsterdetail{Pouvoirs magiques}{Chacun peut être utilisé trois fois par jour : 1. \textbf{Pilier de
flamme :} Transforme l'éfrit en une colonne de feu pour un maximum de 3
tours. Les objets inflammables se trouvant à moins de 1,50 m prennent
feu. Les attaques infligent 1d8 dégâts supplémentaires (3d8 au total).
2. \textbf{Invisibilité} 3. \textbf{Illusion :} Visuelle et auditive.
Aucune concentration n'est requise. L'illusion persiste jusqu'à ce
qu'elle soit touchée ou dissipée. 4. \textbf{Création d'un mur de feu}
5. \textbf{Création de nourriture et de boisson :} Pour 12 humains et
leurs montures pendant un jour. 6. \textbf{Invoquer des objets
métalliques :} Jusqu'à un poids de 1 000 pièces. Temporaires : Le type
de métal choisi détermine la durée (or : 1 jour ; fer : 1 round). 7.
\textbf{Invoquer des objets usuels / en bois :} Jusqu'à un poids de 1
000 pièces. Permanents.
}
\monsterdetail{Capacité de charge}{Peut porter jusqu'à 10 000 pièces (en volant).
}
\monsterdetail{Haine des Djinns}{Attaquent à vue.
}
\monsterdetail{Serviteur lié}{Les sorts de convocation d'éfrit sont très recherchés par les magiciens
de haut niveau. Peut être lié pour servir pour une durée de 101 jours.
L'éfrit, fourbe, suivra les ordres à la lettre tout en en inversant
l'intention.
}
\monstercarac{ca}{3 [16]}
\monstercarac{hd}{10* (45 pv)}
\monstercarac{taco}{11 [+8]}
\monstercarac{moral}{12}
\monstercarac{alignement}{Chaotique}
\monstercarac{xp}{1 600}
\monstercarac{nombre_donjon}{1}
\monstercarac{nombre_exterieur}{1}
\monstercarac{tresor}{Aucun}
\monstercarac{mvt}{27 m (9 m) / 72 m (24 m) vol}
\monstercarac{save_mort_poison}{4}
\monstercarac{save_baguettes}{5}
\monstercarac{save_paralysie_petrification}{6}
\monstercarac{save_souffles}{5}
\monstercarac{save_sorts_sceptres_batons}{8}
\monsterattack{1 × poing }{2d8}
\monsterattack{magie}{}
\end{monster}
