\begin{monster}
\monstercarac{img}{placeholder.jpg}
\monstercarac{name}{Génie des eaux}
\monstercarac{description}{
  Ces esprits, hauts de 1 m, prennent la forme de femmes attirantes à la
peau bleuâtre, verdâtre ou grisâtre. Ils vivent dans les parties les
plus profondes des rivières et des lacs.

}
\monsterdetail{Timide}{Essaye de charmer les intrus plutôt que d'engager le combat avec eux.
}
\monsterdetail{Armes}{Dagues et petits tridents (comme des épieux).
}
\monsterdetail{Charme}{10 génies des eaux ensemble peuvent lancer un charme pour enchanter une
victime afin qu'elle les serve pendant un an. Jet de sauvegarde contre
les sorts ou tomber sous le charme et doit : avancer vers les génies des
eaux (en résistant à ceux qui essaient de l'en empêcher) ; les défendre
; et obéir à leurs ordres (à condition de comprendre). Incapable de
lancer de sorts, d'utiliser d'objets magiques, ou de s'en prendre aux
génies des eaux. Tuer les génies des eaux rompt le charme.
}
\monsterdetail{Invocation d’une carpe géante}{Chaque génie des eaux peut invoquer un poisson pour l'aider au combat
(voir \href{/Poissons_géants}{Carpe Géante}).
}
\monsterdetail{Conférer la respiration aquatique}{Peut être lancé sur des esclaves charmés. Dure une journée, puis doit
être relancé.
}
\monstercarac{ca}{7 [12]}
\monstercarac{hd}{1 (4 pv)}
\monstercarac{taco}{19 [0]}
\monstercarac{moral}{6}
\monstercarac{alignement}{Neutre}
\monstercarac{xp}{10}
\monstercarac{nombre_donjon}{0}
\monstercarac{nombre_exterieur}{2d20}
\monstercarac{tresor}{B}
\monstercarac{mvt}{36 m (12 m)}
\monstercarac{save_mort_poison}{12}
\monstercarac{save_baguettes}{13}
\monstercarac{save_paralysie_petrification}{13}
\monstercarac{save_souffles}{15}
\monstercarac{save_sorts_sceptres_batons}{15}
\monsterattack{1 × arme }{1d4}
\monsterattack{1 × sort de groupe }{charme}
\end{monster}
