\begin{monster}
\monstercarac{name}{Sanglier-garou}
\monstercarac{description}{
  Changeurs de forme capables d'alterner entre leur apparence humaine et
animale.

  Semi-intelligent et irascible, on le confond souvent sous sa forme
humaine avec un berserker.

}
\monsterdetail{Forme humaine}{Possède des caractéristiques physiques qui rappellent sa nature animale.
}
\monsterdetail{Immunité aux dégâts normaux}{Sous sa forme animale, ne peut être blessé que par les armes en argent
ou la magie.
}
\monsterdetail{Langues}{Sous sa forme humaine, il parle normalement, mais sous forme animale, il
ne peut communiquer qu'avec les animaux du même type.
}
\monsterdetail{Armure}{Les armures gênent la transformation d'un lycanthrope ; il n'en porte
jamais.
}
\monsterdetail{Appel d’animaux}{Peut convoquer 1 ou 2 animaux de son espèce présents dans les alentours
(les rats-garous appellent des rats géants --- voir p.~XX). Ils arrivent
en 1d4 rounds.
}
\monsterdetail{Herbe au loup}{Un lycanthrope touché par de l'herbe au loup doit effectuer un jet de
sauvegarde contre le poison ou s'enfuir, terrorisé.
}
\monsterdetail{Retour}{À sa mort, le lycanthrope reprend sa forme humaine.
}
\monsterdetail{Odeur}{Les chevaux et certains autres animaux sentent les lycanthropes et
peuvent prendre peur.
}
\monsterdetail{Infection}{Un personnage qui perd plus de la moitié de ses points de vie à cause
des attaques naturelles d'un lycanthrope (morsure, griffes\ldots)
contracte la lycanthropie. Les humains deviennent des créatures-garous
du même type et passent sous le contrôle de l'arbitre ; les non-humains
meurent. La maladie se déclare après 2d12 jours, des signes d'infection
étant déjà visibles à la moitié de ce temps.
}
\monsterdetail{Rage de bataille}{Sous forme humaine, peut entrer dans une rage folle : bonus de +2 pour
toucher, combat à mort. La rage le pousse parfois à attaquer ses alliés.
}
\monstercarac{ca}{4 [15] (9 [10] sous forme humaine)}
\monstercarac{hd}{4+1* (19 pv)}
\monstercarac{taco}{15 [+4]}
\monstercarac{moral}{9}
\monstercarac{alignement}{Neutre}
\monstercarac{xp}{200}
\monstercarac{nombre_donjon}{1d4}
\monstercarac{nombre_exterieur}{2d4}
\monstercarac{tresor}{C}
\monstercarac{mvt}{45 m (15 m)}
\monstercarac{save_mort_poison}{10}
\monstercarac{save_baguettes}{11}
\monstercarac{save_paralysie_petrification}{12}
\monstercarac{save_souffles}{13}
\monstercarac{save_sorts_sceptres_batons}{14}
\monsterattack{1 × défense / morsure }{2d6}
\end{monster}
