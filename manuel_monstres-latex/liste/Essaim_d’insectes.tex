\begin{monster}
\monstercarac{name}{Essaim d’insectes}
\monstercarac{description}{
  Essaim de nombreux insectes minuscules que l'on peut rencontrer
protégeant leur nid. Ils peuvent également être attirés par la lumière
ou les odeurs inhabituelles.

}
\monsterdetail{Type et locomotion}{Un essaim peut se composer de créatures rampantes (par exemple :
araignées, fourmis, mille-pattes) ou volantes (par exemple : abeilles,
frelons). Certains essaims possèdent les deux types de mouvement (par
exemple : scarabées, sauterelles).
}
\monsterdetail{Taille}{Occupe généralement une surface de 3 m × 9 m.
}
\monsterdetail{Immunité}{Seulement blessé par le feu, le froid extrême, les sortilèges de sommeil
(affectant l'essaim entier), la fumée (qui les fait fuir) ou d'autres
attaques au souhait de l'arbitre.
}
\monsterdetail{Attaque de l’essaim}{Blesse automatiquement tout personnage se trouvant dans la zone de
l'essaim : 2 points de dégâts si l'on porte une armure, 4 sans.
}
\monsterdetail{Repousser}{Les personnages à l'intérieur de l'essaim qui se défendent en
brandissant une arme (ou similaire) subissent la moitié des dégâts de
l'essaim. Une torche brandie blesse l'essaim.
}
\monsterdetail{S’échapper}{En s'échappant de l'essaim, les personnages continuent de subir la
moitié des dégâts jusqu'à ce que 3 rounds soient passés à écraser les
insectes restés attachés.
}
\monsterdetail{Plonger dans l’eau}{Le personnage subit des dégâts pendant un round, puis les insectes
attachés se noient.
}
\monsterdetail{Poursuite}{Un essaim énervé (c'est-à-dire blessé) poursuivra les personnages
jusqu'à ce qu'ils soient hors de vue ou inaccessibles.
}
\monstercarac{ca}{7 [12]}
\monstercarac{hd}{2 to 4 (9/13/18 pv)}
\monstercarac{taco}{18 [+1]/17 [+2]/16 [+3]}
\monstercarac{moral}{11}
\monstercarac{alignement}{Neutre}
\monstercarac{xp}{20/35/75}
\monstercarac{nombre_donjon}{1}
\monstercarac{nombre_exterieur}{1d3}
\monstercarac{tresor}{Aucun}
\monstercarac{mvt}{9 m (3 m) / 18 m (6 m) vol}
\monstercarac{save_mort_poison}{14}
\monstercarac{save_baguettes}{15}
\monstercarac{save_paralysie_petrification}{16}
\monstercarac{save_souffles}{17}
\monstercarac{save_sorts_sceptres_batons}{18}
\monsterattack{1 × essaim }{2 }
\monsterattack{4 pv)}{}
\end{monster}
