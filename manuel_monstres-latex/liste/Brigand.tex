\begin{monster}
\monstercarac{img}{brigand.jpg}
\monstercarac{name}{Brigand}
\monstercarac{description}{
  Hors-la-loi et mercenaire qui vit de pillages et attaque les voyageurs.

}
\monsterdetail{Infanterie}{Une moitié est équipée avec : une armure légère, un bouclier, une épée,
et un arc court.
}
\monsterdetail{Cavalerie}{L'autre moitié monte des chevaux de selle et est équipée avec : une
cotte de mailles, un bouclier, et une épée.
}
\monsterdetail{Chefs et commandants}{Par groupe de 20 brigands, on trouve un chef (guerrier de 2e niveau).
Par groupe de 40 brigands, on trouve un commandant (guerrier de 4e
niveau). Les chefs montent des chevaux de guerre (avec protections
supplémentaires : un caparaçon) et sont équipés avec : une armure de
plates, une épée et une lance.
}
\monsterdetail{Camps fortifiés}{Les bandes se regroupent fréquemment et vivent dans des camps fortifiés
abritant 5d6 × 10 brigands.
}
\monsterdetail{Chefs de camps}{Les camps issus d'un regroupement sont dirigés par un guerrier de 9e
niveau, auquel s'ajoute un guerrier de 5e niveau par groupe de 50
brigands. De plus, il y a 50 \% de chances qu'il s'agisse d'un magicien
(niveau 1d3+8) et 30 \% de chance qu'il s'agisse d'un clerc (8e niveau).
}
\monstercarac{ca}{6 [13]}
\monstercarac{hd}{1 (4 pv)}
\monstercarac{taco}{19 [0]}
\monstercarac{moral}{8}
\monstercarac{alignement}{Chaotique}
\monstercarac{xp}{10}
\monstercarac{nombre_donjon}{0}
\monstercarac{nombre_exterieur}{1d4 × 10}
\monstercarac{tresor}{A}
\monstercarac{mvt}{36 m (12 m)}
\monstercarac{save_mort_poison}{12}
\monstercarac{save_baguettes}{13}
\monstercarac{save_paralysie_petrification}{14}
\monstercarac{save_souffles}{15}
\monstercarac{save_sorts_sceptres_batons}{16}
\monsterattack{1 × arme }{1d6 (selon l’arme)}
\end{monster}
