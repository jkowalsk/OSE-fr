\begin{monster}
\monstercarac{img}{Djinn5E.PNG.png}
\monstercarac{name}{Djinn (mineur)}
\monstercarac{description}{
  Être hautement magique, libre et intelligent, provenant du plan
élémentaire de l'air. Grand humanoïde enveloppé de nuages.

}
\monsterdetail{Immunité aux dégâts normaux}{Ne peut être blessé que par des attaques magiques.
}
\monsterdetail{Pouvoirs magiques}{Chaque pouvoir peut être utilisé trois fois par jour : 1. \textbf{Forme
de tourbillon :} Il faut 5 rounds au Djinn pour se transformer (ou
reprendre une forme humanoïde). Le tourbillon fait 21 m de haut, 6 m de
large à son sommet pour 3 m de large à sa base. Il se déplace à une
vitesse de 36 m (12 m) et inflige 2d6 points de dégâts à toute créature
se trouvant sur son chemin. Les créatures de moins de 2 DV sont balayées
(jet de sauvegarde contre la mort). 2. \textbf{Forme gazeuse} 3.
\textbf{Invisibilité} 4. \textbf{Illusion :} Visuelle et auditive.
L'illusion ne demande aucune concentration au djinn. Elle persiste
jusqu'à ce qu'elle soit touchée ou dissipée. 5. \textbf{Création de
nourriture et de boisson :} Pour 12 humains et leurs montures pendant un
jour. 6. \textbf{Invoquer des objets métalliques :} Jusqu'à un poids de
1 000 pièces. Temporaires : Le type de métal choisi détermine la durée
(or : 1 jour ; fer : 1 round). 7. \textbf{Invoquer des objets usuels /
en bois :} Jusqu'à un poids de 1 000 pièces. Permanents.
}
\monsterdetail{Capacité de charge}{Un djinn peut porter 6 000 pièces sans fatigue. Jusqu'à 12 000 pièces
pendant 3 tours en marchant / 1 tour en volant. Il doit ensuite se
reposer 1 tour.
}
\monsterdetail{S’il est tué}{Il retourne dans le plan de l'air.
}
\monstercarac{ca}{5 [14]}
\monstercarac{hd}{7+1* (32 pv)}
\monstercarac{taco}{12 [+7]}
\monstercarac{moral}{12}
\monstercarac{alignement}{Neutre}
\monstercarac{xp}{850}
\monstercarac{nombre_donjon}{1}
\monstercarac{nombre_exterieur}{1}
\monstercarac{tresor}{Aucun}
\monstercarac{mvt}{27 m (9 m) / 72 m (24 m) vol}
\monstercarac{save_mort_poison}{4}
\monstercarac{save_baguettes}{5}
\monstercarac{save_paralysie_petrification}{6}
\monstercarac{save_souffles}{5}
\monstercarac{save_sorts_sceptres_batons}{8}
\monsterattack{1 × poings }{2d8}
\monsterattack{magie }{-}
\end{monster}
