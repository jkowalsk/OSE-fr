\begin{monster}
\monstercarac{img}{Buccaneer.jpg}
\monstercarac{name}{Boucanier}
\monstercarac{description}{
  Un marin qui vit de raids côtiers et d'abordages de bateaux. Il sillonne
les rivières, les lacs et les côtes, plus rarement les océans.

}
\monsterdetail{Bateaux et équipages}{Ils dépendent de l'endroit où on les rencontre. Sur les rivières et les
lacs : 1d8 embarcations fluviales accueillant chacune 1d2 × 10
boucaniers ; sur les côtes, 1d6 petites galère(s) accueillant chacune
1d3+1 × 10 boucaniers ; partout : 1d4 drakkar(s) accueillant chacun
1d3+2 × 10 boucaniers ; océans : 1d4 navires de guerre accueillant
chacun 1d5+3 × 10 boucaniers (voir
\href{../../Equipement_services/Embarcations.md}{Embarcations} pour les
détails sur les navires).
}
\monsterdetail{Armement}{60 \% du groupe est équipé avec : une armure de cuir, une épée ; 30 \%
est équipé avec : une armure de cuir, une épée, une arbalète ; 10 \% est
équipé avec : une cotte de mailles, une épée, une arbalète.
}
\monsterdetail{Chefs et capitaines}{Par groupe de 30 boucaniers, on trouve un guerrier de 4e niveau. Chaque
navire possède un capitaine (guerrier de 7e niveau).
}
\monsterdetail{Commandant de flotte}{Guerrier de 9e niveau ; 30 \% de chances qu'il s'agisse d'un magicien
(niveau 1d2+9) ; 25 \% de chances qu'il s'agisse d'un clerc (8e niveau).
}
\monsterdetail{Trésor}{Il est réparti entre les différents navires. Au lieu d'être transporté à
bord, il peut y avoir une carte indiquant l'endroit où il est enterré.
}
\monsterdetail{Havres}{Les villes côtières sans loi et fortifiées peuvent servir de havre pour
les boucaniers et les pirates.
}
\monstercarac{ca}{7 [12] or 5 [14]}
\monstercarac{hd}{1 (4 pv)}
\monstercarac{taco}{19[0]}
\monstercarac{moral}{6}
\monstercarac{alignement}{Neutre}
\monstercarac{xp}{10}
\monstercarac{nombre_donjon}{0}
\monstercarac{nombre_exterieur}{voi ci-dessous}
\monstercarac{tresor}{A}
\monstercarac{mvt}{36 m (12 m)}
\monstercarac{save_mort_poison}{12}
\monstercarac{save_baguettes}{13}
\monstercarac{save_paralysie_petrification}{14}
\monstercarac{save_souffles}{15}
\monstercarac{save_sorts_sceptres_batons}{16}
\monsterattack{1 × arme }{1d6 (selon l’arme)}
\end{monster}
