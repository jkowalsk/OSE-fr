\begin{monster}
\monstercarac{name}{Marchand}
\monstercarac{description}{
  Commerçants organisés voyageant entre différents villages dans des
caravanes bien armées, achetant et vendant des biens commerciaux (par
exemple, de l'or, des bijoux, de la soie, des épices, du vin, etc.).

}
\monsterdetail{Armes}{Épée et dague.
}
\monsterdetail{Monté}{À cheval, à dos de chameau ou de mule (selon le terrain).
}
\monsterdetail{Chariots}{2 par marchand. Tirés par des chevaux, des mules ou des chameaux (selon
le terrain).
}
\monsterdetail{Gardes de caravanes}{Pour chaque marchand, 4 gardes sont présents (guerriers de 1er niveau).
CA 4 {[}15{]}, avec des arbalètes, des épées, des poignards.
}
\monsterdetail{Lieutenants}{2 lieutenants sont présents par groupe de 5 marchands (guerriers de 2e
ou 3e niveau). CA 4 {[}15{]}. Équipés comme les gardes de caravanes.
}
\monsterdetail{Capitaine}{Les gardes de caravanes sont dirigés par un guerrier de 5e niveau. CA 4
{[}15{]}. Équipé comme les gardes de caravanes.
}
\monsterdetail{Animaux porteurs}{1d12 chevaux, mules ou chameaux supplémentaires.
}
\monsterdetail{Trésor}{Doit être réduit s'il y a moins de 10 marchands dans le groupe.
}
\monstercarac{ca}{5 [14]}
\monstercarac{hd}{1 (4 pv)}
\monstercarac{taco}{19 [0]}
\monstercarac{moral}{variable}
\monstercarac{alignement}{Neutre}
\monstercarac{xp}{10}
\monstercarac{nombre_donjon}{0}
\monstercarac{nombre_exterieur}{1d20}
\monstercarac{tresor}{A}
\monstercarac{mvt}{27 m (9 m)}
\monstercarac{save_mort_poison}{12}
\monstercarac{save_baguettes}{13}
\monstercarac{save_paralysie_petrification}{14}
\monstercarac{save_souffles}{15}
\monstercarac{save_sorts_sceptres_batons}{16}
\monsterattack{1 × arme }{1d6 }
\monsterattack{selon l’arme)}{}
\end{monster}
