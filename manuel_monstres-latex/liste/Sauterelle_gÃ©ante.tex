\begin{monster}
\monstercarac{img}{placeholder.jpg}
\monstercarac{name}{Sauterelle géante}
\monstercarac{description}{
  Sauterelle géante herbivore, longue de 60 cm à 1 m. Habite dans les
cavernes.

}
\monsterdetail{Se confond avec la pierre}{En raison de sa teinte pierreuse, elle peut passer inaperçue ou être
confondue avec une statue.
}
\monsterdetail{Cri}{Si elle est attaquée ou effrayée, elle crie pour avertir les autres, ce
qui peut attirer des monstres errants (20 \% de chances par tour).
}
\monsterdetail{Saut}{Très craintive. Lorsqu'elle subit une attaque, elle fuit généralement en
faisant un bond de 18 m, puis s'envole. Il y a 50 \% de chances qu'elle
saute sur un adversaire pris au hasard, auquel cas il faut traiter le
saut comme une attaque.
}
\monsterdetail{Crachat}{Utilisé défensivement. Portée de 3 m. On considère que la cible a une CA
9 {[}10{]}. Le personnage affecté est recouvert d'une projection de
matière puante : il est incapable d'agir pendant 1 tour (jet de
sauvegarde contre le poison). Jusqu'à ce que cette matière soit
nettoyée, les personnes approchant à moins de 1,50 m doivent elles aussi
effectuer un jet de sauvegarde contre le poison ou se sentir nauséeuses.
}
\monsterdetail{Immunité aux poisons}{Immunisée contre la \href{Moisissure_jaune.md}{Moisissure jaune} et la
plupart des poisons, en raison de son habitude à manger des champignons.
}
\monstercarac{ca}{4 [15]}
\monstercarac{hd}{2 (9 pv)}
\monstercarac{taco}{18 [+1]}
\monstercarac{moral}{5}
\monstercarac{alignement}{Neutre}
\monstercarac{xp}{20}
\monstercarac{nombre_donjon}{2d10}
\monstercarac{nombre_exterieur}{1d10}
\monstercarac{tresor}{Aucun}
\monstercarac{mvt}{18 m (6 m) / 54 m (18 m) vol}
\monstercarac{save_mort_poison}{12}
\monstercarac{save_baguettes}{13}
\monstercarac{save_paralysie_petrification}{14}
\monstercarac{save_souffles}{15}
\monstercarac{save_sorts_sceptres_batons}{16}
\monsterattack{1 × morsure }{1d2}
\monsterattack{1 × saut }{1d4}
\monsterattack{1 × crachat }{puanteur}
\end{monster}
