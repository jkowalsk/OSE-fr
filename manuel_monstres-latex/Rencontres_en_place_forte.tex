\section*{Rencontres en place forte}\label{rencontres-en-place-forte}

Lorsque les PJ vagabondent à proximité de la place forte d'un PNJ de
haut niveau, un accueil chaleureux n'est pas toujours garanti. L'arbitre
peut utiliser les règles suivantes s'il n'a pas de notes spécifiques sur
le dirigeant d'une place forte et sur les ordres donnés aux patrouilles
de gardes.

\subsection*{Dirigeant}\label{dirigeant}

L'arbitre doit décider de quelle classe est le PNJ qui revendique la
propriété de la place forte et des terres environnantes :

\begin{itemize}
\item
  \textbf{Clerc :} De niveau 1d8+6.
\item
  \textbf{Guerrier :} De niveau 1d6+8.
\item
  \textbf{Magicien :} De niveau 1d4+10.
\end{itemize}

Les places fortes appartenant à des semi-humains sont des cas
relativement rares et doivent être détaillées à l'avance par l'arbitre.
En règle générale, ils tenteront d'éviter tout contact avec des
voyageurs.

\subsection*{Patrouilles}\label{patrouilles}

Les étrangers parcourant les terres autour d'une place forte seront
généralement repérés par des groupes de mercenaires embauchés pour
patrouiller dans les environs. Le type de troupes dépend de la classe du
dirigeant :

\begin{itemize}
\item
  \textbf{Clerc :} 2d6 cavaliers moyens. Équipés de cottes de mailles
  (CA 5 {[}14{]}) et de lances. Moral 9.
\item
  \textbf{Guerrier :} 2d6 cavaliers lourds 2d6. Équipés d'armures de
  plates (CA 3 {[}16{]}), de lances et d'épées. Moral 9.
\item
  \textbf{Magicien :} 2d6 fantassins lourds. Équipés de cottes de
  mailles + bouclier (CA 4 {[}15{]}) et d'épées. Moral 8.
\end{itemize}

\subsection*{Garnison}\label{garnison}

Les patrouilles, telles que décrites ci-dessus, ne sont qu'une petite
partie de la garnison du dirigeant. D'autres forces peuvent inclure des
monstres magiques ou des humains montés sur des créatures volantes.

\subsection*{Réaction aux voyageurs}\label{ruxe9action-aux-voyageurs}

La réaction du dirigeant face aux voyageurs sur son domaine dépend de sa
classe. On peut la déterminer en lançant 1d6 et en consultant ce tableau
:

\begin{table}[H]
	\centering
\begin{tabular}[]{llll}
\titlecell{d6} & \titlecell{Clerc} & \titlecell{Guerrier} & \titlecell{Magicien} \\
1 & Expulsion & Expulsion & Expulsion \\
2 & Expulsion & Expulsion & Ignorance \\
3 & Ignorance & Expulsion & Ignorance \\
4 & Ignorance & Ignorance & Ignorance \\
5 & Invitation & Ignorance & Ignorance \\
6 & Invitation & Invitation & Invitation \\
\end{tabular}
\caption{Réaction du dirigeant selon sa Classe}\label{ruxe9action-du-dirigeant-selon-sa-classe}
\end{table}

\textbf{Expulsion :} La patrouille est chargée d'expulser les intrus
hors du domaine. Alternativement, ils peuvent exiger un péage des
voyageurs de passage. Le prix exact demandé dépend de la personnalité du
dirigeant, de la richesse apparente des PJ, etc. Si les PJ refusent de
payer le péage, la patrouille peut les attaquer, les expulser ou tenter
de les faire prisonniers.

\textbf{Ignorance :} La patrouille quitte les PJ pour aller vaquer à ses
occupations.

\textbf{Invitation :} La patrouille apporte un message du dirigeant du
domaine, invitant les PJ à rester dans la place forte. Le motif exact de
cette décision dépend de sa personnalité --- ce qui n'est pas
nécessairement sans conséquence.
