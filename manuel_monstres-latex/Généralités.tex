\chapter{Généralités}\label{guxe9nuxe9ralituxe9s}

\begin{multicols}{2}[\section*{Statistiques des monstres}\label{statistiques-des-monstres}]

Les descriptions des monstres comportent les valeurs suivantes.

\subsection*{Classe d'armure (CA)}\label{classe-darmure-ca}

La capacité du monstre à éviter les dégâts en combat.

\textbf{CA ascendante :} La Classe d'armure ascendante (CAA,
optionnelle) est notée à la suite entre crochets.

\subsection*{Dés de vie (DV)}\label{duxe9s-de-vie-dv}

Le nombre de d8 lancés pour déterminer les points de vie d'un monstre.

\textbf{Astérisques :} Un ou plusieurs astérisques après le nombre de DV
indique le nombre de capacités spéciales du monstre. On s'en sert pour
calculer sa valeur en XP.

\textbf{Modificateurs :} Les modificateurs aux DV (comme +3 ou --1)
s'appliquent au total de points de vie après le lancer des d8.

\textbf{Fractions de DV :} Certains monstres ont moins d'un DV : soit ½
DV (on lance 1d4), soit un nombre de points de vie fixe.

\textbf{Points de vie moyens :} La moyenne de points de vie pour le
nombre de DV est notée entre parenthèses.

\subsection*{Attaques par round (Att)}\label{attaques-par-round-att}

Les attaques que le monstre peut effectuer à chaque round, avec les
dégâts entre parenthèses. Notez que, sauf indication contraire, les
attaques et les dégâts des monstres ne sont pas modifiés par la FOR ou
la DEX.

\textbf{Attaques alternatives :} On trouve entre crochets les séries
d'attaques alternatives que le monstre peut choisir d'utiliser.

\subsection*{Scores de sauvegarde (Sv)}\label{scores-de-sauvegarde-sv}

Les valeurs à atteindre pour les jets de sauvegarde du monstre :

\begin{itemize}
\tightlist
\item
  \textbf{MP :} Mort et Poison
\item
  \textbf{B :} Baguettes
\item
  \textbf{PP :} Paralysie et Pétrification
\item
  \emph{'S :} Souffles
\item
  \textbf{SSB :} Sorts, Sceptres et Bâtons
\end{itemize}

\textbf{Sauvegarde et DV :} Tous les monstres n'ont pas les mêmes scores
de sauvegarde, aussi le nombre de DV auxquels les scores indiqués
correspondent est précisé entre parenthèses (HN signifie que le monstre
se sauvegarde comme un Humain normal). Ce chiffre n'est pas toujours
égal au nombre de DV du monstre. En général, les monstres d'intelligence
inférieure ont un score de sauvegarde comme s'ils avaient moitié moins
de DV, et les monstres magiques comme s'ils avaient un ou plusieurs DV
de plus.

Certains monstres ont les mêmes jets de sauvegarde qu'une classe de
personnage. Dans ce cas, la classe et le niveau sont notés entre
parenthèse après les scores : C = clerc, E = elfe, G = guerrier, M =
magicien, N = nain, T = tinigens, V = voleur.

\subsection*{Jet d'attaque « Toucher une Armure de Classe de 0 » (TAC0)}\label{jet-dattaque-toucher-une-armure-de-classe-de-0-tac0}

La capacité d'un monstre à porter des coups en combat est déterminée par
ses DV.

\textbf{Bonus d'attaque :} Ce chiffre, nécessaire pour les arbitres
ayant adopté la règle de Classe d'armure ascendante, est noté entre
crochets.

\subsection*{Déplacement (DP)}\label{duxe9placement-dp}

La vitesse à laquelle le monstre se déplace. Chaque monstre a un
déplacement de base et un déplacement de rencontre (noté entre
parenthèses) égal au tiers du déplacement de base.

\textbf{Modes de déplacement :} Si le monstre peut se déplacer de
plusieurs manières (par exemple, en marchant, en volant, ou en
escaladant), les vitesses sont séparées par des barres obliques ( / ).

\subsection*{Score de moral (Ml)}\label{score-de-moral-ml}

La probabilité qu'un monstre persévère en combat.

\subsection*{Alignement (Al)}\label{alignement-al}

L'affiliation d'un monstre à la Loi, à la Neutralité ou au Chaos. S'il
est indiqué « tous », l'arbitre peut tirer au hasard ou décider lui-même
de l'alignement du monstre.

\subsection*{Valeur en XP (XP)}\label{valeur-en-xp-xp}

Les points d'expérience reçus par le groupe pour avoir vaincu le
monstre.

\subsection*{Nombre apparaissant (NA)}\label{nombre-apparaissant-na}

On trouve ici deux valeurs, la seconde étant entre parenthèses.

\textbf{Zéros :} Si la première valeur indique 0, les monstres de ce
type ne se trouvent en général pas dans les donjons. Si la deuxième
valeur est 0, ils ne se trouvent pas en extérieur et n'ont normalement
pas de repaire.

\textbf{Utilisation :} L'utilisation de ces valeurs dépend de la
situation dans laquelle les monstres sont rencontrés :

\begin{itemize}
\tightlist
\item
  \textbf{Monstres errants dans un donjon :} La première valeur
  détermine le nombre de monstres rencontrés dans un niveau de donjon
  égal à leur DV. Si le monstre est rencontré à un niveau supérieur à
  son DV, le nombre apparaissant peut être augmenté ; s'il est rencontré
  à un niveau inférieur à son DV, le nombre apparaissant devrait être
  réduit.
\item
  \textbf{Repaire des monstres dans un donjon :} La deuxième valeur
  répertorie le nombre de monstres trouvés dans un repaire dans un
  donjon.
\item 
	\textbf{Monstres errants dans les contrées sauvages :} La deuxième valeur
	indique le nombre de monstres rencontrés errant dans les contrées
	sauvages.
\item
  \textbf{Repaire des monstres dans les contrées sauvages :} La deuxième
  valeur multipliée par 5 indique le nombre de monstres trouvés dans un
  repaire dans les contrées sauvages.
\end{itemize}

\subsection*{Type de trésor (TT)}\label{type-de-truxe9sor-tt}

Le code de lettre utilisé pour déterminer la quantité et le type de
trésor possédés par le(s) monstre(s). Les lettres
citées sont utilisées comme suit :

\begin{itemize}
\tightlist
\item
  \textbf{De A à O :} Indique un trésor : la somme des richesses d'un
  grand monstre ou d'une communauté de petits monstres, généralement
  cachée dans son repaire. Pour les monstres qui apparaissent avec un
  nombre aléatoire supérieur à 1d4 (voir Nombre apparaissant), la
  quantité de trésor peut être réduite si le nombre de monstres était
  inférieur à la moyenne.
\item
  \textbf{De P à V :} Indique un trésor contenu dans les affaires d'un
  monstre intelligent (P à T) ou dans celles d'un groupe de monstres
  intelligents (U, V). Si un monstre dépourvu d'intelligence possède
  l'un de ces trésors, il se trouve sur les cadavres de ses victimes.
\end{itemize}
\end{multicols}

\begin{multicols}{2}[\section*{Notes}\label{notes-guxe9nuxe9rales}]

\subsection*{Infravision}\label{infravision}

Tous les monstres non-humains sont dotés de l'infravision. À moins
d'une indication contraire dans la description d'un monstre, il voit
dans le noir jusqu'à 18 m.

\subsection*{Langues}\label{langues}

20 \% des monstres intelligents parlent le commun, sauf indication contraire
dans leur description. De nombreuses espèces intelligentes ont aussi
leur propre langue.

\subsection*{Personnes}\label{personnes}

Certains sorts et effets magiques n'affectent que les « personnes ».
Cette catégorie inclut tous les humains et semi-humains, de même que les
monstres humanoïdes de 4+1 HD ou moins. L'arbitre décide si d'autres
monstres peuvent être considérés comme des personnes.

Les monstres suivants sont considérés comme des « Personnes » pour les
effets magiques qui peuvent les affecter : acolyte, bandit, berserker,
boucanier, brigand, commerçant, derviche, elfe, esprit follet, génie des
eaux, gnoll, gnome, gobelin, goblours, hobgobelin, homme-lézard,
homme-poisson, humain normal, kobold, lutin, marchand, médium, devin,
nain, néandertalien, négociant, noble, ogre, orque, pirate, pixie,
tinigens, troglodyte, vétéran.
\end{multicols}