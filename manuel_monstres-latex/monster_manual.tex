\documentclass[10pt]{report}
\usepackage{vanilla}
\usepackage[utf8]{inputenc}
\usepackage[T1]{fontenc}

%\usepackage{amsmath}
%\usepackage{amsfonts}
%\usepackage{amssymb}
%\usepackage{graphicx}
%\usepackage[space]{grffile} % espace dans les noms de fichier image
%\usepackage{wrapfig}
%\usepackage{multirow}
%\usepackage{float}
%\usepackage{multicol}
%\usepackage{hyperref}
%\usepackage[most]{tcolorbox}
\usepackage{xinttools}
\usepackage{etoolbox}
\usepackage{titletoc}
\usepackage{float}


\makeatletter

\global\@namedef{monstercarac@list}{}
\global\@namedef{monsterdetail@list}{}
\global\@namedef{monsterdetailname@list}{}
\global\@namedef{monsterattack@list}{}
\global\@namedef{monsterattackname@list}{}

\newcommand{\resetcaraclist}{
	\xintFor* ##1 in \monstercarac@list\do {
		\global\@namedef{##1}{}
	}
	\global\@namedef{monstercarac@list}{}
}

\newcommand{\resetdetaillist}{
	\xintFor* ##1 in \monsterdetail@list\do {
		\global\@namedef{##1}{}
		\global\@namedef{name::##1}{}
	}
	\global\@namedef{monsterdetail@list}{}
    \global\@namedef{monsterdetailname@list}{}
}

\newcommand{\resetattacklist}{
	\xintFor* ##1 in \monsterattack@list\do {
		\global\@namedef{##1}{}
		\global\@namedef{name::##1}{}
	}
	\global\@namedef{monsterattack@list}{}
	\global\@namedef{monsterattackname@list}{}
}

\newcommand{\monstercarac}[2]{
	\global\@namedef{#1}{#2}
	\g@addto@macro\monstercarac@list{{#1}}
}

\newcommand{\monsterdetail}[2]{
	\global\@namedef{#1}{#2}
	\global\@namedef{name::#1}{#1}
	\g@addto@macro\monsterdetail@list{{#1}}
    \g@addto@macro\monsterdetailname@list{{#1}}
}

\newcommand{\monsterattack}[2]{
	\global\@namedef{#1}{#2}
	\global\@namedef{name::#1}{#1}
	\g@addto@macro\monsterattack@list{{#1}}
	\g@addto@macro\monsterattackname@list{{#1}}
}

\newcommand{\printcarac}[2]{
	#2 : \@nameuse{#1}
}

\newcommand{\teststring}[2]{% or \def\teststring#1#2{
		\edef\@teststringA{#1}%
		\edef\@teststringB{#2}%
		\ifx\@teststringA\@teststringB
		\expandafter\@firstoftwo
		\else
		\expandafter\@secondoftwo
		\fi
	}

\def\forcetwocols{Vampire}% special too long details

\newcommand{\printmonster}{
	\clearpage
	\chapter{\@nameuse{name}}
	\vspace{-1cm}
	\begin{multicols}{2}[]
	\section*{Description}
	\@nameuse{description}
	\section*{Attaques}
	\begin{itemize}
		\xintFor* ##1 in \monsterattack@list\do {
			\item \textbf{\@nameuse{name::##1} :} \@nameuse{##1}
		}
	\end{itemize}
	\vfill\null
	\columnbreak
	\raggedright
	\begin{tabular}{lp{0.7\columnwidth}}
		\hline
		\textbf{Classe d'armure} & \@nameuse{ca} \\
		\textbf{Dés de vie }	& \@nameuse{hd} \\
		\textbf{TAC0} 	& \@nameuse{taco} \\
		\textbf{Mouvement} 	& \@nameuse{mvt} \\
		\textbf{Sauvegardes} & MP \@nameuse{save_mort_poison} B \@nameuse{save_baguettes} PP \@nameuse{save_paralysie_petrification} S \@nameuse{save_souffles} SSB \@nameuse{save_sorts_sceptres_batons}\\
		\textbf{Moral} & \@nameuse{moral} \\
		\textbf{Alignement} 	& \@nameuse{alignement} \\
		\textbf{XP} 	& \@nameuse{xp} \\
		\textbf{Nombre} & \@nameuse{nombre_donjon} (\@nameuse{nombre_exterieur}) \\
		\textbf{Type de trésor} & \@nameuse{tresor} \\
		\hline
	\end{tabular}

  	\end{multicols}
	\ifx\monsterdetail@list\@empty
	%%
	\else
	  \teststring{Vampire}{\@nameuse{name}}{
		\begin{multicols*}{2}[\section*{Détails}\vspace{-0.5cm}]
		}{\section*{Détails}}
	\begin{itemize}
		\xintFor* ##1 in \monsterdetail@list\do {
		\item \textbf{\@nameuse{name::##1} :} \@nameuse{##1}
		}
	\end{itemize}
	\teststring{Vampire}{\@nameuse{name}}{\end{multicols*}}{}
	\fi
}

\newenvironment{monster}{
	\resetcaraclist
	\resetdetaillist
	\resetattacklist
}{
	\printmonster
	\resetcaraclist
	\resetdetaillist
	\resetattacklist
}

\renewcommand\tableofcontents{%
	\@starttoc{toc}%
}

\makeatother

\hypersetup{colorlinks=false, hidelinks}

\title{Manuel des monstres}
\subtitle{Old School Essentials}
\author{jki}
%% https://mobile.twitter.com/FreeRPGImagery/status/1227634112778817536/photo/1
\setcoverimage{1.0}{cover_midjourney.png}

\topskip0pt
\def\tightlist{}

\begin{document}
\maketitle

\titlecontents{chapter}[0em]
{\small}%
{}% numbered sections formatting
{}% unnumbered sections formatting
{\titlerule*{.}\contentspage}%
\renewcommand{\baselinestretch}{0.5}\normalsize
\renewcommand{\contentsname}{Sommaire}
\begin{multicols}{4}[\section*{\contentsname}]
\tableofcontents
\end{multicols}
\renewcommand{\baselinestretch}{1.0}\normalsize

\part{Points de règles}
\chapter{Généralités}\label{guxe9nuxe9ralituxe9s}

\begin{multicols}{2}[\section*{Statistiques des monstres}\label{statistiques-des-monstres}]

Les descriptions des monstres comportent les valeurs suivantes.

\subsection*{Classe d'armure (CA)}\label{classe-darmure-ca}

La capacité du monstre à éviter les dégâts en combat.

\textbf{CA ascendante :} La Classe d'armure ascendante (CAA,
optionnelle) est notée à la suite entre crochets.

\subsection*{Dés de vie (DV)}\label{duxe9s-de-vie-dv}

Le nombre de d8 lancés pour déterminer les points de vie d'un monstre.

\textbf{Astérisques :} Un ou plusieurs astérisques après le nombre de DV
indique le nombre de capacités spéciales du monstre. On s'en sert pour
calculer sa valeur en XP.

\textbf{Modificateurs :} Les modificateurs aux DV (comme +3 ou --1)
s'appliquent au total de points de vie après le lancer des d8.

\textbf{Fractions de DV :} Certains monstres ont moins d'un DV : soit ½
DV (on lance 1d4), soit un nombre de points de vie fixe.

\textbf{Points de vie moyens :} La moyenne de points de vie pour le
nombre de DV est notée entre parenthèses.

\subsection*{Attaques par round (Att)}\label{attaques-par-round-att}

Les attaques que le monstre peut effectuer à chaque round, avec les
dégâts entre parenthèses. Notez que, sauf indication contraire, les
attaques et les dégâts des monstres ne sont pas modifiés par la FOR ou
la DEX.

\textbf{Attaques alternatives :} On trouve entre crochets les séries
d'attaques alternatives que le monstre peut choisir d'utiliser.

\subsection*{Scores de sauvegarde (Sv)}\label{scores-de-sauvegarde-sv}

Les valeurs à atteindre pour les jets de sauvegarde du monstre :

\begin{itemize}
\tightlist
\item
  \textbf{MP :} Mort et Poison
\item
  \textbf{B :} Baguettes
\item
  \textbf{PP :} Paralysie et Pétrification
\item
  \emph{'S :} Souffles
\item
  \textbf{SSB :} Sorts, Sceptres et Bâtons
\end{itemize}

\textbf{Sauvegarde et DV :} Tous les monstres n'ont pas les mêmes scores
de sauvegarde, aussi le nombre de DV auxquels les scores indiqués
correspondent est précisé entre parenthèses (HN signifie que le monstre
se sauvegarde comme un Humain normal). Ce chiffre n'est pas toujours
égal au nombre de DV du monstre. En général, les monstres d'intelligence
inférieure ont un score de sauvegarde comme s'ils avaient moitié moins
de DV, et les monstres magiques comme s'ils avaient un ou plusieurs DV
de plus.

Certains monstres ont les mêmes jets de sauvegarde qu'une classe de
personnage. Dans ce cas, la classe et le niveau sont notés entre
parenthèse après les scores : C = clerc, E = elfe, G = guerrier, M =
magicien, N = nain, T = tinigens, V = voleur.

\subsection*{Jet d'attaque « Toucher une Armure de Classe de 0 » (TAC0)}\label{jet-dattaque-toucher-une-armure-de-classe-de-0-tac0}

La capacité d'un monstre à porter des coups en combat est déterminée par
ses DV.

\textbf{Bonus d'attaque :} Ce chiffre, nécessaire pour les arbitres
ayant adopté la règle de Classe d'armure ascendante, est noté entre
crochets.

\subsection*{Déplacement (DP)}\label{duxe9placement-dp}

La vitesse à laquelle le monstre se déplace. Chaque monstre a un
déplacement de base et un déplacement de rencontre (noté entre
parenthèses) égal au tiers du déplacement de base.

\textbf{Modes de déplacement :} Si le monstre peut se déplacer de
plusieurs manières (par exemple, en marchant, en volant, ou en
escaladant), les vitesses sont séparées par des barres obliques ( / ).

\subsection*{Score de moral (Ml)}\label{score-de-moral-ml}

La probabilité qu'un monstre persévère en combat.

\subsection*{Alignement (Al)}\label{alignement-al}

L'affiliation d'un monstre à la Loi, à la Neutralité ou au Chaos. S'il
est indiqué « tous », l'arbitre peut tirer au hasard ou décider lui-même
de l'alignement du monstre.

\subsection*{Valeur en XP (XP)}\label{valeur-en-xp-xp}

Les points d'expérience reçus par le groupe pour avoir vaincu le
monstre.

\subsection*{Nombre apparaissant (NA)}\label{nombre-apparaissant-na}

On trouve ici deux valeurs, la seconde étant entre parenthèses.

\textbf{Zéros :} Si la première valeur indique 0, les monstres de ce
type ne se trouvent en général pas dans les donjons. Si la deuxième
valeur est 0, ils ne se trouvent pas en extérieur et n'ont normalement
pas de repaire.

\textbf{Utilisation :} L'utilisation de ces valeurs dépend de la
situation dans laquelle les monstres sont rencontrés :

\begin{itemize}
\tightlist
\item
  \textbf{Monstres errants dans un donjon :} La première valeur
  détermine le nombre de monstres rencontrés dans un niveau de donjon
  égal à leur DV. Si le monstre est rencontré à un niveau supérieur à
  son DV, le nombre apparaissant peut être augmenté ; s'il est rencontré
  à un niveau inférieur à son DV, le nombre apparaissant devrait être
  réduit.
\item
  \textbf{Repaire des monstres dans un donjon :} La deuxième valeur
  répertorie le nombre de monstres trouvés dans un repaire dans un
  donjon.
\item 
	\textbf{Monstres errants dans les contrées sauvages :} La deuxième valeur
	indique le nombre de monstres rencontrés errant dans les contrées
	sauvages.
\item
  \textbf{Repaire des monstres dans les contrées sauvages :} La deuxième
  valeur multipliée par 5 indique le nombre de monstres trouvés dans un
  repaire dans les contrées sauvages.
\end{itemize}

\subsection*{Type de trésor (TT)}\label{type-de-truxe9sor-tt}

Le code de lettre utilisé pour déterminer la quantité et le type de
trésor possédés par le(s) monstre(s). Les lettres
citées sont utilisées comme suit :

\begin{itemize}
\tightlist
\item
  \textbf{De A à O :} Indique un trésor : la somme des richesses d'un
  grand monstre ou d'une communauté de petits monstres, généralement
  cachée dans son repaire. Pour les monstres qui apparaissent avec un
  nombre aléatoire supérieur à 1d4 (voir Nombre apparaissant), la
  quantité de trésor peut être réduite si le nombre de monstres était
  inférieur à la moyenne.
\item
  \textbf{De P à V :} Indique un trésor contenu dans les affaires d'un
  monstre intelligent (P à T) ou dans celles d'un groupe de monstres
  intelligents (U, V). Si un monstre dépourvu d'intelligence possède
  l'un de ces trésors, il se trouve sur les cadavres de ses victimes.
\end{itemize}
\end{multicols}

\begin{multicols}{2}[\section*{Notes}\label{notes-guxe9nuxe9rales}]

\subsection*{Infravision}\label{infravision}

Tous les monstres non-humains sont dotés de l'infravision. À moins
d'une indication contraire dans la description d'un monstre, il voit
dans le noir jusqu'à 18 m.

\subsection*{Langues}\label{langues}

20 \% des monstres intelligents parlent le commun, sauf indication contraire
dans leur description. De nombreuses espèces intelligentes ont aussi
leur propre langue.

\subsection*{Personnes}\label{personnes}

Certains sorts et effets magiques n'affectent que les « personnes ».
Cette catégorie inclut tous les humains et semi-humains, de même que les
monstres humanoïdes de 4+1 HD ou moins. L'arbitre décide si d'autres
monstres peuvent être considérés comme des personnes.

Les monstres suivants sont considérés comme des « Personnes » pour les
effets magiques qui peuvent les affecter : acolyte, bandit, berserker,
boucanier, brigand, commerçant, derviche, elfe, esprit follet, génie des
eaux, gnoll, gnome, gobelin, goblours, hobgobelin, homme-lézard,
homme-poisson, humain normal, kobold, lutin, marchand, médium, devin,
nain, néandertalien, négociant, noble, ogre, orque, pirate, pixie,
tinigens, troglodyte, vétéran.
\end{multicols}
\chapter{Groupe d'aventuriers}\label{groupe-daventuriers}
Cette procédure génère des groupes de PNJ aventuriers. Comme la
procédure est assez complexe, l'arbitre peut souhaiter prétirer quelques
groupes de PNJ afin de les utiliser lors de rencontres aléatoires. Les
détails généraux suivants s'appliquent à tous les types de PNJ décrits :

\begin{itemize}
\tightlist
\item
  \textbf{Sorts :} Si des lanceurs de sorts sont présents, choisissez ou
  tirez aléatoirement leurs sorts mémorisés.
\item
  \textbf{Équipement :} L'équipement classique pour partir à l'aventure.
\item
  \textbf{Trésor :} Trésor de type U+V, réparti parmi le groupe.
\item
  \textbf{Ordre de marche :} Décidé par l'arbitre.
\end{itemize}

Si les classes présentées dans cet ouvrage ne sont pas utilisées,
l'arbitre doit remplacer les classes listées par des équivalents
utilisés dans sa campagne.

\section*{Aventuriers basiques}\label{aventuriers-basiques}

\begin{itemize}
\tightlist
\item
  \textbf{Composition :} 1d4+4 personnages de classe et niveau choisis
  au hasard (voir ci-dessous).
\item
  \textbf{Alignement :} Lancez pour définir l'alignement de chaque PNJ
  ou une seule fois pour tout le groupe.
\end{itemize}

\section*{Aventuriers experts}\label{aventuriers-experts}

\begin{itemize}
\tightlist
\item
  \textbf{Composition :} 1d6+3 personnages de classe et niveau choisis
  au hasard (voir ci-dessous).
\item
  \textbf{Alignement :} Lancez pour définir l'alignement de chaque PNJ
  ou une seule fois pour tout le groupe.
\item
  \textbf{Montures :} Dans les contrées sauvages, il y a 75 \% de
  chances qu'elles soient montées.
\item
  \textbf{Objets magiques :} Par individu : Il est possible que le PNJ
  ait un objet magique mentionné dans chaque sous-table d'objets
  magiques appropriée.
  La probabilité par sous-table est de 5 \% par niveau du PNJ. Les
  objets tirés qui ne peuvent pas être utilisés par ce PNJ doivent être
  ignorés (pas de relance).
\end{itemize}

\section*{Clerc de haut-niveau}\label{clerc-de-haut-niveau}

Un clerc de haut niveau et ses fidèles (montures et objets magiques
comme des Aventuriers experts).

\begin{itemize}
\tightlist
\item
  \textbf{Composition :} Chef (Clerc de niveau 1d6+6), 1d4 Clercs
  (niveau 1d4+1), 1d3 Guerriers (niveau 1d6).
\item
  \textbf{Alignement :} Tirer pour l'ensemble du groupe.
\end{itemize}

\section*{Guerrier de haut-niveau}\label{guerrier-de-haut-niveau}

Un guerrier de haut niveau et son groupe de suivants, souvent en route
vers une guerre ou en revenant (montures et objets magiques comme des
Aventuriers experts).

\begin{itemize}
\tightlist
\item
  \textbf{Composition :} Chef (guerrier de niveau 1d4+6), 2d4 Suivants
  (niveau 1d4+2, n'importe quelle classe).
\item
  \textbf{Alignement :} Tirer pour l'ensemble du groupe.
\end{itemize}

\section*{Magicien de haut-niveau}\label{magicien-de-haut-niveau}

Un magicien de haut niveau, accompagné de ses apprentis et d'un groupe
de gardes embauchés, souvent à la recherche de savoirs mystérieux
(montures et objets magiques comme des Aventuriers experts).

\begin{itemize}
\tightlist
\item
  \textbf{Composition :} Chef (Magicien de niveau 1d4+6), 1d4 Apprentis
  (Magiciens de niveau 1d3), 1d4 Mercenaires (Guerriers de niveau
  1d4+1).
\item
  \textbf{Alignement :} Tirer pour définir l'alignement du chef. Les
  apprentis ont le même alignement que le chef, tandis que les
  mercenaires peuvent être d'un alignement différent.
\end{itemize}

\section*{Classe, Niveau et alignement des PNJ Aventuriers}\label{classe-et-niveau-des-pnj-aventuriers}

\begin{table}[h]
\parbox{.45\linewidth}{
\centering
\begin{tabular}[]{@{}lll@{}}
\titlecell Niveau (d8) & \titlecell Classe & \titlecell Basique \\
1 & Clerc & 1d3 \\
2 & Nain & 1d3 \\
3 & Elfe & 1d3 \\
4 & Guerrier & 1d3 \\
5 & Guerrier & 1d3 \\
6 & Tinigens & 1d3 \\
7 & Magicien & 1d3 \\
8 & Voleur & 1d3 \\
\end{tabular}
\caption{Classe et Niveau des PNJ Aventuriers}
} 
\parbox{.45\linewidth}{
\centering
\begin{tabular}[]{@{}ll@{}}
\titlecell d6 & \titlecell Alignement \\
1--2 & Loyal \\
3--4 & Neutre \\
5--6 & Chaotique \\
\end{tabular}\\
\caption{Alignement des PNJ Aventuriers}
}

\end{table}

\chapter{Tables de rencontres}
\section*{Tableaux de rencontre des donjons}\label{tableaux-de-rencontre-des-donjons}

Cette section fournit des tableaux de rencontres par niveau de donjon,
en utilisant les monstres disponibles dans ce livre. Les arbitres se
servant de monstres supplémentaires (ou alternatifs) doivent, soit
adapter les tableaux afin de les inclure, soit créer leurs propres
tableaux de rencontre. Des tableaux spéciaux peuvent également être
créés pour refléter l'équilibre différent de monstres habitant un donjon
spécifique.

\subsection*{Comment tirer une rencontre}\label{comment-tirer-une-rencontre}

Lancez 1d20 et cherchez le résultat dans la colonne du tableau
ci-dessous correspondant au niveau du donjon exploré. Le résultat
indique le monstre rencontré, ainsi que le nombre qui apparaît entre
parenthèses.

\subsection*{Remarques}\label{remarques}

\paragraph{PNJ Aventuriers :} Les rencontres avec des PNJ aventuriers sont
répertoriées dans les tableaux comme « Aventurier basique » ou «
Aventurier expert ». On trouvera les étapes pour définir ces PNJ
aventuriers dans \href{Groupe_d’aventuriers.md}{Groupes d'aventurier}.

\paragraph{Nombre apparaissant :} Le nombre de monstres indiqué pour
certaines créatures dans le tableau ne correspond pas aux valeurs
répertoriées dans leur description. S'il le souhaite et par souci de
cohérence, l'arbitre peut à la place utiliser la valeur apparaissant
dans la description du monstre.

\subsection*{Rencontre selon le niveau du donjon (Niveaux 1 à 3)}\label{rencontre-selon-le-niveau-du-donjon-niveaux-1-uxe0-3}

\begin{table}[H]
 \centering
\begin{tabular}[]{llll}
\titlecell D20 & \titlecell Niveau 1 & \titlecell Niveau 2 & \titlecell Niveau 3 \\
1 & Abeille tueuse (1d10) & Araignée Veuve noire (1d3) & Araignée
Tarentelle (1d3) \\
2 & Acolyte (1d8) & Babouin des rochers (2d6) & Aventurier basique
(1d4+4) \\
3 & Araignée crabe (1d4) & Berserker (1d6) & Cube gélatineux (1) \\
4 & Bandit (1d8) & Elfe (1d4) & Doppelganger (1d6) \\
5 & Commerçant (1d8) & Félin, Lion des montagnes (1d6) & Fourmi géante
(2d4) \\
6 & Esprit follet (3d6) & Gnoll (1d6) & Gargouille (1d6) \\
7 & Gnome (1d6) & Goule (1d6) & Gelée ocre (1) \\
8 & Gobelin (2d4) & Hobgobelin (1d6) & Goblours (2d4) \\
9 & Kobold (4d4) & Homme-lézard (2d4) & Gorille blanc (1d6) \\
10 & Limon vert (1d4) & Lézard, Draco (1d4) & Harpie (1d6) \\
11 & Loup (2d6) & Mouche voleuse (1d6) & Rat-garou (1d8) \\
12 & Lézard Gecko (1d3) & Noble (2d6) & Médium (1d4) \\
13 & Musaraigne géante (1d10) & Néandertalien (1d10) & Méduse (1d3) \\
14 & Nain (1d6) & Pixie (2d4) & Nécrophage (1d6) \\
15 & Orque (2d4) & Scarabée à huile (1d8) & Ogre (1d6) \\
16 & Scarabée de feu (1d8) & Serpent, Crotale (1d8) & Ombre (1d8) \\
17 & Serpent, Cobra cracheur (1d6) & Troglodyte (1d8) & Scarabée tigré
(1d6) \\
18 & Squelette (3d4) & Vase grise (1) & Statue vivante de cristal
(1d6) \\
19 & Strige (1d10) & Vétéran (2d4) & Thoul (1d6) \\
20 & Tinigens (3d6) & Zombie (2d4) & Ver charognard (1d3) \\
\end{tabular}
\end{table}
\pagebreak
\input{Tableaux_des_rencontres_en_Contrées_sauvages.tex}
\pagebreak
\section*{Rencontres en place forte}\label{rencontres-en-place-forte}

Lorsque les PJ vagabondent à proximité de la place forte d'un PNJ de
haut niveau, un accueil chaleureux n'est pas toujours garanti. L'arbitre
peut utiliser les règles suivantes s'il n'a pas de notes spécifiques sur
le dirigeant d'une place forte et sur les ordres donnés aux patrouilles
de gardes.

\subsection*{Dirigeant}\label{dirigeant}

L'arbitre doit décider de quelle classe est le PNJ qui revendique la
propriété de la place forte et des terres environnantes :

\begin{itemize}
\item
  \textbf{Clerc :} De niveau 1d8+6.
\item
  \textbf{Guerrier :} De niveau 1d6+8.
\item
  \textbf{Magicien :} De niveau 1d4+10.
\end{itemize}

Les places fortes appartenant à des semi-humains sont des cas
relativement rares et doivent être détaillées à l'avance par l'arbitre.
En règle générale, ils tenteront d'éviter tout contact avec des
voyageurs.

\subsection*{Patrouilles}\label{patrouilles}

Les étrangers parcourant les terres autour d'une place forte seront
généralement repérés par des groupes de mercenaires embauchés pour
patrouiller dans les environs. Le type de troupes dépend de la classe du
dirigeant :

\begin{itemize}
\item
  \textbf{Clerc :} 2d6 cavaliers moyens. Équipés de cottes de mailles
  (CA 5 {[}14{]}) et de lances. Moral 9.
\item
  \textbf{Guerrier :} 2d6 cavaliers lourds 2d6. Équipés d'armures de
  plates (CA 3 {[}16{]}), de lances et d'épées. Moral 9.
\item
  \textbf{Magicien :} 2d6 fantassins lourds. Équipés de cottes de
  mailles + bouclier (CA 4 {[}15{]}) et d'épées. Moral 8.
\end{itemize}

\subsection*{Garnison}\label{garnison}

Les patrouilles, telles que décrites ci-dessus, ne sont qu'une petite
partie de la garnison du dirigeant. D'autres forces peuvent inclure des
monstres magiques ou des humains montés sur des créatures volantes.

\subsection*{Réaction aux voyageurs}\label{ruxe9action-aux-voyageurs}

La réaction du dirigeant face aux voyageurs sur son domaine dépend de sa
classe. On peut la déterminer en lançant 1d6 et en consultant ce tableau
:

\begin{table}[H]
	\centering
\begin{tabular}[]{llll}
\titlecell{d6} & \titlecell{Clerc} & \titlecell{Guerrier} & \titlecell{Magicien} \\
1 & Expulsion & Expulsion & Expulsion \\
2 & Expulsion & Expulsion & Ignorance \\
3 & Ignorance & Expulsion & Ignorance \\
4 & Ignorance & Ignorance & Ignorance \\
5 & Invitation & Ignorance & Ignorance \\
6 & Invitation & Invitation & Invitation \\
\end{tabular}
\caption{Réaction du dirigeant selon sa Classe}\label{ruxe9action-du-dirigeant-selon-sa-classe}
\end{table}

\textbf{Expulsion :} La patrouille est chargée d'expulser les intrus
hors du domaine. Alternativement, ils peuvent exiger un péage des
voyageurs de passage. Le prix exact demandé dépend de la personnalité du
dirigeant, de la richesse apparente des PJ, etc. Si les PJ refusent de
payer le péage, la patrouille peut les attaquer, les expulser ou tenter
de les faire prisonniers.

\textbf{Ignorance :} La patrouille quitte les PJ pour aller vaquer à ses
occupations.

\textbf{Invitation :} La patrouille apporte un message du dirigeant du
domaine, invitant les PJ à rester dans la place forte. Le motif exact de
cette décision dépend de sa personnalité --- ce qui n'est pas
nécessairement sans conséquence.


\part{Description des monstres}
\begin{monster}
\monstercarac{img}{Killer_Bee.jpg}
\monstercarac{name}{Abeille tueuse}
\monstercarac{description}{
  Abeilles géantes, longues de 30 cm, au tempérament agressif. Installent
leurs ruches dans les souterrains.

}
\monsterdetail{Agressive}{Attaque en général à vue. Attaque toujours un intrus se trouvant à moins
de 9 m de la ruche.
}
\monsterdetail{Meurt après l’attaque}{Sur une attaque de dard réussie, l'abeille tueuse meurt.
}
\monsterdetail{Poison}{Provoque la mort (jet de sauvegarde contre le poison).
}
\monsterdetail{Dard fiché}{Reçoit 1 dégât par round à cause du dard qui pénètre plus profondément
dans la blessure. Le retirer prend un round.
}
\monsterdetail{Reine}{Une reine avec 2 DV vit dans la ruche. Elle ne meurt pas après avoir
piqué.
}
\monsterdetail{Gardes}{Au moins 10 abeilles (dont 4 ou plus ont 1 DV) sont toujours présentes
dans ou à proximité de la ruche pour veiller sur la reine.
}
\monsterdetail{Miel}{La ruche contient environ un litre de miel magique. Quiconque mange
cette quantité regagne 1d4 points de vie.
}
\monstercarac{ca}{7 [12]}
\monstercarac{hd}{½* (2 pv)}
\monstercarac{taco}{19 [0]}
\monstercarac{moral}{9}
\monstercarac{alignement}{Neutre}
\monstercarac{xp}{6 (garde : 13, reine : 25)}
\monstercarac{nombre_donjon}{1d6}
\monstercarac{nombre_exterieur}{5d6}
\monstercarac{tresor}{Miel}
\monstercarac{mvt}{45 m (15 m) vol}
\monstercarac{save_mort_poison}{12}
\monstercarac{save_baguettes}{13}
\monstercarac{save_paralysie_petrification}{14}
\monstercarac{save_souffles}{15}
\monstercarac{save_sorts_sceptres_batons}{16}
\monsterattack{1 × dard }{1d3 + poison + dard fiché}
\end{monster}

\begin{monster}
\monstercarac{name}{Acolyte}
\monstercarac{description}{
  Clerc de 1er niveau en mission pour son dieu.

}
\monsterdetail{Chef}{Chaque groupe de 4 acolytes ou plus est dirigé par un clerc de niveau
supérieur (1d10 : 1--4 : 2e niveau, 5--7 : 3e niveau, 8--9 : 4e niveau,
10 : 5e niveau). Choisissez ou tirez au sort les sorts du chef.
}
\monstercarac{ca}{2 [17]}
\monstercarac{hd}{1 (4 pv)}
\monstercarac{taco}{19 [0]}
\monstercarac{moral}{7}
\monstercarac{alignement}{Tous}
\monstercarac{xp}{10}
\monstercarac{nombre_donjon}{1d8}
\monstercarac{nombre_exterieur}{1d20}
\monstercarac{tresor}{U}
\monstercarac{mvt}{18 m (6 m)}
\monstercarac{save_mort_poison}{11}
\monstercarac{save_baguettes}{12}
\monstercarac{save_paralysie_petrification}{14}
\monstercarac{save_souffles}{16}
\monstercarac{save_sorts_sceptres_batons}{15}
\monsterattack{1 × masse }{1d6}
\end{monster}

\input{liste/Aigle_géant.tex}
\begin{monster}
\monstercarac{name}{Aigle normal}
\monstercarac{description}{
  Oiseau de proie qui plane dans les courants d'air chauds et chasse ses
proies au sol.

  Petit oiseau de proie. N'attaque les humains que s'ils lui semblent sans
défense.

}
\monsterdetail{Dressage}{Peut être dressé en tant qu'animal gardien ou de chasse.
}
\monsterdetail{Descente en piqué}{Peut plonger sur les victimes visibles en contrebas. Si la victime est
surprise, l'attaque lui inflige le double de dégâts. Sur un jet
d'attaque de 18 ou plus, la victime peut être emportée (à condition
qu'elle soit de taille appropriée).
}
\monstercarac{ca}{8 [11]}
\monstercarac{hd}{½ (2 pv)}
\monstercarac{taco}{19 [0]}
\monstercarac{moral}{7}
\monstercarac{alignement}{Neutre}
\monstercarac{xp}{5}
\monstercarac{nombre_donjon}{0}
\monstercarac{nombre_exterieur}{1d6}
\monstercarac{tresor}{Aucun}
\monstercarac{mvt}{144 m (48 m) vol}
\monstercarac{save_mort_poison}{14}
\monstercarac{save_baguettes}{15}
\monstercarac{save_paralysie_petrification}{16}
\monstercarac{save_souffles}{17}
\monstercarac{save_sorts_sceptres_batons}{18}
\monsterattack{1 × serre }{}
\monsterattack{bec }{1d2}
\end{monster}

\begin{monster}
\monstercarac{name}{Apparition}
\monstercarac{description}{
  Monstre mort-vivant incorporel qui apparaît comme une forme pâle et
humaine de brume regroupée. Habite dans les régions désertiques ou dans
les maisons ayant appartenu à d'anciennes victimes.

}
\monsterdetail{Mort-vivant}{Ne fait aucun bruit jusqu'à ce qu'elle porte son attaque. Immunisée
contre les effets affectant les créatures vivantes (par ex., le poison).
Immunisée contre les sorts affectant ou lisant l'esprit (par ex.,
charme, paralysie, sommeil).
}
\monsterdetail{Immunité aux armes normales}{Uniquement affectée par des armes en argent ou magiques.
}
\monsterdetail{Réduction des dégâts}{Ne reçoit que la moitié des dégâts des armes en argent.
}
\monsterdetail{Absorption d’énergie}{Une cible touchée avec succès perd un niveau d'expérience (ou Dé de vie)
de manière permanente. Ceci entraîne la perte de points de vie
correspondant au DV de perdu, ainsi que les autres capacités dues aux
niveaux perdus (sorts, jets de sauvegarde, etc.). L'XP du personnage est
réduite au minimum du nouveau niveau. Une personne qui perd tous ses
niveaux devient une apparition en une journée, sous le contrôle de celle
qui l'a tuée.
}
\monstercarac{ca}{3 [16]}
\monstercarac{hd}{4** (18 pv)}
\monstercarac{taco}{16 [+3]}
\monstercarac{moral}{12}
\monstercarac{alignement}{Chaotique}
\monstercarac{xp}{175}
\monstercarac{nombre_donjon}{1d4}
\monstercarac{nombre_exterieur}{1d6}
\monstercarac{tresor}{E}
\monstercarac{save_mort_poison}{10}
\monstercarac{save_baguettes}{11}
\monstercarac{save_paralysie_petrification}{12}
\monstercarac{save_souffles}{13}
\monstercarac{save_sorts_sceptres_batons}{14}
\monsterattack{1 × toucher }{1d6 + absorption d’énergie)}
\end{monster}

\input{liste/Araignée_crabe.tex}
\begin{monster}
\monstercarac{name}{Tarentelle}
\monstercarac{description}{
  Cette araignée chasseuse aux longs poils, faisant 2,10 m de long, est
magique et ressemble à une tarentule.

}
\monsterdetail{Poison}{Jet de sauvegarde contre le poison ou la cible se met à danser pendant
2d6 tours (souffrant de spasmes douloureux et saccadés qui ressemblent à
une danse macabre).
}
\monsterdetail{Les spectateurs}{Les spectateurs de la personne affectée par le poison doivent effectuer
un jet de sauvegarde contre les sorts ou commencer à danser de la même
manière, aussi longtemps que la victime est empoisonnée.
}
\monsterdetail{Danse}{Les personnes affectées subissent un malus de --4 sur leurs jets
d'attaque et à leur CA. Après 5 tours de danse, ils sont épuisés et
s'écroulent à terre, sans défense.
}
\monstercarac{ca}{5 [14]}
\monstercarac{hd}{4* (18 pv)}
\monstercarac{taco}{16 [+3]}
\monstercarac{moral}{8}
\monstercarac{alignement}{Neutre}
\monstercarac{xp}{125}
\monstercarac{nombre_donjon}{1d3}
\monstercarac{nombre_exterieur}{1d3}
\monstercarac{tresor}{U}
\monstercarac{mvt}{36 m (12 m)}
\monstercarac{save_mort_poison}{12}
\monstercarac{save_baguettes}{13}
\monstercarac{save_paralysie_petrification}{14}
\monstercarac{save_souffles}{15}
\monstercarac{save_sorts_sceptres_batons}{16}
\monsterattack{1 × morsure }{1d8 + poison}
\end{monster}

\begin{monster}
\monstercarac{name}{Veuve noire}
\monstercarac{description}{
  Araignée noire, de 1,80 m de long avec un motif de sablier rouge sur
l'abdomen ; habite dans des tanières remplies de toiles et s'attaque
parfois aux humains.

}
\monsterdetail{Poison}{Provoque la mort en 1 tour (jet de sauvegarde contre le poison).
}
\monsterdetail{Toiles}{Les créatures prises dans les toiles s'emmêlent et ne peuvent plus
bouger. Se libérer dépend de la Force : 2d4 tours pour la force d'un
humain normal ; 4 rounds pour une force supérieure à 18 ; 2 rounds pour
les créatures ayant une force de géant. Les toiles d'araignée peuvent
être détruites par le feu en deux rounds. Toutes les créatures prises
dans une toile enflammée subissent 1d6 points de dégâts.
}
\monstercarac{ca}{6 [13]}
\monstercarac{hd}{3* (13 pv)}
\monstercarac{taco}{17 [+2]}
\monstercarac{moral}{8}
\monstercarac{alignement}{Neutre}
\monstercarac{xp}{50}
\monstercarac{nombre_donjon}{1d3}
\monstercarac{nombre_exterieur}{1d3}
\monstercarac{tresor}{U}
\monstercarac{save_mort_poison}{12}
\monstercarac{save_baguettes}{13}
\monstercarac{save_paralysie_petrification}{14}
\monstercarac{save_souffles}{15}
\monstercarac{save_sorts_sceptres_batons}{16}
\monsterattack{1 × morsure }{2d6 + poison)}
\end{monster}

\begin{monster}
\monstercarac{name}{Babouin des rochers}
\monstercarac{description}{
  Grands babouins féroces et semi-intelligents qui vivent en meutes
dirigées par un mâle puissant. Communiquent par des cris, sont
omnivores, mais privilégient la viande.

}
\monsterdetail{Armes}{Utilisent des os ou des branches en guise de gourdins, mais pas d'autres
outils.
}
\monstercarac{ca}{6 [13]}
\monstercarac{hd}{2 (9 pv)}
\monstercarac{taco}{18 [+1]}
\monstercarac{moral}{8}
\monstercarac{alignement}{Neutre}
\monstercarac{xp}{20}
\monstercarac{nombre_donjon}{2d6}
\monstercarac{nombre_exterieur}{5d6}
\monstercarac{tresor}{U}
\monstercarac{mvt}{36 m (12 m)}
\monstercarac{save_mort_poison}{12}
\monstercarac{save_baguettes}{13}
\monstercarac{save_paralysie_petrification}{14}
\monstercarac{save_souffles}{15}
\monstercarac{save_sorts_sceptres_batons}{16}
\monsterattack{1 × gourdin }{1d6}
\monsterattack{1 × morsure }{1d3}
\end{monster}

\begin{monster}
\monstercarac{name}{Cachalot}
\monstercarac{description}{
  Baleine gigantesque ; mesure jusqu'à 18 m de long et habite dans les
grands océans. Chasse les monstres des grands fonds (par ex. les calmars
géants).

}
\monsterdetail{Gobe entièrement}{Sur un jet d'attaque de 4 ou plus que le nombre requis pour toucher,
toute cible de la taille d'un humain ou inférieure est avalée. Une fois
gobée, la cible subit 3d6 points de dégâts par round (jusqu'à ce que le
cachalot soit tué ou que la cible meure) ; la cible peut attaquer avec
des armes tranchantes en subissant un malus de --4 ; le cadavre de la
victime est digéré en 6 tours après la mort.
}
\monsterdetail{Éperonnement d’un navire}{Environ 10 \% de chances d'attaquer un navire.
}
\monstercarac{ca}{6 [13]}
\monstercarac{hd}{36 (162 pv)}
\monstercarac{taco}{5 [+14]}
\monstercarac{moral}{7}
\monstercarac{alignement}{Neutre}
\monstercarac{xp}{6,250}
\monstercarac{nombre_donjon}{0}
\monstercarac{nombre_exterieur}{1d3}
\monstercarac{tresor}{V}
\monstercarac{save_mort_poison}{4}
\monstercarac{save_baguettes}{5}
\monstercarac{save_paralysie_petrification}{6}
\monstercarac{save_souffles}{5}
\monstercarac{save_sorts_sceptres_batons}{8}
\monsterattack{1 × morsure }{3d20) ou 1 × éperonnement}
\end{monster}

\input{liste/Épaulard.tex}
\begin{monster}
\monstercarac{name}{Narval}
\monstercarac{description}{
}
\monsterdetail{Corne}{Vaut 1d6 × 1 000 po. Des rumeurs disent que la corne d'un narval vibre
lorsque le mal est à proximité.
}
\monstercarac{ca}{7 [12]}
\monstercarac{hd}{12 (54 pv)}
\monstercarac{taco}{10 [+9]}
\monstercarac{moral}{8}
\monstercarac{alignement}{Loyal}
\monstercarac{xp}{1 100}
\monstercarac{nombre_donjon}{0}
\monstercarac{nombre_exterieur}{1d4}
\monstercarac{tresor}{Corne}
\monstercarac{save_mort_poison}{6}
\monstercarac{save_baguettes}{7}
\monstercarac{save_paralysie_petrification}{8}
\monstercarac{save_souffles}{8}
\monstercarac{save_sorts_sceptres_batons}{10}
\monsterattack{1 × morsure }{1d8)}
\monsterattack{1 × corne }{2d6)}
\end{monster}

\begin{monster}
\monstercarac{img}{brigand.jpg}
\monstercarac{name}{Bandit}
\monstercarac{description}{
  Des voleurs PNJ qui vivent de rapine.

}
\monsterdetail{Tromperie}{Surprend ses victimes grâce à des déguisements et autres ruses.
}
\monsterdetail{Chef}{Peuvent avoir un chef de 2e niveau ou plus, de n'importe quelle classe
humaine.
}
\monsterdetail{Butin}{Ne possèdent qu'un trésor de type A lorsqu'on les rencontre dans leur
repaire situé dans les contrées sauvages.
}
\monstercarac{ca}{6 [13]}
\monstercarac{hd}{1 (4 pv)}
\monstercarac{taco}{19 [0]}
\monstercarac{moral}{8}
\monstercarac{alignement}{Neutre ou Chaotique}
\monstercarac{xp}{10}
\monstercarac{nombre_donjon}{1d8}
\monstercarac{nombre_exterieur}{3d10}
\monstercarac{tresor}{U (A)}
\monstercarac{mvt}{36 m (12 m)}
\monstercarac{save_mort_poison}{13}
\monstercarac{save_baguettes}{14}
\monstercarac{save_paralysie_petrification}{13}
\monstercarac{save_souffles}{16}
\monstercarac{save_sorts_sceptres_batons}{15}
\monsterattack{1 × arme }{1d6 (selon l’arme)}
\end{monster}

\begin{monster}
\monstercarac{name}{Basilic}
\monstercarac{description}{
  Lézard de forme serpentine, long de 3 m, et dépourvu d'intelligence,
mais hautement magique. Il niche dans les cavernes et les ronces
épineuses.

}
\monsterdetail{Surprise}{Les personnages surpris par le basilic subissent une attaque de son
regard pétrifiant.
}
\monsterdetail{Toucher pétrifiant}{Toute créature touchée par le basilic est transformée en pierre (jet de
sauvegarde contre la pétrification).
}
\monsterdetail{Regard pétrifiant}{Toute créature rencontrant le regard du basilic est transformée en
pierre (jet de sauvegarde contre la pétrification). À moins d'éviter son
regard ou d'utiliser un miroir, les personnes en mêlée contre le basilic
sont affectées chaque round.
}
\monsterdetail{Éviter son regard}{Une personne qui cherche à combattre le basilic en évitant son regard
subit un malus de --4 pour toucher, tandis que le basilic bénéficie d'un
bonus de +2.
}
\monsterdetail{Utiliser un miroir}{Le reflet du basilic est inoffensif. Se battre en regardant dans un
miroir inflige un malus de --1 à l'attaque. Si le basilic voit son
propre reflet (2 chances sur 6), il doit effectuer un jet de sauvegarde
ou être pétrifié.
}
\monstercarac{ca}{4 [15]}
\monstercarac{hd}{6+1** (28 pv)}
\monstercarac{taco}{13 [+6]}
\monstercarac{moral}{9}
\monstercarac{alignement}{Neutre}
\monstercarac{xp}{950}
\monstercarac{nombre_donjon}{1d6}
\monstercarac{nombre_exterieur}{1d6}
\monstercarac{tresor}{F}
\monstercarac{save_mort_poison}{10}
\monstercarac{save_baguettes}{11}
\monstercarac{save_paralysie_petrification}{12}
\monstercarac{save_souffles}{13}
\monstercarac{save_sorts_sceptres_batons}{14}
\monsterattack{1 × morsure }{1d10 + pétrification)}
\monsterattack{1 × regard }{pétrification)}
\end{monster}

\input{liste/Belette_géante.tex}
\begin{monster}
\monstercarac{img}{berseker.jpg}
\monstercarac{name}{Berserker}
\monstercarac{description}{
  Guerrier qui entre en rage pendant le combat. Il ne fait jamais de
prisonniers.

}
\monsterdetail{Rage de bataille}{Bonus de +2 à l'attaque contre les humains et humanoïdes similaires (par
exemple, orques, gobelins, etc.). La rage le pousse parfois à attaquer
ses alliés.
}
\monsterdetail{Butin}{Ne possède qu'un trésor de type B lorsqu'on le rencontre dans les
contrées sauvages.
}
\monstercarac{ca}{7 [12]}
\monstercarac{hd}{1+1* (5 pv)}
\monstercarac{taco}{18 [+1]}
\monstercarac{moral}{12}
\monstercarac{alignement}{Neutral}
\monstercarac{xp}{19}
\monstercarac{nombre_donjon}{1d6}
\monstercarac{nombre_exterieur}{3d10}
\monstercarac{tresor}{P (B)}
\monstercarac{mvt}{36 m (12 m)}
\monstercarac{save_mort_poison}{12}
\monstercarac{save_baguettes}{13}
\monstercarac{save_paralysie_petrification}{14}
\monstercarac{save_souffles}{15}
\monstercarac{save_sorts_sceptres_batons}{16}
\monsterattack{1 × arme }{1d8 (selon l’arme)}
\end{monster}

\begin{monster}
\monstercarac{name}{Boucanier}
\monstercarac{description}{
  Un marin qui vit de raids côtiers et d'abordages de bateaux. Il sillonne
les rivières, les lacs et les côtes, plus rarement les océans.

}
\monsterdetail{Bateaux et équipages}{Ils dépendent de l'endroit où on les rencontre. Sur les rivières et les
lacs : 1d8 embarcations fluviales accueillant chacune 1d2 × 10
boucaniers ; sur les côtes, 1d6 petites galère(s) accueillant chacune
1d3+1 × 10 boucaniers ; partout : 1d4 drakkar(s) accueillant chacun
1d3+2 × 10 boucaniers ; océans : 1d4 navires de guerre accueillant
chacun 1d5+3 × 10 boucaniers (voir
\href{../../Equipement_services/Embarcations.md}{Embarcations} pour les
détails sur les navires).
}
\monsterdetail{Armement}{60 \% du groupe est équipé avec : une armure de cuir, une épée ; 30 \%
est équipé avec : une armure de cuir, une épée, une arbalète ; 10 \% est
équipé avec : une cotte de mailles, une épée, une arbalète.
}
\monsterdetail{Chefs et capitaines}{Par groupe de 30 boucaniers, on trouve un guerrier de 4e niveau. Chaque
navire possède un capitaine (guerrier de 7e niveau).
}
\monsterdetail{Commandant de flotte}{Guerrier de 9e niveau ; 30 \% de chances qu'il s'agisse d'un magicien
(niveau 1d2+9) ; 25 \% de chances qu'il s'agisse d'un clerc (8e niveau).
}
\monsterdetail{Trésor}{Il est réparti entre les différents navires. Au lieu d'être transporté à
bord, il peut y avoir une carte indiquant l'endroit où il est enterré.
}
\monsterdetail{Havres}{Les villes côtières sans loi et fortifiées peuvent servir de havre pour
les boucaniers et les pirates.
}
\monstercarac{ca}{7 [12] or 5 [14]}
\monstercarac{hd}{1 (4 pv)}
\monstercarac{taco}{19[0]}
\monstercarac{moral}{6}
\monstercarac{alignement}{Neutre}
\monstercarac{xp}{10}
\monstercarac{nombre_donjon}{0}
\monstercarac{nombre_exterieur}{voi ci-dessous}
\monstercarac{tresor}{A}
\monstercarac{save_mort_poison}{12}
\monstercarac{save_baguettes}{13}
\monstercarac{save_paralysie_petrification}{14}
\monstercarac{save_souffles}{15}
\monstercarac{save_sorts_sceptres_batons}{16}
\monsterattack{1 × arme }{1d6 ou selon l’arme)}
\end{monster}

\begin{monster}
\monstercarac{img}{brigand.jpg}
\monstercarac{name}{Brigand}
\monstercarac{description}{
  Hors-la-loi et mercenaire qui vit de pillages et attaque les voyageurs.

}
\monsterdetail{Infanterie}{Une moitié est équipée avec : une armure légère, un bouclier, une épée,
et un arc court.
}
\monsterdetail{Cavalerie}{L'autre moitié monte des chevaux de selle et est équipée avec : une
cotte de mailles, un bouclier, et une épée.
}
\monsterdetail{Chefs et commandants}{Par groupe de 20 brigands, on trouve un chef (guerrier de 2e niveau).
Par groupe de 40 brigands, on trouve un commandant (guerrier de 4e
niveau). Les chefs montent des chevaux de guerre (avec protections
supplémentaires : un caparaçon) et sont équipés avec : une armure de
plates, une épée et une lance.
}
\monsterdetail{Camps fortifiés}{Les bandes se regroupent fréquemment et vivent dans des camps fortifiés
abritant 5d6 × 10 brigands.
}
\monsterdetail{Chefs de camps}{Les camps issus d'un regroupement sont dirigés par un guerrier de 9e
niveau, auquel s'ajoute un guerrier de 5e niveau par groupe de 50
brigands. De plus, il y a 50 \% de chances qu'il s'agisse d'un magicien
(niveau 1d3+8) et 30 \% de chance qu'il s'agisse d'un clerc (8e niveau).
}
\monstercarac{ca}{6 [13]}
\monstercarac{hd}{1 (4 pv)}
\monstercarac{taco}{19 [0]}
\monstercarac{moral}{8}
\monstercarac{alignement}{Chaotique}
\monstercarac{xp}{10}
\monstercarac{nombre_donjon}{0}
\monstercarac{nombre_exterieur}{1d4 × 10}
\monstercarac{tresor}{A}
\monstercarac{mvt}{36 m (12 m)}
\monstercarac{save_mort_poison}{12}
\monstercarac{save_baguettes}{13}
\monstercarac{save_paralysie_petrification}{14}
\monstercarac{save_souffles}{15}
\monstercarac{save_sorts_sceptres_batons}{16}
\monsterattack{1 × arme }{1d6 (selon l’arme)}
\end{monster}

\input{liste/Bête_dimensionnelle.tex}
\begin{monster}
\monstercarac{img}{caecilia.png}
\monstercarac{name}{Caecilia}
\monstercarac{description}{
  Amphibien gigantesque, long de 10 m, de couleur grise, ressemble à un
vers muni d'une énorme gueule dentée.

}
\monsterdetail{Gobe entièrement}{Sur un jet d'attaque de 19 ou plus. Une fois gobée, la cible subit 1d8
points de dégâts par round (jusqu'à ce que la caecilia soit tuée ou que
la cible meure) ; peut attaquer avec une dague en subissant un malus de
--4 pour toucher ; le cadavre de la victime est digéré en 6 tours après
la mort.
}
\monstercarac{ca}{6 [13]}
\monstercarac{hd}{6* (27 pv)}
\monstercarac{taco}{14 [+5]}
\monstercarac{moral}{9}
\monstercarac{alignement}{Neutre}
\monstercarac{xp}{500}
\monstercarac{nombre_donjon}{1d3}
\monstercarac{nombre_exterieur}{1d3}
\monstercarac{tresor}{B}
\monstercarac{mvt}{18 m (6 m)}
\monstercarac{save_mort_poison}{12}
\monstercarac{save_baguettes}{13}
\monstercarac{save_paralysie_petrification}{14}
\monstercarac{save_souffles}{15}
\monstercarac{save_sorts_sceptres_batons}{16}
\monsterattack{1 × morsure }{1d8}
\end{monster}

\input{liste/Calmar_géant.tex}
\begin{monster}
\monstercarac{img}{centaur.png}
\monstercarac{name}{Centaure}
\monstercarac{description}{
  Créature fantastique qui possède les jambes et le corps d'un cheval, le
torse et la tête d'un humain. Vit en petits groupes familiaux ou
tribaux, dans des prairies sauvages et des forêts isolées.

}
\monsterdetail{Armes}{Arcs, gourdins, lances.
}
\monsterdetail{Repaire}{Caché dans des bois denses, le long de chemins tortueux et gardés.
}
\monsterdetail{Femelles et petits}{Restent habituellement dans le repaire et fuient s'ils sont attaqués.
Les petits ont 2 DV et font 2 attaques de sabot (1d2) et 1 attaque
d'arme (1d4 ou selon l'arme).
}
\monstercarac{ca}{5 [14]}
\monstercarac{hd}{4 (18 pv)}
\monstercarac{taco}{16 [+3]}
\monstercarac{moral}{8}
\monstercarac{alignement}{Neutre}
\monstercarac{xp}{75}
\monstercarac{nombre_donjon}{0}
\monstercarac{nombre_exterieur}{2d10}
\monstercarac{tresor}{A}
\monstercarac{mvt}{54 m (18 m)}
\monstercarac{save_mort_poison}{10}
\monstercarac{save_baguettes}{11}
\monstercarac{save_paralysie_petrification}{12}
\monstercarac{save_souffles}{13}
\monstercarac{save_sorts_sceptres_batons}{14}
\monsterattack{2 × sabot }{1d6}
\monsterattack{1 × arme }{1d6 }
\monsterattack{selon l’arme)}{}
\end{monster}

\begin{monster}
\monstercarac{img}{Camel.png}
\monstercarac{name}{Chameau}
\monstercarac{description}{
  Animal irascible adapté à la vie dans les climats secs et souvent
employé pour le transport dans les déserts.

}
\monsterdetail{De mauvaise humeur}{Il mord ou donne un coup de pied aux créatures sur son chemin, y compris
ses propriétaires.
}
\monsterdetail{Eau}{Après avoir bien bu, il peut survivre 2 semaines sans eau.
}
\monsterdetail{Voyage dans le désert}{Se déplace à pleine vitesse à travers les terres ravagées et les
déserts.
}
\monsterdetail{Bête de somme}{Peut transporter jusqu'à 3 000 pièces en charge normale et jusqu'à 6 000
pièces surchargé et à la moitié de sa vitesse.
}
\monsterdetail{Charge}{Il n'est pas possible de charger lorsqu'on est monté sur un chameau.
}
\monstercarac{ca}{7 [12]}
\monstercarac{hd}{2 (9 pv)}
\monstercarac{taco}{18 [+1]}
\monstercarac{moral}{7}
\monstercarac{alignement}{Neutre}
\monstercarac{xp}{20}
\monstercarac{nombre_donjon}{0}
\monstercarac{nombre_exterieur}{2d4}
\monstercarac{tresor}{None}
\monstercarac{mvt}{45 m (15 m)}
\monstercarac{save_mort_poison}{12}
\monstercarac{save_baguettes}{13}
\monstercarac{save_paralysie_petrification}{14}
\monstercarac{save_souffles}{15}
\monstercarac{save_sorts_sceptres_batons}{16}
\monsterattack{1 × morsure }{1}
\monsterattack{1 × sabot }{1d4}
\end{monster}

\begin{monster}
\monstercarac{img}{InvisibleStalker.jpg}
\monstercarac{name}{Chasseur invisible (Monstre)}
\monstercarac{description}{
  Créature magique très intelligente convoquée depuis un autre plan
d'existence pour servir un puissant magicien.

}
\monsterdetail{Pistage}{Sans défaillance.
}
\monsterdetail{Surprise}{Sur un résultat de 1-à-5, à moins que la cible ne détecte
l'invisibilité.
}
\monsterdetail{S’il est tué}{Retourne dans son plan d'origine.
}
\monstercarac{ca}{3 [16]}
\monstercarac{hd}{8* (36 pv)}
\monstercarac{taco}{12 [+7]}
\monstercarac{moral}{12}
\monstercarac{alignement}{Neutre}
\monstercarac{xp}{1 200}
\monstercarac{nombre_donjon}{1}
\monstercarac{nombre_exterieur}{1}
\monstercarac{tresor}{None}
\monstercarac{mvt}{36 m (12 m)}
\monstercarac{save_mort_poison}{8}
\monstercarac{save_baguettes}{9}
\monstercarac{save_paralysie_petrification}{10}
\monstercarac{save_souffles}{10}
\monstercarac{save_sorts_sceptres_batons}{12}
\monsterattack{1 × coup }{4d4}
\end{monster}

\begin{monster}
\monstercarac{name}{Chauve-souris normale}
\monstercarac{description}{
  Mammifère volant nocturne qui perche dans les cavernes et les ruines.

}
\monsterdetail{Écholocation}{Insensible aux effets qui nuisent à, modifient ou dépendent de la vue.
Aveuglée par un silence d'origine magique.
}
\monsterdetail{ Essaim}{10 chauves-souris peuvent tourner en essaim autour de la tête d'une
cible, la plongeant en pleine confusion : malus de --2 aux jets
d'attaques et de sauvegardes ; impossibilité de lancer des sorts.
}
\monsterdetail{ Attaques}{Comme un humain normal.
}
\monsterdetail{ Peureux}{À moins d'être convoquées ou contrôlées magiquement, les chauves-souris
normales doivent effectuer un jet de moral chaque round ou fuir.
}
\monstercarac{ca}{6 [13]}
\monstercarac{hd}{1 pv}
\monstercarac{taco}{20 [–1]}
\monstercarac{moral}{6}
\monstercarac{alignement}{Neutre}
\monstercarac{xp}{5}
\monstercarac{nombre_donjon}{1d100}
\monstercarac{nombre_exterieur}{1d100}
\monstercarac{tresor}{None}
\monstercarac{mvt}{3 m (1 m) / 36 m (18 m) vol}
\monstercarac{save_mort_poison}{14}
\monstercarac{save_baguettes}{15}
\monstercarac{save_paralysie_petrification}{16}
\monstercarac{save_souffles}{}
\monstercarac{save_sorts_sceptres_batons}{18}
\monsterattack{1 × essaim }{confusion}
\end{monster}

\input{liste/Chauve-souris_géante.tex}
\input{liste/Chauve-souris_vampire_géante.tex}
\begin{monster}
\monstercarac{name}{Cheval de guerre}
\monstercarac{description}{
  Animal vivant en troupeaux et souvent utilisé pour le transport. Il
existe de nombreuses races domestiques différentes.

  Élevé pour sa force et son courage au combat. Adapté aux courtes
accélérations pour charger ; ne convient pas à une chevauchée de longue
distance.

}
\monsterdetail{Charge}{Lorsqu'il n'est pas encore engagé dans la mêlée. Nécessite une
trajectoire dégagée d'au moins 20 m entre lui et sa cible. Si elle
touche, la lance du cavalier inflige le double de dégâts. Le cheval ne
peut pas utiliser ses attaques naturelles lors de la charge.
}
\monsterdetail{Corps à corps}{En mêlée, le cavalier et le cheval peuvent tous les deux attaquer.
}
\monsterdetail{Domestique}{On ne le rencontre pas dans la nature sauvage.
}
\monsterdetail{Bête de somme}{Il peut transporter jusqu'à 4 000 pièces en charge normale et jusqu'à 8
000 pièces surchargé et à la moitié de sa vitesse.
}
\monstercarac{ca}{7 [12]}
\monstercarac{hd}{3 (13 pv)}
\monstercarac{taco}{17 [+2]}
\monstercarac{moral}{9}
\monstercarac{alignement}{Neutre}
\monstercarac{xp}{35}
\monstercarac{nombre_donjon}{0}
\monstercarac{nombre_exterieur}{0}
\monstercarac{tresor}{Aucun}
\monstercarac{mvt}{36 m (12 m)}
\monstercarac{save_mort_poison}{12}
\monstercarac{save_baguettes}{13}
\monstercarac{save_paralysie_petrification}{14}
\monstercarac{save_souffles}{15}
\monstercarac{save_sorts_sceptres_batons}{16}
\monsterattack{2 × sabot }{1d6}
\end{monster}

\begin{monster}
\monstercarac{name}{Cheval de selle}
\monstercarac{description}{
  Animal vivant en troupeaux et souvent utilisé pour le transport. Il
existe de nombreuses races domestiques différentes.

  Chevaux légers adaptés pour courir à grande vitesse. Peut survivre en se
nourrissant uniquement d'herbe, partout où elle est disponible.

}
\monsterdetail{Domestique}{On ne le rencontre pas dans la nature sauvage.
}
\monsterdetail{Bête de somme}{Peut transporter jusqu'à 3 000 pièces en charge normale et jusqu'à 6 000
pièces surchargé et à la moitié de sa vitesse.
}
\monstercarac{ca}{7 [12]}
\monstercarac{hd}{2 (9 pv)}
\monstercarac{taco}{18 [+1]}
\monstercarac{moral}{7}
\monstercarac{alignement}{Neutre}
\monstercarac{xp}{20}
\monstercarac{nombre_donjon}{0}
\monstercarac{nombre_exterieur}{0}
\monstercarac{tresor}{Aucun}
\monstercarac{save_mort_poison}{12}
\monstercarac{save_baguettes}{13}
\monstercarac{save_paralysie_petrification}{14}
\monstercarac{save_souffles}{15}
\monstercarac{save_sorts_sceptres_batons}{16}
\monsterattack{2 × sabot }{1d4)}
\end{monster}

\begin{monster}
\monstercarac{name}{Cheval de trait}
\monstercarac{description}{
  Animal vivant en troupeaux et souvent utilisé pour le transport. Il
existe de nombreuses races domestiques différentes.

  Élevé pour sa grande force et son endurance. Utilisé pour tirer des
véhicules et des charrues ou comme bête de somme.

}
\monsterdetail{Non-combattant}{Il fuit s'il est attaqué.
}
\monsterdetail{Domestique}{On ne le rencontre pas dans la nature sauvage.
}
\monsterdetail{Bête de somme}{Il peut transporter jusqu'à 4 500 pièces en charge normale et jusqu'à 9
000 pièces surchargé et à la moitié de sa vitesse.
}
\monstercarac{ca}{7 [12]}
\monstercarac{hd}{3 (13 pv)}
\monstercarac{taco}{17 [+2]}
\monstercarac{moral}{6}
\monstercarac{alignement}{Neutre}
\monstercarac{xp}{35}
\monstercarac{nombre_donjon}{0}
\monstercarac{nombre_exterieur}{0}
\monstercarac{tresor}{Aucun}
\monstercarac{save_mort_poison}{12}
\monstercarac{save_baguettes}{13}
\monstercarac{save_paralysie_petrification}{14}
\monstercarac{save_souffles}{15}
\monstercarac{save_sorts_sceptres_batons}{16}
\monsterattack{Aucune }{-)}
\end{monster}

\begin{monster}
\monstercarac{name}{Cheval sauvage}
\monstercarac{description}{
  Animal vivant en troupeaux et souvent utilisé pour le transport. Il
existe de nombreuses races domestiques différentes.

  Chevaux légers adaptés pour courir à grande vitesse. Peut survivre en se
nourrissant uniquement d'herbe, partout où elle est disponible.

}
\monsterdetail{Débandade}{Les troupeaux de 20 individus ou plus peuvent piétiner ceux qui se
trouvent sur leur passage. 3-sur-4 chances à chaque round. Bonus de +4
pour toucher les créatures de taille humaine ou plus petites. 1d20
points de dégâts.
}
\monsterdetail{Dressage}{Le cheval sauvage peut être dressé pour servir de monture (cheval de
selle).
}
\monstercarac{ca}{7 [12]}
\monstercarac{hd}{2 (9 pv)}
\monstercarac{taco}{18 [+1]}
\monstercarac{moral}{7}
\monstercarac{alignement}{Neutre}
\monstercarac{xp}{20}
\monstercarac{nombre_donjon}{0}
\monstercarac{nombre_exterieur}{1d10 × 10}
\monstercarac{tresor}{Aucun}
\monstercarac{mvt}{72 m (24 m)}
\monstercarac{save_mort_poison}{12}
\monstercarac{save_baguettes}{13}
\monstercarac{save_paralysie_petrification}{14}
\monstercarac{save_souffles}{15}
\monstercarac{save_sorts_sceptres_batons}{16}
\monsterattack{2 × sabot }{1d4}
\end{monster}

\begin{monster}
\monstercarac{name}{Chien esquiveur}
\monstercarac{description}{
  Chien très intelligent, semblable à un dingo, qui vit en meute et a la
capacité innée de voyager dans et hors de la réalité.

}
\monsterdetail{Intermittence}{En combat, se téléporte près d'un ennemi, attaque, puis réapparaît à 1d4
× 3 m de distance. S'il a l'initiative, il peut utiliser cette capacité
sans que l'adversaire puisse contre-attaquer.
}
\monsterdetail{Disparition}{Au bord de la défaite, la meute peut s'enfuir en disparaissant
entièrement.
}
\monsterdetail{Haine des bêtes dimensionnelles}{Le chien esquiveur les attaque toujours.
}
\monstercarac{ca}{5 [14]}
\monstercarac{hd}{4* (18 pv)}
\monstercarac{taco}{16 [+3]}
\monstercarac{moral}{6}
\monstercarac{alignement}{Loyal}
\monstercarac{xp}{125}
\monstercarac{nombre_donjon}{1d6}
\monstercarac{nombre_exterieur}{1d6}
\monstercarac{tresor}{C}
\monstercarac{save_mort_poison}{10}
\monstercarac{save_baguettes}{11}
\monstercarac{save_paralysie_petrification}{12}
\monstercarac{save_souffles}{13}
\monstercarac{save_sorts_sceptres_batons}{14}
\monsterattack{1 × morsure }{1d6)}
\end{monster}

\input{liste/Chimère.tex}
\begin{monster}
\monstercarac{name}{Cockatrice}
\monstercarac{description}{
  Un hybride magique de petite taille, à la fois oiseau et reptile. Elle
possède une longue queue de serpent et la tête, les pattes et les ailes
d'un coq. Vit dans tous les types d'environnements.

}
\monsterdetail{Pétrification}{Quiconque est touché par une cockatrice doit effectuer un jet de
sauvegarde contre la pétrification.
}
\monstercarac{ca}{6 [13]}
\monstercarac{hd}{5** (22 pv)}
\monstercarac{taco}{15 [+4]}
\monstercarac{moral}{7}
\monstercarac{alignement}{Neutre}
\monstercarac{xp}{425}
\monstercarac{nombre_donjon}{1d4}
\monstercarac{nombre_exterieur}{1d8}
\monstercarac{tresor}{D}
\monstercarac{mvt}{27 m (9 m) / 54 m (18 m) vol}
\monstercarac{save_mort_poison}{10}
\monstercarac{save_baguettes}{11}
\monstercarac{save_paralysie_petrification}{12}
\monstercarac{save_souffles}{13}
\monstercarac{save_sorts_sceptres_batons}{14}
\monsterattack{1 × coup de bec }{1d6 + pétrification}
\end{monster}

\input{liste/Commerçant.tex}
\input{liste/Crabe_géant.tex}
\input{liste/Crapaud_géant.tex}
\begin{monster}
\monstercarac{name}{Criard}
\monstercarac{description}{
  Champignon géant qui pousse dans les cavernes et peut se déplacer
lentement.

}
\monsterdetail{Cri}{Déclenché par la lumière (jusqu'à 18 m) ou le mouvement (jusqu'à 9 m).
Le cri dure 1d3 rounds ; à chaque round, il y a 50 \% de chances qu'un
monstre errant soit attiré (il arrive en 2d6 rounds).
}
\monstercarac{ca}{7 [12]}
\monstercarac{hd}{3 (13 pv)}
\monstercarac{taco}{17 [+2]}
\monstercarac{moral}{12}
\monstercarac{alignement}{Neutre}
\monstercarac{xp}{35}
\monstercarac{nombre_donjon}{1d8}
\monstercarac{nombre_exterieur}{0}
\monstercarac{tresor}{Aucun}
\monstercarac{save_mort_poison}{12}
\monstercarac{save_baguettes}{13}
\monstercarac{save_paralysie_petrification}{14}
\monstercarac{save_souffles}{15}
\monstercarac{save_sorts_sceptres_batons}{16}
\monsterattack{Aucune }{-)}
\end{monster}

\begin{monster}
\monstercarac{name}{Crocodile normal}
\monstercarac{description}{
  Ce grand reptile est maladroit sur terre et vit principalement dans
l'eau, se cachant juste sous la surface des marécages subtropicaux et
des rivières à faible débit. En cas de faim, il attaquera toute créature
s'aventurant dans l'eau.

}
\monsterdetail{Frénésie dévorante}{Attirés par l'odeur du sang ou des mouvements violents dans l'eau.
}
\monstercarac{ca}{5 [14]}
\monstercarac{hd}{2 (9 pv)}
\monstercarac{taco}{18 [+1]}
\monstercarac{moral}{7}
\monstercarac{alignement}{Neutre}
\monstercarac{xp}{20}
\monstercarac{nombre_donjon}{0}
\monstercarac{nombre_exterieur}{1d8}
\monstercarac{tresor}{Aucun}
\monstercarac{mvt}{27 m (9 m) / 27 m (9 m) nage}
\monstercarac{save_mort_poison}{12}
\monstercarac{save_baguettes}{13}
\monstercarac{save_paralysie_petrification}{14}
\monstercarac{save_souffles}{15}
\monstercarac{save_sorts_sceptres_batons}{16}
\monsterattack{1 × morsure }{1d8}
\end{monster}

\begin{monster}
\monstercarac{img}{Crocodile.png}
\monstercarac{name}{Grand crocodile}
\monstercarac{description}{
  Ce grand reptile est maladroit sur terre et vit principalement dans
l'eau, se cachant juste sous la surface des marécages subtropicaux et
des rivières à faible débit. En cas de faim, il attaquera toute créature
s'aventurant dans l'eau.

  Long de 6 m ou plus, il peut attaquer les petites embarcations (canoës,
radeaux).

}
\monsterdetail{Frénésie dévorante}{Attirés par l'odeur du sang ou des mouvements violents dans l'eau.
}
\monstercarac{ca}{3 [16]}
\monstercarac{hd}{6 (27 pv)}
\monstercarac{taco}{14 [+5]}
\monstercarac{moral}{7}
\monstercarac{alignement}{Neutre}
\monstercarac{xp}{275}
\monstercarac{nombre_donjon}{0}
\monstercarac{nombre_exterieur}{1d4}
\monstercarac{tresor}{Aucun}
\monstercarac{mvt}{27 m (9 m) / 27 m (9 m) nage}
\monstercarac{save_mort_poison}{12}
\monstercarac{save_baguettes}{13}
\monstercarac{save_paralysie_petrification}{14}
\monstercarac{save_souffles}{15}
\monstercarac{save_sorts_sceptres_batons}{16}
\monsterattack{1 × morsure }{2d8}
\end{monster}

\input{liste/Crocodile_géant.tex}
\input{liste/Cube_gélatineux.tex}
\begin{monster}
\monstercarac{name}{Cyclope}
\monstercarac{description}{
  Humanoïdes hauts de 6 m, qui possèdent un unique œil central et habitent
dans les grottes, seuls ou en petits groupes. Cultivent la vigne et
élèvent des moutons.

}
\monsterdetail{Malus à l’attaque}{Malus de --2 sur tous les jets pour toucher, à cause de leur difficulté
à estimer les distances.
}
\monsterdetail{Lancer de pierres}{Jusqu'à 60 m.
}
\monsterdetail{Esprit lent}{Peut être trompé par des PJ intelligents.
}
\monsterdetail{Malédiction}{1 cyclope sur 20 a la capacité de jeter des malédictions. Il peut le
faire une fois par semaine. La cible doit réussir un jet de sauvegarde
contre les sorts ou être affligée par une malédiction choisie par
l'arbitre. Effets possibles maximums : Malus de --2 aux jets de
sauvegardes, malus de --4 pour toucher, une caractéristique réduite de
50 \%.
}
\monstercarac{ca}{5 [14]}
\monstercarac{hd}{13* (58 pv)}
\monstercarac{taco}{10 [+9]}
\monstercarac{moral}{9}
\monstercarac{alignement}{Chaotique}
\monstercarac{xp}{2 300}
\monstercarac{nombre_donjon}{1}
\monstercarac{nombre_exterieur}{1d4}
\monstercarac{tresor}{E + 5,000gp}
\monstercarac{mvt}{27 m (9 m)}
\monstercarac{save_mort_poison}{4}
\monstercarac{save_baguettes}{5}
\monstercarac{save_paralysie_petrification}{6}
\monstercarac{save_souffles}{5}
\monstercarac{save_sorts_sceptres_batons}{8}
\monsterattack{1 × gourdin }{3d10}
\monsterattack{1 × rocher }{3d6}
\end{monster}

\begin{monster}
\monstercarac{img}{Dervish.png}
\monstercarac{name}{Derviche}
\monstercarac{description}{
  Fanatique religieux nomade qui vit dans des tentes et erre dans les
steppes et les régions désertiques.

}
\monsterdetail{Monté}{Sur des chevaux de guerre.
}
\monsterdetail{Armes}{50 \% du groupe est équipé avec : une armure de cuir, un bouclier, une
lance ; 25 \% est équipé avec : une armure de cuir, un bouclier, un arc
court, une épée ; 25 \% est équipés avec : une cotte de mailles, un
bouclier, une lance.
}
\monsterdetail{Camps}{Les groupes se rassemblent et vivent généralement dans un camp ou une
tribu comptant jusqu'à 300 derviches, plus les femmes, les enfants et
les animaux. Il y a 25 \% de chances d'y avoir une enceinte en bois ou
en brique.
}
\monsterdetail{Chefs de camps}{Un camp est dirigé par un clerc de 10e niveau.
}
\monsterdetail{Intolérant}{Aux autres opinions religieuses.
}
\monsterdetail{Guerre sainte}{Font parfois la guerre à d'autres factions religieuses. Ils tuent alors
les gens de confessions différentes ou les font prisonniers. Les
prisonniers doivent se convertir, sinon ils sont tués ou servent
d'esclave. Les personnages d'alignement Loyal peuvent être invités à
rejoindre la guerre sainte. Les derviches se méfient fortement de ceux
qui refusent (sans raison valable).
}
\monstercarac{ca}{6 [13] or 4 [15]}
\monstercarac{hd}{1 (4 pv)}
\monstercarac{taco}{19 [0]}
\monstercarac{moral}{10}
\monstercarac{alignement}{Loyal}
\monstercarac{xp}{10}
\monstercarac{nombre_donjon}{0}
\monstercarac{nombre_exterieur}{1d6+1 × 10}
\monstercarac{tresor}{A}
\monstercarac{mvt}{27 m (9 m)}
\monstercarac{save_mort_poison}{12}
\monstercarac{save_baguettes}{13}
\monstercarac{save_paralysie_petrification}{14}
\monstercarac{save_souffles}{15}
\monstercarac{save_sorts_sceptres_batons}{16}
\monsterattack{1 × arme }{1d6 }
\monsterattack{selon l’arme)}{}
\end{monster}

\begin{monster}
\monstercarac{name}{Djinn (mineur)}
\monstercarac{description}{
  Être hautement magique, libre et intelligent, provenant du plan
élémentaire de l'air. Grand humanoïde enveloppé de nuages.

}
\monsterdetail{Immunité aux dégâts normaux}{Ne peut être blessé que par des attaques magiques.
}
\monsterdetail{Pouvoirs magiques}{Chaque pouvoir peut être utilisé trois fois par jour : 1. \textbf{Forme
de tourbillon :} Il faut 5 rounds au Djinn pour se transformer (ou
reprendre une forme humanoïde). Le tourbillon fait 21 m de haut, 6 m de
large à son sommet pour 3 m de large à sa base. Il se déplace à une
vitesse de 36 m (12 m) et inflige 2d6 points de dégâts à toute créature
se trouvant sur son chemin. Les créatures de moins de 2 DV sont balayées
(jet de sauvegarde contre la mort). 2. \textbf{Forme gazeuse} 3.
\textbf{Invisibilité} 4. \textbf{Illusion :} Visuelle et auditive.
L'illusion ne demande aucune concentration au djinn. Elle persiste
jusqu'à ce qu'elle soit touchée ou dissipée. 5. \textbf{Création de
nourriture et de boisson :} Pour 12 humains et leurs montures pendant un
jour. 6. \textbf{Invoquer des objets métalliques :} Jusqu'à un poids de
1 000 pièces. Temporaires : Le type de métal choisi détermine la durée
(or : 1 jour ; fer : 1 round). 7. \textbf{Invoquer des objets usuels /
en bois :} Jusqu'à un poids de 1 000 pièces. Permanents.
}
\monsterdetail{Capacité de charge}{Un djinn peut porter 6 000 pièces sans fatigue. Jusqu'à 12 000 pièces
pendant 3 tours en marchant / 1 tour en volant. Il doit ensuite se
reposer 1 tour.
}
\monsterdetail{S’il est tué}{Il retourne dans le plan de l'air.
}
\monstercarac{ca}{5 [14]}
\monstercarac{hd}{7+1* (32 pv)}
\monstercarac{taco}{12 [+7]}
\monstercarac{moral}{12}
\monstercarac{alignement}{Neutre}
\monstercarac{xp}{850}
\monstercarac{nombre_donjon}{1}
\monstercarac{nombre_exterieur}{1}
\monstercarac{tresor}{Aucun}
\monstercarac{save_mort_poison}{4}
\monstercarac{save_baguettes}{5}
\monstercarac{save_paralysie_petrification}{6}
\monstercarac{save_souffles}{5}
\monstercarac{save_sorts_sceptres_batons}{8}
\monsterattack{1 × poings }{2d8)}
\monsterattack{magie }{-)}
\end{monster}

\begin{monster}
\monstercarac{img}{Doppleganger.png}
\monstercarac{name}{Doppelganger}
\monstercarac{description}{
  Changeur de forme de nature magique, malfaisant et très rusé.

}
\monsterdetail{Vol d’apparence}{Peut adopter la forme d'une créature de taille humaine (2,10 m ou moins)
qu'il a observée, avant de l'attaquer.
}
\monsterdetail{Tromperie}{Va tenter de tuer un PJ pour le remplacer, avant d'attaquer le groupe
par surprise (par exemple, pendant un combat).
}
\monsterdetail{Retour}{À sa mort, le doppelganger reprend sa forme naturelle.
}
\monsterdetail{Immunité aux sorts}{N'est pas affecté par les sorts sommeil et charme.
}
\monstercarac{ca}{5 [14]}
\monstercarac{hd}{4* (18 pv)}
\monstercarac{taco}{16 [+3]}
\monstercarac{moral}{10}
\monstercarac{alignement}{Chaotique}
\monstercarac{xp}{125}
\monstercarac{nombre_donjon}{1d6}
\monstercarac{nombre_exterieur}{1d6}
\monstercarac{tresor}{E}
\monstercarac{mvt}{27 m (9 m)}
\monstercarac{save_mort_poison}{6}
\monstercarac{save_baguettes}{7}
\monstercarac{save_paralysie_petrification}{8}
\monstercarac{save_souffles}{8}
\monstercarac{save_sorts_sceptres_batons}{10}
\monsterattack{1 × morsure }{1d12}
\end{monster}

\begin{monster}
\monstercarac{name}{Dragon-tortue}
\monstercarac{description}{
}
\monsterdetail{Souffle}{Nuage de vapeur de 27 m de long et de 9 m de large. Peut être utilisé
jusqu'à trois fois par jour. Toutes les personnes présentes dans la zone
subissent des dégâts égaux aux points de vie actuels du dragon-tortue
(jet de sauvegarde contre les souffles pour n'en subir que la moitié).
}
\monsterdetail{Pris pour une île}{Lorsqu'il flotte, on peut facilement confondre le dragon-tortue avec une
petite île.
}
\monsterdetail{Attaquer les navires}{Il essaie parfois, en faisant surface sous eux, de détruire les navires
et de dévorer l'équipage.
}
\monsterdetail{Trésor}{Provenant des navires coulés.
}
\monstercarac{ca}{–2 [21]}
\monstercarac{hd}{30* (135 pv)}
\monstercarac{taco}{5 [+14]}
\monstercarac{moral}{10}
\monstercarac{alignement}{Chaotique}
\monstercarac{xp}{9 000}
\monstercarac{nombre_donjon}{0}
\monstercarac{nombre_exterieur}{1}
\monstercarac{tresor}{H}
\monstercarac{save_mort_poison}{4}
\monstercarac{save_baguettes}{5}
\monstercarac{save_paralysie_petrification}{6}
\monstercarac{save_souffles}{5}
\monstercarac{save_sorts_sceptres_batons}{8}
\monsterattack{2 × griffes }{1d8)}
\monsterattack{1 × morsure }{1d6 × 10) ou souffle}
\end{monster}

\begin{monster}
\monstercarac{name}{Dragon blanc}
\monstercarac{description}{
  Une race fière et ancienne de reptiles gigantesques, carnivores et
ailés. Il existe de nombreuses sous-espèces de dragons, dont beaucoup se
distinguent par la couleur de leurs écailles. Tous les dragons couvent
leurs œufs et amassent un trésor dans leurs repaires, loin des zones de
la civilisation humaine.

  On le trouve dans les régions froides.

}
\monsterdetail{Comportement}{Les dragons chaotiques essaient généralement de dévorer les humains,
mais ils peuvent parfois simplement les capturer. Les dragons neutres
peuvent attaquer ou ignorer les humains. Les dragons loyaux peuvent
aider les groupes qu'ils jugent dignes de cet honneur.
}
\monsterdetail{Fierté}{Les dragons sont des créatures extrêmement fières et auront toujours une
oreille attentive à la flatterie.
}
\monsterdetail{Schéma d’attaque}{Un dragon attaque toujours en premier avec son souffle, puis il souffle
à nouveau ou effectue des attaques de mêlée (à probabilité égale).
}
\monsterdetail{Souffle - général}{Peut être utilisé jusqu'à trois fois par jour. Toutes les créatures se
trouvant dans la zone concernée subissent des dégâts égaux aux points de
vie actuels du dragon (jet de sauvegarde contre les souffles divise par
deux). Formes de souffle :

\begin{itemize}
\tightlist
\item
  \textbf{Nuage :} 15 m de long, 12 m de large, 6 m de haut.
\item
  \textbf{Cône :} 60 cm de large à l'entrée, 9 m à l'extrémité.
\item
  \textbf{Ligne :} 1,50 m de large sur toute la longueur.
\end{itemize}
}
\monsterdetail{Immunité naturelle}{Les dragons sont immunisés aux dégâts de leur propre souffle ou à des
versions inférieures du même type. Ils réussissent automatiquement leurs
jets de sauvegarde contre les formes d'attaques similaires. Par exemple,
un dragon rouge est immunisé contre l'huile enflammée et ne subit que la
moitié des dégâts du sort boule de feu.
}
\monsterdetail{Langue et sorts}{Certains dragons sont capables de parler (leur propre langue, ainsi que
le Commun). La probabilité est répertoriée par sous-espèces. Ceux qui
sont doués de parole peuvent également lancer des sorts de magicien
choisis au hasard (le nombre et le niveau des sorts répertoriés).
}
\monsterdetail{Dormir}{Les chances qu'un dragon soit endormi lorsqu'on le rencontre sont
répertoriées par sous-espèces. On peut attaquer un dragon endormi
pendant un round avec un bonus de +2 pour toucher. Mais attention : les
dragons peuvent parfois faire semblant de dormir !
}
\monsterdetail{Soumettre}{Le dragon se rendra s'il est réduit à 0 pv par des attaques non létales
(voir Soumettre dans \href{/Combat_:_autres_détails}{Combat : autres
détails}), admettant qu'il a été vaincu. Notez que les dégâts de
soumission ne réduisent pas les dégâts infligés par le souffle. Un
dragon soumis tentera de s'échapper ou d'attaquer ses ravisseurs, si
l'occasion se présente ou s'il reçoit un ordre suicidaire. Un dragon
soumis peut être vendu jusqu'à 1 000 po par pv.
}
\monsterdetail{Âge}{Les statistiques suivantes définissent des dragons de taille moyenne.
Les jeunes dragons peuvent avoir jusqu'à 3 DV de moins et ne posséder
que ¼ à ½ trésors. Les dragons plus âgés peuvent avoir jusqu'à 3 DV de
plus et deux fois plus de trésors.
}
\monsterdetail{Repaire}{Le trésor d'un dragon est toujours conservé dans son repaire bien caché
et est rarement laissé sans surveillance.
}
\monsterdetail{Souffle}{Cône de froid de 24 m de long.
}
\monsterdetail{Langue et sorts}{10 \% ; 3 × 1er niveau.
}
\monsterdetail{Sommeil}{50 \%.
}
\monstercarac{ca}{3 [16]}
\monstercarac{hd}{6** (27 pv)}
\monstercarac{taco}{14 [+5]}
\monstercarac{moral}{8}
\monstercarac{alignement}{Neutre}
\monstercarac{xp}{725}
\monstercarac{nombre_donjon}{1d4}
\monstercarac{nombre_exterieur}{1d4}
\monstercarac{tresor}{H}
\monstercarac{mvt}{27 m (9 m) / 72 m (24 m) vol}
\monstercarac{save_mort_poison}{10}
\monstercarac{save_baguettes}{11}
\monstercarac{save_paralysie_petrification}{12}
\monstercarac{save_souffles}{13}
\monstercarac{save_sorts_sceptres_batons}{14}
\monsterattack{2 × griffes }{1d4}
\monsterattack{1 × morsure }{2d8}
\monsterattack{souffle}{}
\end{monster}

\begin{monster}
\monstercarac{name}{Dragon bleu}
\monstercarac{description}{
  Une race fière et ancienne de reptiles gigantesques, carnivores et
ailés. Il existe de nombreuses sous-espèces de dragons, dont beaucoup se
distinguent par la couleur de leurs écailles. Tous les dragons couvent
leurs œufs et amassent un trésor dans leurs repaires, loin des zones de
la civilisation humaine.

  Privilégie les plaines ouvertes et les déserts.

}
\monsterdetail{Comportement}{Les dragons chaotiques essaient généralement de dévorer les humains,
mais ils peuvent parfois simplement les capturer. Les dragons neutres
peuvent attaquer ou ignorer les humains. Les dragons loyaux peuvent
aider les groupes qu'ils jugent dignes de cet honneur.
}
\monsterdetail{Fierté}{Les dragons sont des créatures extrêmement fières et auront toujours une
oreille attentive à la flatterie.
}
\monsterdetail{Schéma d’attaque}{Un dragon attaque toujours en premier avec son souffle, puis il souffle
à nouveau ou effectue des attaques de mêlée (à probabilité égale).
}
\monsterdetail{Souffle - général}{Peut être utilisé jusqu'à trois fois par jour. Toutes les créatures se
trouvant dans la zone concernée subissent des dégâts égaux aux points de
vie actuels du dragon (jet de sauvegarde contre les souffles divise par
deux). Formes de souffle :

\begin{itemize}
\tightlist
\item
  \textbf{Nuage :} 15 m de long, 12 m de large, 6 m de haut.
\item
  \textbf{Cône :} 60 cm de large à l'entrée, 9 m à l'extrémité.
\item
  \textbf{Ligne :} 1,50 m de large sur toute la longueur.
\end{itemize}
}
\monsterdetail{Immunité naturelle}{Les dragons sont immunisés aux dégâts de leur propre souffle ou à des
versions inférieures du même type. Ils réussissent automatiquement leurs
jets de sauvegarde contre les formes d'attaques similaires. Par exemple,
un dragon rouge est immunisé contre l'huile enflammée et ne subit que la
moitié des dégâts du sort boule de feu.
}
\monsterdetail{Langue et sorts}{Certains dragons sont capables de parler (leur propre langue, ainsi que
le Commun). La probabilité est répertoriée par sous-espèces. Ceux qui
sont doués de parole peuvent également lancer des sorts de magicien
choisis au hasard (le nombre et le niveau des sorts répertoriés).
}
\monsterdetail{Dormir}{Les chances qu'un dragon soit endormi lorsqu'on le rencontre sont
répertoriées par sous-espèces. On peut attaquer un dragon endormi
pendant un round avec un bonus de +2 pour toucher. Mais attention : les
dragons peuvent parfois faire semblant de dormir !
}
\monsterdetail{Soumettre}{Le dragon se rendra s'il est réduit à 0 pv par des attaques non létales
(voir Soumettre dans \href{/Combat_:_autres_détails}{Combat : autres
détails}), admettant qu'il a été vaincu. Notez que les dégâts de
soumission ne réduisent pas les dégâts infligés par le souffle. Un
dragon soumis tentera de s'échapper ou d'attaquer ses ravisseurs, si
l'occasion se présente ou s'il reçoit un ordre suicidaire. Un dragon
soumis peut être vendu jusqu'à 1 000 po par pv.
}
\monsterdetail{Âge}{Les statistiques suivantes définissent des dragons de taille moyenne.
Les jeunes dragons peuvent avoir jusqu'à 3 DV de moins et ne posséder
que ¼ à ½ trésors. Les dragons plus âgés peuvent avoir jusqu'à 3 DV de
plus et deux fois plus de trésors.
}
\monsterdetail{Repaire}{Le trésor d'un dragon est toujours conservé dans son repaire bien caché
et est rarement laissé sans surveillance.
}
\monsterdetail{Souffle}{Ligne de foudre de 30 m de long.
}
\monsterdetail{Langue et sorts}{40 \% ; 4 × 1er niveau, 4 × 2e niveau.
}
\monsterdetail{Sommeil}{20 \%.
}
\monstercarac{ca}{0 [19]}
\monstercarac{hd}{9** (40 pv)}
\monstercarac{taco}{12 [+7]}
\monstercarac{moral}{9}
\monstercarac{alignement}{Neutre}
\monstercarac{xp}{2 300}
\monstercarac{nombre_donjon}{1d4}
\monstercarac{nombre_exterieur}{1d4}
\monstercarac{tresor}{H}
\monstercarac{mvt}{27 m (9 m) / 72 m (24 m) vol}
\monstercarac{save_mort_poison}{8}
\monstercarac{save_baguettes}{9}
\monstercarac{save_paralysie_petrification}{10}
\monstercarac{save_souffles}{10}
\monstercarac{save_sorts_sceptres_batons}{12}
\monsterattack{2 × griffes }{1d6 + 1}
\monsterattack{1 × morsure }{3d10}
\monsterattack{souffle}{}
\end{monster}

\begin{monster}
\monstercarac{img}{Sea Dragon.jpg}
\monstercarac{name}{Dragon de mer}
\monstercarac{description}{
	\small
  Une race fière et ancienne de reptiles gigantesques, carnivores et
ailés. Il existe de nombreuses sous-espèces de dragons, dont beaucoup se
distinguent par la couleur de leurs écailles. Tous les dragons couvent
leurs œufs et amassent un trésor dans leurs repaires, loin des zones de
la civilisation humaine.

  Ce dragon aquatique intelligent aux écailles vertes, avec des ailes
ressemblant à des nageoires et une crête jaunâtre, habite dans les
profondeurs de l'océan, dans des cavernes ou des épaves. Il attaque
parfois des navires pour voler leurs trésors et dévorer l'équipage.

}
\monsterdetail{Comportement}{Les dragons chaotiques essaient généralement de dévorer les humains,
mais ils peuvent parfois simplement les capturer. Les dragons neutres
peuvent attaquer ou ignorer les humains. Les dragons loyaux peuvent
aider les groupes qu'ils jugent dignes de cet honneur.
}
\monsterdetail{Fierté}{Les dragons sont des créatures extrêmement fières et auront toujours une
oreille attentive à la flatterie.
}
\monsterdetail{Schéma d’attaque}{Un dragon attaque toujours en premier avec son souffle, puis il souffle
à nouveau ou effectue des attaques de mêlée (à probabilité égale).
}
\monsterdetail{Souffle - général}{Peut être utilisé jusqu'à trois fois par jour. Toutes les créatures se
trouvant dans la zone concernée subissent des dégâts égaux aux points de
vie actuels du dragon (jet de sauvegarde contre les souffles divise par
deux). Formes de souffle :

\begin{itemize}
\tightlist
\item
  \textbf{Nuage :} 15 m de long, 12 m de large, 6 m de haut.
\item
  \textbf{Cône :} 60 cm de large à l'entrée, 9 m à l'extrémité.
\item
  \textbf{Ligne :} 1,50 m de large sur toute la longueur.
\end{itemize}
}
\monsterdetail{Immunité naturelle}{Les dragons sont immunisés aux dégâts de leur propre souffle ou à des
versions inférieures du même type. Ils réussissent automatiquement leurs
jets de sauvegarde contre les formes d'attaques similaires. Par exemple,
un dragon rouge est immunisé contre l'huile enflammée et ne subit que la
moitié des dégâts du sort boule de feu.
}
\monsterdetail{Langue et sorts}{Certains dragons sont capables de parler (leur propre langue, ainsi que
le Commun). La probabilité est répertoriée par sous-espèces. Ceux qui
sont doués de parole peuvent également lancer des sorts de magicien
choisis au hasard (le nombre et le niveau des sorts répertoriés).
}
\monsterdetail{Dormir}{Les chances qu'un dragon soit endormi lorsqu'on le rencontre sont
répertoriées par sous-espèces. On peut attaquer un dragon endormi
pendant un round avec un bonus de +2 pour toucher. Mais attention : les
dragons peuvent parfois faire semblant de dormir !
}
\monsterdetail{Soumettre}{Le dragon se rendra s'il est réduit à 0 pv par des attaques non létales
(voir Soumettre dans \href{/Combat_:_autres_détails}{Combat : autres
détails}), admettant qu'il a été vaincu. Notez que les dégâts de
soumission ne réduisent pas les dégâts infligés par le souffle. Un
dragon soumis tentera de s'échapper ou d'attaquer ses ravisseurs, si
l'occasion se présente ou s'il reçoit un ordre suicidaire. Un dragon
soumis peut être vendu jusqu'à 1 000 po par pv.
}
\monsterdetail{Âge}{Les statistiques suivantes définissent des dragons de taille moyenne.
Les jeunes dragons peuvent avoir jusqu'à 3 DV de moins et ne posséder
que ¼ à ½ trésors. Les dragons plus âgés peuvent avoir jusqu'à 3 DV de
plus et deux fois plus de trésors.
}
\monsterdetail{Repaire}{Le trésor d'un dragon est toujours conservé dans son repaire bien caché
et est rarement laissé sans surveillance.
}
\monsterdetail{Souffle}{Crachat de poison de 6 m de diamètre et d'une portée de 30 m. Jet de
sauvegarde contre les souffles ou meurt (le poison est inoffensif après
1 round).
}
\monsterdetail{Langue et sorts}{20 \% ; 3 × 1er niveau, 3 × 2e niveau.
}
\monsterdetail{Sommeil}{30 \%.
}
\monsterdetail{Vol plané}{Le dragon de mer peut sauter hors de l'eau et planer jusqu'à 6 tours.
}
\monstercarac{ca}{1 [18]}
\monstercarac{hd}{8** (36 pv)}
\monstercarac{taco}{12 [+7]}
\monstercarac{moral}{9}
\monstercarac{alignement}{Neutre}
\monstercarac{xp}{1 750}
\monstercarac{nombre_donjon}{0}
\monstercarac{nombre_exterieur}{1d4}
\monstercarac{tresor}{H}
\monstercarac{mvt}{54 m (18 m) nage / 54 m (18 m) vol plané}
\monstercarac{save_mort_poison}{8}
\monstercarac{save_baguettes}{9}
\monstercarac{save_paralysie_petrification}{10}
\monstercarac{save_souffles}{10}
\monstercarac{save_sorts_sceptres_batons}{12}
\monsterattack{1 × morsure }{3d8}
\monsterattack{souffle}{}
\end{monster}

\begin{monster}
\monstercarac{name}{Dragon noir}
\monstercarac{description}{
  Une race fière et ancienne de reptiles gigantesques, carnivores et
ailés. Il existe de nombreuses sous-espèces de dragons, dont beaucoup se
distinguent par la couleur de leurs écailles. Tous les dragons couvent
leurs œufs et amassent un trésor dans leurs repaires, loin des zones de
la civilisation humaine.

  Habite dans les marécages et les marais.

}
\monsterdetail{Comportement}{Les dragons chaotiques essaient généralement de dévorer les humains,
mais ils peuvent parfois simplement les capturer. Les dragons neutres
peuvent attaquer ou ignorer les humains. Les dragons loyaux peuvent
aider les groupes qu'ils jugent dignes de cet honneur.
}
\monsterdetail{Fierté}{Les dragons sont des créatures extrêmement fières et auront toujours une
oreille attentive à la flatterie.
}
\monsterdetail{Schéma d’attaque}{Un dragon attaque toujours en premier avec son souffle, puis il souffle
à nouveau ou effectue des attaques de mêlée (à probabilité égale).
}
\monsterdetail{Souffle}{Peut être utilisé jusqu'à trois fois par jour. Toutes les créatures se
trouvant dans la zone concernée subissent des dégâts égaux aux points de
vie actuels du dragon (jet de sauvegarde contre les souffles divise par
deux). Formes de souffle : - \textbf{Nuage :} 15 m de long, 12 m de
large, 6 m de haut. - \textbf{Cône :} 60 cm de large à l'entrée, 9 m à
l'extrémité. - \textbf{Ligne :} 1,50 m de large sur toute la longueur.
}
\monsterdetail{Immunité naturelle}{Les dragons sont immunisés aux dégâts de leur propre souffle ou à des
versions inférieures du même type. Ils réussissent automatiquement leurs
jets de sauvegarde contre les formes d'attaques similaires. Par exemple,
un dragon rouge est immunisé contre l'huile enflammée et ne subit que la
moitié des dégâts du sort boule de feu.
}
\monsterdetail{Langue et sorts}{Certains dragons sont capables de parler (leur propre langue, ainsi que
le Commun). La probabilité est répertoriée par sous-espèces. Ceux qui
sont doués de parole peuvent également lancer des sorts de magicien
choisis au hasard (le nombre et le niveau des sorts répertoriés).
}
\monsterdetail{Dormir}{Les chances qu'un dragon soit endormi lorsqu'on le rencontre sont
répertoriées par sous-espèces. On peut attaquer un dragon endormi
pendant un round avec un bonus de +2 pour toucher. Mais attention : les
dragons peuvent parfois faire semblant de dormir !
}
\monsterdetail{Soumettre}{Le dragon se rendra s'il est réduit à 0 pv par des attaques non létales
(voir Soumettre dans \href{/Combat_:_autres_détails}{Combat : autres
détails}), admettant qu'il a été vaincu. Notez que les dégâts de
soumission ne réduisent pas les dégâts infligés par le souffle. Un
dragon soumis tentera de s'échapper ou d'attaquer ses ravisseurs, si
l'occasion se présente ou s'il reçoit un ordre suicidaire. Un dragon
soumis peut être vendu jusqu'à 1 000 po par pv.
}
\monsterdetail{Âge}{Les statistiques suivantes définissent des dragons de taille moyenne.
Les jeunes dragons peuvent avoir jusqu'à 3 DV de moins et ne posséder
que ¼ à ½ trésors. Les dragons plus âgés peuvent avoir jusqu'à 3 DV de
plus et deux fois plus de trésors.
}
\monsterdetail{Repaire}{Le trésor d'un dragon est toujours conservé dans son repaire bien caché
et est rarement laissé sans surveillance.
}
\monsterdetail{Souffle}{Ligne d'acide de 18 m de long.
}
\monsterdetail{Langue et sorts}{20 \% ; 4 × 1er niveau.
}
\monsterdetail{Sommeil}{40 \%.
}
\monstercarac{ca}{2 [17]}
\monstercarac{hd}{7** (31 pv)}
\monstercarac{taco}{13 [+6]}
\monstercarac{moral}{8}
\monstercarac{alignement}{Chaotique}
\monstercarac{xp}{1 250}
\monstercarac{nombre_donjon}{1d4}
\monstercarac{nombre_exterieur}{1d4}
\monstercarac{tresor}{H}
\monstercarac{mvt}{27 m (9 m) / 72 m (24 m) vol}
\monstercarac{save_mort_poison}{8}
\monstercarac{save_baguettes}{9}
\monstercarac{save_paralysie_petrification}{10}
\monstercarac{save_souffles}{10}
\monstercarac{save_sorts_sceptres_batons}{12}
\monsterattack{2 × griffes }{1d4+1}
\monsterattack{1 × morsure }{2d10}
\monsterattack{souffle}{}
\end{monster}

\input{liste/Dragon_d’or.tex}
\begin{monster}
\monstercarac{name}{Dragon rouge}
\monstercarac{description}{
  Une race fière et ancienne de reptiles gigantesques, carnivores et
ailés. Il existe de nombreuses sous-espèces de dragons, dont beaucoup se
distinguent par la couleur de leurs écailles. Tous les dragons couvent
leurs œufs et amassent un trésor dans leurs repaires, loin des zones de
la civilisation humaine.

  Habite dans les collines et les montagnes.

}
\monsterdetail{Comportement}{Les dragons chaotiques essaient généralement de dévorer les humains,
mais ils peuvent parfois simplement les capturer. Les dragons neutres
peuvent attaquer ou ignorer les humains. Les dragons loyaux peuvent
aider les groupes qu'ils jugent dignes de cet honneur.
}
\monsterdetail{Fierté}{Les dragons sont des créatures extrêmement fières et auront toujours une
oreille attentive à la flatterie.
}
\monsterdetail{Schéma d’attaque}{Un dragon attaque toujours en premier avec son souffle, puis il souffle
à nouveau ou effectue des attaques de mêlée (à probabilité égale).
}
\monsterdetail{Souffle - général}{Peut être utilisé jusqu'à trois fois par jour. Toutes les créatures se
trouvant dans la zone concernée subissent des dégâts égaux aux points de
vie actuels du dragon (jet de sauvegarde contre les souffles divise par
deux). Formes de souffle :

\begin{itemize}
\tightlist
\item
  \textbf{Nuage :} 15 m de long, 12 m de large, 6 m de haut.
\item
  \textbf{Cône :} 60 cm de large à l'entrée, 9 m à l'extrémité.
\item
  \textbf{Ligne :} 1,50 m de large sur toute la longueur.
\end{itemize}
}
\monsterdetail{Immunité naturelle}{Les dragons sont immunisés aux dégâts de leur propre souffle ou à des
versions inférieures du même type. Ils réussissent automatiquement leurs
jets de sauvegarde contre les formes d'attaques similaires. Par exemple,
un dragon rouge est immunisé contre l'huile enflammée et ne subit que la
moitié des dégâts du sort boule de feu.
}
\monsterdetail{Langue et sorts}{Certains dragons sont capables de parler (leur propre langue, ainsi que
le Commun). La probabilité est répertoriée par sous-espèces. Ceux qui
sont doués de parole peuvent également lancer des sorts de magicien
choisis au hasard (le nombre et le niveau des sorts répertoriés).
}
\monsterdetail{Dormir}{Les chances qu'un dragon soit endormi lorsqu'on le rencontre sont
répertoriées par sous-espèces. On peut attaquer un dragon endormi
pendant un round avec un bonus de +2 pour toucher. Mais attention : les
dragons peuvent parfois faire semblant de dormir !
}
\monsterdetail{Soumettre}{Le dragon se rendra s'il est réduit à 0 pv par des attaques non létales
(voir Soumettre dans \href{/Combat_:_autres_détails}{Combat : autres
détails}), admettant qu'il a été vaincu. Notez que les dégâts de
soumission ne réduisent pas les dégâts infligés par le souffle. Un
dragon soumis tentera de s'échapper ou d'attaquer ses ravisseurs, si
l'occasion se présente ou s'il reçoit un ordre suicidaire. Un dragon
soumis peut être vendu jusqu'à 1 000 po par pv.
}
\monsterdetail{Âge}{Les statistiques suivantes définissent des dragons de taille moyenne.
Les jeunes dragons peuvent avoir jusqu'à 3 DV de moins et ne posséder
que ¼ à ½ trésors. Les dragons plus âgés peuvent avoir jusqu'à 3 DV de
plus et deux fois plus de trésors.
}
\monsterdetail{Repaire}{Le trésor d'un dragon est toujours conservé dans son repaire bien caché
et est rarement laissé sans surveillance.
}
\monsterdetail{Souffle}{Cône de feu de 27 m de long.
}
\monsterdetail{Langue et sorts}{50 \% ; 3 × 1er niveau, 3 × 2e niveau, 3 × 3e niveau.
}
\monsterdetail{Sommeil}{10 \%.
}
\monstercarac{ca}{–1 [20]}
\monstercarac{hd}{10** (45 pv)}
\monstercarac{taco}{11 [+8]}
\monstercarac{moral}{10}
\monstercarac{alignement}{Chaotique}
\monstercarac{xp}{2 300}
\monstercarac{nombre_donjon}{1d4}
\monstercarac{nombre_exterieur}{1d4}
\monstercarac{tresor}{H}
\monstercarac{mvt}{27 m (9 m) / 72 m (24 m) vol}
\monstercarac{save_mort_poison}{6}
\monstercarac{save_baguettes}{7}
\monstercarac{save_paralysie_petrification}{8}
\monstercarac{save_souffles}{8}
\monstercarac{save_sorts_sceptres_batons}{10}
\monsterattack{2 × griffes }{1d8}
\monsterattack{1 × morsure }{4d8}
\monsterattack{souffle}{}
\end{monster}

\begin{monster}
\monstercarac{name}{Dragon vert}
\monstercarac{description}{
  Une race fière et ancienne de reptiles gigantesques, carnivores et
ailés. Il existe de nombreuses sous-espèces de dragons, dont beaucoup se
distinguent par la couleur de leurs écailles. Tous les dragons couvent
leurs œufs et amassent un trésor dans leurs repaires, loin des zones de
la civilisation humaine.

  Son repaire se trouve dans les jungles et les forêts.

}
\monsterdetail{Comportement}{Les dragons chaotiques essaient généralement de dévorer les humains,
mais ils peuvent parfois simplement les capturer. Les dragons neutres
peuvent attaquer ou ignorer les humains. Les dragons loyaux peuvent
aider les groupes qu'ils jugent dignes de cet honneur.
}
\monsterdetail{Fierté}{Les dragons sont des créatures extrêmement fières et auront toujours une
oreille attentive à la flatterie.
}
\monsterdetail{Schéma d’attaque}{Un dragon attaque toujours en premier avec son souffle, puis il souffle
à nouveau ou effectue des attaques de mêlée (à probabilité égale).
}
\monsterdetail{Souffle}{Peut être utilisé jusqu'à trois fois par jour. Toutes les créatures se
trouvant dans la zone concernée subissent des dégâts égaux aux points de
vie actuels du dragon (jet de sauvegarde contre les souffles divise par
deux). Formes de souffle : - \textbf{Nuage :} 15 m de long, 12 m de
large, 6 m de haut. - \textbf{Cône :} 60 cm de large à l'entrée, 9 m à
l'extrémité. - \textbf{Ligne :} 1,50 m de large sur toute la longueur.
}
\monsterdetail{Immunité naturelle}{Les dragons sont immunisés aux dégâts de leur propre souffle ou à des
versions inférieures du même type. Ils réussissent automatiquement leurs
jets de sauvegarde contre les formes d'attaques similaires. Par exemple,
un dragon rouge est immunisé contre l'huile enflammée et ne subit que la
moitié des dégâts du sort boule de feu.
}
\monsterdetail{Langue et sorts}{Certains dragons sont capables de parler (leur propre langue, ainsi que
le Commun). La probabilité est répertoriée par sous-espèces. Ceux qui
sont doués de parole peuvent également lancer des sorts de magicien
choisis au hasard (le nombre et le niveau des sorts répertoriés).
}
\monsterdetail{Dormir}{Les chances qu'un dragon soit endormi lorsqu'on le rencontre sont
répertoriées par sous-espèces. On peut attaquer un dragon endormi
pendant un round avec un bonus de +2 pour toucher. Mais attention : les
dragons peuvent parfois faire semblant de dormir !
}
\monsterdetail{Soumettre}{Le dragon se rendra s'il est réduit à 0 pv par des attaques non létales
(voir Soumettre dans \href{/Combat_:_autres_détails}{Combat : autres
détails}), admettant qu'il a été vaincu. Notez que les dégâts de
soumission ne réduisent pas les dégâts infligés par le souffle. Un
dragon soumis tentera de s'échapper ou d'attaquer ses ravisseurs, si
l'occasion se présente ou s'il reçoit un ordre suicidaire. Un dragon
soumis peut être vendu jusqu'à 1 000 po par pv.
}
\monsterdetail{Âge}{Les statistiques suivantes définissent des dragons de taille moyenne.
Les jeunes dragons peuvent avoir jusqu'à 3 DV de moins et ne posséder
que ¼ à ½ trésors. Les dragons plus âgés peuvent avoir jusqu'à 3 DV de
plus et deux fois plus de trésors.
}
\monsterdetail{Repaire}{Le trésor d'un dragon est toujours conservé dans son repaire bien caché
et est rarement laissé sans surveillance.
}
\monsterdetail{Souffle}{Nuage de gaz chloré.
}
\monsterdetail{Langue et sorts}{30 \% ; 3 × 1er niveau, 3 × 2e niveau.
}
\monsterdetail{Sommeil}{30 \%.
}
\monstercarac{ca}{1 [18]}
\monstercarac{hd}{8** (36 pv)}
\monstercarac{taco}{12 [+7]}
\monstercarac{moral}{9}
\monstercarac{alignement}{Chaotique}
\monstercarac{xp}{1 750}
\monstercarac{nombre_donjon}{1d4}
\monstercarac{nombre_exterieur}{1d4}
\monstercarac{tresor}{H}
\monstercarac{save_mort_poison}{8}
\monstercarac{save_baguettes}{9}
\monstercarac{save_paralysie_petrification}{10}
\monstercarac{save_souffles}{10}
\monstercarac{save_sorts_sceptres_batons}{12}
\monsterattack{\[2 × griffes }{1d6)}
\monsterattack{1 × morsure }{3d8)\] ou souffle}
\end{monster}

\begin{monster}
\monstercarac{name}{Dryade}
\monstercarac{description}{
  Esprit des arbres pacifique, craintif et discret. Vivent dans de
profondes forêts et peuvent se manifester sous la forme de femmes
humanoïdes d'une grande beauté.

}
\monsterdetail{Liée aux arbres}{Reliée spirituellement à un arbre. Si cet arbre meurt ou si elle s'en
éloigne de plus de 72 m, elle meurt.
}
\monsterdetail{Fusion avec l’arbre}{Peut disparaître en s'unissant à son arbre.
}
\monsterdetail{Défensive}{Se méfie des étrangers et tente de charmer quiconque l'approche ou la
suit.
}
\monsterdetail{Charme}{La victime est contrainte de s'approcher de l'arbre et disparaît à
l'intérieur (jet de sauvegarde contre les sorts avec un malus de --2).
Si elle n'est pas secourue immédiatement, la victime est perdue à
jamais.
}
\monsterdetail{Trésor}{Caché sous les racines de l'arbre.
}
\monstercarac{ca}{5 [14]}
\monstercarac{hd}{2* (9 pv)}
\monstercarac{taco}{18 [+1]}
\monstercarac{moral}{6}
\monstercarac{alignement}{Neutre}
\monstercarac{xp}{25}
\monstercarac{nombre_donjon}{0}
\monstercarac{nombre_exterieur}{1d6}
\monstercarac{tresor}{D}
\monstercarac{mvt}{36 m (12 m)}
\monstercarac{save_mort_poison}{10}
\monstercarac{save_baguettes}{11}
\monstercarac{save_paralysie_petrification}{12}
\monstercarac{save_souffles}{13}
\monstercarac{save_sorts_sceptres_batons}{14}
\monsterattack{1 × magique }{charme}
\end{monster}

\input{liste/Éfrit_(mineur).tex}
\begin{monster}
\monstercarac{img}{elf.png}
\monstercarac{name}{Elfe (Monstre)}
\monstercarac{description}{
  Ce semi-humain mince et féerique, aux oreilles pointues, vit en harmonie
avec la nature, dans de magnifiques décors naturels.

}
\monsterdetail{Sorts}{Chaque individu connaît un sort arcanique de 1er niveau choisi au
hasard.
}
\monsterdetail{Chef}{Les groupes de 15 elfes ou plus sont dirigés par un elfe de niveau
1d6+1. Le chef peut avoir des objets magiques : 5 \% de chances par
niveau pour chaque tableau d'objets magiques (voir
\href{../../Tresors/Objets_magiques_(généralités).md}{Objets magiques}).
}
\monstercarac{ca}{5 [14]}
\monstercarac{hd}{1+1* (5 pv)}
\monstercarac{taco}{18 [+1]}
\monstercarac{moral}{8 (10 avec un Chef)}
\monstercarac{alignement}{Neutre}
\monstercarac{xp}{19}
\monstercarac{nombre_donjon}{1d4}
\monstercarac{nombre_exterieur}{2d12}
\monstercarac{tresor}{E}
\monstercarac{mvt}{36 m (12 m)}
\monstercarac{save_mort_poison}{12}
\monstercarac{save_baguettes}{13}
\monstercarac{save_paralysie_petrification}{13}
\monstercarac{save_souffles}{15}
\monstercarac{save_sorts_sceptres_batons}{15}
\monsterattack{1 × arme }{1d8 (selon l’arme)}
\end{monster}

\begin{monster}
\monstercarac{name}{Élémental de l’air}
\monstercarac{description}{
  Cet être formé de matière élémentaire pure (air, terre, feu ou eau) peut
être convoqué depuis son plan d'origine afin de servir un magicien.

  Prend la forme d'énormes tourbillons d'air.

}
\monsterdetail{Mineur}{\textbf{Sv} MP 8 B 9 PP 10 S 10 SSB 12 (8). (Convoqué par un bâton
magique)
}
\monsterdetail{Intermédiaire}{(2d8), \textbf{Sv} MP 6 B 7 PP 8 S 8 SSB 10 (12). (Convoqué par un
dispositif magique)
}
\monsterdetail{Supérieur}{(3d8), \textbf{Sv} MP 2 B 3 PP 4 S 3 SSB 6 (16). (Convoqué par un sort)
}
\monsterdetail{Taille}{Hauteur 4,80 m, largeur 1,20 m / Hauteur 7,20 m, largeur 1,80 m /
Hauteur 9,60 m, largeur 2,40 m.
}
\monsterdetail{Tourbillon}{Les créatures ayant moins de 2 DV sont balayées (jet de sauvegarde
contre la mort).
}
\monsterdetail{Immunité aux dégâts normaux}{Ne peut être blessé que par des attaques magiques.* '\,'\,'
}
\monsterdetail{Néfaste aux créatures volantes}{Inflige 1d8 dégâts supplémentaires à ce type de créature.
}
\monstercarac{ca}{2 [17] / 0 [19] / –2 [21]}
\monstercarac{hd}{8/12/16* (36/54/72 pv)}
\monstercarac{taco}{12 [+7] / 10 [+9] / 8 [+11]}
\monstercarac{moral}{10}
\monstercarac{alignement}{Neutre}
\monstercarac{xp}{1 200/1 900/2 300}
\monstercarac{nombre_donjon}{1}
\monstercarac{nombre_exterieur}{1}
\monstercarac{tresor}{Aucun}
\monstercarac{save_mort_poison}{}
\monstercarac{save_baguettes}{}
\monstercarac{save_paralysie_petrification}{}
\monstercarac{save_souffles}{}
\monstercarac{save_sorts_sceptres_batons}{}
\monsterattack{1 × coup }{1d8 / 2d8 / 3d8)}
\end{monster}

\begin{monster}
\monstercarac{img}{Water Elemental.jpg}
\monstercarac{name}{Élémental de l’eau}
\monstercarac{description}{
  Cet être formé de matière élémentaire pure (air, terre, feu ou eau) peut
être convoqué depuis son plan d'origine afin de servir un magicien.

  Prend la forme d'énormes vagues d'eau.

}
\monsterdetail{Mineur}{\textbf{Sv} MP 8 B 9 PP 10 S 10 SSB 12 (8). (Convoqué par un bâton
magique)
}
\monsterdetail{Intermédiaire}{(2d8), \textbf{Sv} MP 6 B 7 PP 8 S 8 SSB 10 (12). (Convoqué par un
dispositif magique)
}
\monsterdetail{Supérieur}{(3d8), \textbf{Sv} MP 2 B 3 PP 4 S 3 SSB 6 (16). (Convoqué par un sort)
}
\monsterdetail{Taille}{Hauteur 1,20 m, largeur 4,80 m / Hauteur 1,80 m, largeur 7,20 m /
Hauteur 2,40 m, largeur 9,60 m.
}
\monsterdetail{À l’eau}{L'élémental doit rester à moins de 18 m de l'eau.
}
\monsterdetail{Immunité aux dégâts normaux}{Ne peut être blessé que par des attaques magiques.
}
\monsterdetail{Néfaste aux créatures dans l’eau}{Inflige 1d8 points de dégâts supplémentaires à ce type de créatures.
}
\monstercarac{ca}{2 [17] / 0 [19] / –2 [21]}
\monstercarac{hd}{8/12/16* (36/54/72 pv)}
\monstercarac{taco}{12 [+7] / 10 [+9] / 8 [+11]}
\monstercarac{moral}{10}
\monstercarac{alignement}{Neutre}
\monstercarac{xp}{1 200/1 900/2 300}
\monstercarac{nombre_donjon}{1}
\monstercarac{nombre_exterieur}{1}
\monstercarac{tresor}{Aucun}
\monstercarac{mvt}{18 m (6 m) / 54 m (18 m) nage}
\monstercarac{save_mort_poison}{}
\monstercarac{save_baguettes}{}
\monstercarac{save_paralysie_petrification}{}
\monstercarac{save_souffles}{}
\monstercarac{save_sorts_sceptres_batons}{}
\monsterattack{1 × coup }{1d8 / 2d8 / 3d8}
\end{monster}

\begin{monster}
\monstercarac{img}{Fire Elemental.jpg}
\monstercarac{name}{Élémental de feu}
\monstercarac{description}{
  Cet être formé de matière élémentaire pure (air, terre, feu ou eau) peut
être convoqué depuis son plan d'origine afin de servir un magicien.

  Prend la forme de colonnes de feu tourbillonnantes.

}
\monsterdetail{Mineur}{\textbf{Sv} MP 8 B 9 PP 10 S 10 SSB 12 (8). (Convoqué par un bâton
magique)
}
\monsterdetail{Intermédiaire}{(2d8), \textbf{Sv} MP 6 B 7 PP 8 S 8 SSB 10 (12). (Convoqué par un
dispositif magique)
}
\monsterdetail{Supérieur}{(3d8), \textbf{Sv} MP 2 B 3 PP 4 S 3 SSB 6 (16). (Convoqué par un sort)
}
\monsterdetail{Taille}{Hauteur 2,40 m, largeur 2,40 m / Hauteur 3,60 m, largeur 3,60 m /
Hauteur 4,80 m, largeur 4,80 m.
}
\monsterdetail{Bloqué par l’eau}{L'élémental ne peut pas traverser une étendue d'eau plus large que son
propre diamètre.
}
\monsterdetail{Immunité aux dégâts normaux}{Ne peut être blessé que par des attaques magiques.
}
\monsterdetail{Néfaste aux créatures basées sur le froid}{Inflige 1d8 dégâts supplémentaires à ce type de créatures.
}
\monstercarac{ca}{2 [17] / 0 [19] / –2 [21]}
\monstercarac{hd}{8/12/16* (36/54/72 pv)}
\monstercarac{taco}{12 [+7] / 10 [+9] / 8 [+11]}
\monstercarac{moral}{10}
\monstercarac{alignement}{Neutre}
\monstercarac{xp}{1 200/1 900/2 300}
\monstercarac{nombre_donjon}{1}
\monstercarac{nombre_exterieur}{1}
\monstercarac{tresor}{Aucun}
\monstercarac{mvt}{36 m (12 m)}
\monstercarac{save_mort_poison}{}
\monstercarac{save_baguettes}{}
\monstercarac{save_paralysie_petrification}{}
\monstercarac{save_souffles}{}
\monstercarac{save_sorts_sceptres_batons}{}
\monsterattack{1 × coup }{1d8 / 2d8 / 3d8}
\end{monster}

\begin{monster}
\monstercarac{name}{Élémental de terre}
\monstercarac{description}{
  Cet être formé de matière élémentaire pure (air, terre, feu ou eau) peut
être convoqué depuis son plan d'origine afin de servir un magicien.

  Prend la forme d'énormes figures humanoïdes de terre ou de pierre.

}
\monsterdetail{Mineur}{\textbf{Sv} MP 8 B 9 PP 10 S 10 SSB 12 (8). (Convoqué par un bâton
magique)
}
\monsterdetail{Intermédiaire}{(2d8), \textbf{Sv} MP 6 B 7 PP 8 S 8 SSB 10 (12). (Convoqué par un
dispositif magique)
}
\monsterdetail{Supérieur}{(3d8), \textbf{Sv} MP 2 B 3 PP 4 S 3 SSB 6 (16). (Convoqué par un sort)
}
\monsterdetail{Taille}{Hauteur 2,40 m / 3,60 m / 4,80 m.
}
\monsterdetail{Bloqué par l’eau}{L'élémental ne peut pas traverser une étendue d'eau plus large que sa
propre hauteur.
}
\monsterdetail{Immunité aux dégâts normaux}{Ne peut être blessé que par des attaques magiques.
}
\monsterdetail{Néfaste aux créatures au sol}{Inflige 1d8 dégâts supplémentaires à ce type de créatures.
}
\monstercarac{ca}{2 [17] / 0 [19] / –2 [21]}
\monstercarac{hd}{8/12/16* (36/54/72 pv)}
\monstercarac{taco}{12 [+7] / 10 [+9] / 8 [+11]}
\monstercarac{moral}{10}
\monstercarac{alignement}{Neutre}
\monstercarac{xp}{1 200/1 900/2 300}
\monstercarac{nombre_donjon}{1}
\monstercarac{nombre_exterieur}{1}
\monstercarac{tresor}{None}
\monstercarac{save_mort_poison}{}
\monstercarac{save_baguettes}{}
\monstercarac{save_paralysie_petrification}{}
\monstercarac{save_souffles}{}
\monstercarac{save_sorts_sceptres_batons}{}
\monsterattack{1 × coup }{1d8 / 2d8 / 3d8)}
\end{monster}

\begin{monster}
\monstercarac{name}{Éléphant}
\monstercarac{description}{
  Ces animaux à défenses, massifs, habitent près des forêts subtropicales
et se rencontrent soit en individus isolés soit en troupeaux entiers.

}
\monsterdetail{Charge}{Au premier round de combat, lorsqu'il n'est pas encore engagé dans la
mêlée. Nécessite une trajectoire dégagée d'au moins 20 m entre lui et sa
cible. Les défenses infligent alors le double des dégâts.
}
\monsterdetail{Piétinement}{3-sur-4 chancesde piétiner à chaque round. Bonus de +4 pour toucher les
créatures de taille humaine ou plus petites.
}
\monsterdetail{Ivoire}{Chaque défense vaut 1d6 × 100 po.
}
\monstercarac{ca}{5 [14]}
\monstercarac{hd}{9 (40 pv)}
\monstercarac{taco}{12 [+7]}
\monstercarac{moral}{8}
\monstercarac{alignement}{Neutre}
\monstercarac{xp}{900}
\monstercarac{nombre_donjon}{0}
\monstercarac{nombre_exterieur}{1d20}
\monstercarac{tresor}{Défenses}
\monstercarac{save_mort_poison}{10}
\monstercarac{save_baguettes}{11}
\monstercarac{save_paralysie_petrification}{12}
\monstercarac{save_souffles}{13}
\monstercarac{save_sorts_sceptres_batons}{14}
\monsterattack{2 × défense }{2d4) ou 1 × piétinement}
\end{monster}

\begin{monster}
\monstercarac{img}{Nixie.jpg}
\monstercarac{name}{Esprit follet}
\monstercarac{description}{
  Humanoïde ailé de 30 cm de haut, apparenté aux lutins et aux elfes, d'un
naturel timide, mais curieux, et possède un étrange sens de l'humour.

}
\monsterdetail{Malédiction}{Cinq esprits follets peuvent maudire collectivement une cible (pas de
jet d'attaque ; jet de sauvegarde contre les sorts). L'effet est
déterminé par l'arbitre, mais il sera toujours utilisé pour produire un
effet comique (par exemple, faire trébucher la cible, lui allonger le
nez).
}
\monsterdetail{Farceur}{Même s'il est attaqué, l'esprit follet ne voit aucun intérêt à tuer et
se contente simplement de faire des blagues.
}
\monstercarac{ca}{5 [14]}
\monstercarac{hd}{½* (2 pv)}
\monstercarac{taco}{19 [0]}
\monstercarac{moral}{7}
\monstercarac{alignement}{Neutre}
\monstercarac{xp}{6}
\monstercarac{nombre_donjon}{3d6}
\monstercarac{nombre_exterieur}{5d8}
\monstercarac{tresor}{S}
\monstercarac{mvt}{18 m (6 m) / 54 m (18 m) vol}
\monstercarac{save_mort_poison}{12}
\monstercarac{save_baguettes}{13}
\monstercarac{save_paralysie_petrification}{13}
\monstercarac{save_souffles}{15}
\monstercarac{save_sorts_sceptres_batons}{15}
\monsterattack{1 × sort }{malédiction}
\end{monster}

\input{liste/Essaim_d’insectes.tex}
\input{liste/Fourmi_géante.tex}
\input{liste/Furet_géant.tex}
\begin{monster}
\monstercarac{img}{lion.jpg}
\monstercarac{name}{Lion}
\monstercarac{description}{
  Chasseur prudent qui évite généralement de combattre les humains, à
moins de mourir de faim ou d'être acculé. Les grands félins peuvent être
joueurs, mais sont prompts à la colère. Ils vivent en extérieur et
s'aventurent rarement sous terre.

  Il chasse en groupe appelé « bande » et vit dans des régions chaudes,
généralement dans la savane ou dans les brousses à proximité des
déserts.

}
\monsterdetail{Poursuite}{Poursuivent toujours leurs proies en fuite.
}
\monsterdetail{Proies favorites}{Développent souvent un goût pour un certain type de viande (ce qui peut
inclure les humains !) et chassent de préférence cette créature.
}
\monsterdetail{Curieux}{Peuvent suivre les PJ par curiosité.
}
\monsterdetail{Poursuite}{Poursuivent toujours leurs proies en fuite.
}
\monstercarac{ca}{6 [13]}
\monstercarac{hd}{5 (22 pv)}
\monstercarac{taco}{15 [+4]}
\monstercarac{moral}{9}
\monstercarac{alignement}{Neutre}
\monstercarac{xp}{175}
\monstercarac{nombre_donjon}{1d4}
\monstercarac{nombre_exterieur}{1d8}
\monstercarac{tresor}{U}
\monstercarac{mvt}{45 m (15 m)}
\monstercarac{save_mort_poison}{12}
\monstercarac{save_baguettes}{13}
\monstercarac{save_paralysie_petrification}{14}
\monstercarac{save_souffles}{15}
\monstercarac{save_sorts_sceptres_batons}{16}
\monsterattack{2 × griffes }{1d4 + 1}
\monsterattack{1 × morsure }{1d10}
\end{monster}

\begin{monster}
\monstercarac{name}{Lion des montagnes (Puma)}
\monstercarac{description}{
  Chasseur prudent qui évite généralement de combattre les humains, à
moins de mourir de faim ou d'être acculé. Les grands félins peuvent être
joueurs, mais sont prompts à la colère. Ils vivent en extérieur et
s'aventurent rarement sous terre.

  Il a une fourrure qui va du marron au jaune et préfère vivre dans les
montagnes, les déserts et les forêts. Il peut parfois s'aventurer dans
les donjons.

}
\monsterdetail{Poursuite}{Poursuivent toujours leurs proies en fuite.
}
\monsterdetail{Proies favorites}{Développent souvent un goût pour un certain type de viande (ce qui peut
inclure les humains !) et chassent de préférence cette créature.
}
\monsterdetail{Curieux}{Peuvent suivre les PJ par curiosité.
}
\monsterdetail{Poursuite}{Poursuivent toujours leurs proies en fuite.
}
\monstercarac{ca}{6 [13]}
\monstercarac{hd}{3+2 (15 pv)}
\monstercarac{taco}{16 [+3]}
\monstercarac{moral}{8}
\monstercarac{alignement}{Neutre}
\monstercarac{xp}{50}
\monstercarac{nombre_donjon}{1d4}
\monstercarac{nombre_exterieur}{1d4}
\monstercarac{tresor}{U}
\monstercarac{mvt}{45 m (15 m)}
\monstercarac{save_mort_poison}{12}
\monstercarac{save_baguettes}{13}
\monstercarac{save_paralysie_petrification}{14}
\monstercarac{save_souffles}{15}
\monstercarac{save_sorts_sceptres_batons}{16}
\monsterattack{2 × griffes }{1d3}
\monsterattack{1 × morsure }{1d6}
\end{monster}

\input{liste/Panthère.tex}
\begin{monster}
\monstercarac{name}{Tigre}
\monstercarac{description}{
  Chasseur prudent qui évite généralement de combattre les humains, à
moins de mourir de faim ou d'être acculé. Les grands félins peuvent être
joueurs, mais sont prompts à la colère. Ils vivent en extérieur et
s'aventurent rarement sous terre.

  Ce grand chasseur solitaire, dont la fourrure rayée lui sert de
camouflage, apprécie les régions boisées et froides.

}
\monsterdetail{Poursuite}{Poursuivent toujours leurs proies en fuite.
}
\monsterdetail{Proies favorites}{Développent souvent un goût pour un certain type de viande (ce qui peut
inclure les humains !) et chassent de préférence cette créature.
}
\monsterdetail{Curieux}{Peuvent suivre les PJ par curiosité.
}
\monsterdetail{Poursuite}{Poursuivent toujours leurs proies en fuite.
}
\monsterdetail{Surprise}{Dans les régions boisées, grâce à son camouflage, il attaque par
surprise sur un résultat de 1-4.
}
\monstercarac{ca}{6 [13]}
\monstercarac{hd}{6 (27 pv)}
\monstercarac{taco}{14 [+5]}
\monstercarac{moral}{9}
\monstercarac{alignement}{Neutre}
\monstercarac{xp}{275}
\monstercarac{nombre_donjon}{1}
\monstercarac{nombre_exterieur}{1d3}
\monstercarac{tresor}{U}
\monstercarac{mvt}{45 m (15 m)}
\monstercarac{save_mort_poison}{12}
\monstercarac{save_baguettes}{13}
\monstercarac{save_paralysie_petrification}{14}
\monstercarac{save_souffles}{15}
\monstercarac{save_sorts_sceptres_batons}{16}
\monsterattack{2 × griffes }{1d6}
\monsterattack{1 × morsure }{2d6}
\end{monster}

\input{liste/Tigre_à_dents_de_sabre.tex}
\begin{monster}
\monstercarac{img}{Gargoyle.png}
\monstercarac{name}{Gargouille}
\monstercarac{description}{
  Monstre magique a l'apparence d'une statue hideuse, cornue et ailée.
Semi-intelligent et doté d'une grande ruse.

}
\monsterdetail{Se confond avec la pierre}{Peut passer inaperçue ou être confondue avec une statue inanimée.
}
\monsterdetail{Gardienne}{Attaque presque toujours lorsqu'on l'approche.
}
\monsterdetail{Immunité aux dégâts normaux}{Ne peut être blessée que par des attaques magiques.
}
\monsterdetail{Immunité aux sorts}{Insensible aux sorts de sommeil ou de charme.
}
\monstercarac{ca}{5 [14]}
\monstercarac{hd}{4 (18 pv)}
\monstercarac{taco}{16 [+3]}
\monstercarac{moral}{11}
\monstercarac{alignement}{Chaotique}
\monstercarac{xp}{75}
\monstercarac{nombre_donjon}{1d6}
\monstercarac{nombre_exterieur}{2d4}
\monstercarac{tresor}{C}
\monstercarac{mvt}{27 m (9 m) / 45 m (15 m) vol}
\monstercarac{save_mort_poison}{8}
\monstercarac{save_baguettes}{9}
\monstercarac{save_paralysie_petrification}{10}
\monstercarac{save_souffles}{10}
\monstercarac{save_sorts_sceptres_batons}{12}
\monsterattack{2 × griffe }{1d3}
\monsterattack{1 × morsure }{1d6}
\monsterattack{1 × corne }{1d4}
\end{monster}

\input{liste/Gelée_ocre.tex}
\begin{monster}
\monstercarac{name}{Gnoll}
\monstercarac{description}{
  Hyène humanoïde et paresseuse, dotée d'une faible intelligence, qui vit
d'intimidation et de vol.~Les légendes disent que les gnolls ont été
créés par magie par un sorcier en croisant un gnome et un troll.

}
\monsterdetail{Chefs}{Les groupes de 20 gnolls et plus sont dirigés par un gnoll possédant 3DV
(16 pv).
}
\monstercarac{ca}{5 [14]}
\monstercarac{hd}{2 (9 pv)}
\monstercarac{taco}{18 [+1]}
\monstercarac{moral}{8}
\monstercarac{alignement}{Chaotique}
\monstercarac{xp}{20 (leader: 35)}
\monstercarac{nombre_donjon}{1d6}
\monstercarac{nombre_exterieur}{3d6}
\monstercarac{tresor}{D}
\monstercarac{mvt}{27 m (9 m)}
\monstercarac{save_mort_poison}{12}
\monstercarac{save_baguettes}{13}
\monstercarac{save_paralysie_petrification}{14}
\monstercarac{save_souffles}{15}
\monstercarac{save_sorts_sceptres_batons}{16}
\monsterattack{1 × arme }{2d4 }
\monsterattack{selon l’arme + 1)}{}
\end{monster}

\begin{monster}
\monstercarac{name}{Gnome}
\monstercarac{description}{
  Petit semi-humain au long nez et à la barbe fournie. Cousin plus petit
du nain, avec lequel il s'entend bien. Préfère vivre dans des terriers,
loin des régions montagneuses.

}
\monsterdetail{Armes}{Portent le plus souvent des marteaux de guerre et des arbalètes.
}
\monsterdetail{Infravision}{27 m.
}
\monsterdetail{Chef}{Par groupe de 20 gnomes, on trouve un chef avec 2 DV (11 pv).
}
\monsterdetail{Chef de clan et garde du corps}{Dans le repaire des gnomes vivent un chef de clan avec 4 DV (18 pv) et
1d6 gardes du corps avec 3 DV (1d4+9 pv). Le chef de clan reçoit un
bonus de +1 à ses jets de dégâts.
}
\monsterdetail{Haine des kobolds}{Les attaquent normalement à vue.
}
\monsterdetail{Mines et machines}{Sont férus de machinerie, de prospection, d'or et de gemmes. Font
parfois des choix malavisés pour acquérir des objets précieux. Partent
en guerre contre les
}
\monstercarac{ca}{5 [14]}
\monstercarac{hd}{1 (4 pv)}
\monstercarac{taco}{19 [0]}
\monstercarac{moral}{8 (10 si le chef ou chef de clan est présent)}
\monstercarac{alignement}{Loyal ou Neutre}
\monstercarac{xp}{10}
\monstercarac{nombre_donjon}{1d8}
\monstercarac{nombre_exterieur}{5d8}
\monstercarac{tresor}{C}
\monstercarac{save_mort_poison}{8}
\monstercarac{save_baguettes}{9}
\monstercarac{save_paralysie_petrification}{10}
\monstercarac{save_souffles}{13}
\monstercarac{save_sorts_sceptres_batons}{12}
\monsterattack{1 × arme }{1d6 ou selon l’arme)}
\end{monster}

\begin{monster}
\monstercarac{name}{Gobelin}
\monstercarac{description}{
  Petit humanoïde hideux, à la peau pâle couleur terre et aux yeux rouges
et brillants. Vit sous terre.

}
\monsterdetail{Infravision}{27 m.
}
\monsterdetail{Déteste le soleil}{Malus de --1 à ses attaques en plein jour.
}
\monsterdetail{Chevaucheur de loups}{20 \% des groupes de gobelins rencontrés ont des chevaucheurs de loups :
¼ du groupe chevauche des Loups géants.
}
\monsterdetail{Haine des nains}{Les attaquent à vue.
}
\monsterdetail{Roi gobelin et gardes du corps}{Un roi avec 3 DV (15 pv) et 2d6 gardes du corps avec 2 DV (2d6 pv)
vivent dans le repaire des gobelins. Ils ne subissent pas de malus aux
attaques en plein jour. Le roi gagne un bonus de +1 aux dégâts.
}
\monsterdetail{Butin}{Lorsqu'on les rencontre dans les contrées sauvages ou dans leur repaire,
les gobelins ne possèdent qu'un trésor de type C.
}
\monstercarac{ca}{6 [13]}
\monstercarac{hd}{1–1 (3 pv)}
\monstercarac{taco}{19 [0]}
\monstercarac{moral}{7 (9 avec un roi)}
\monstercarac{alignement}{Chaotique}
\monstercarac{xp}{5 (garde du corps : 20, roi : 35)}
\monstercarac{nombre_donjon}{2d4}
\monstercarac{nombre_exterieur}{6d10}
\monstercarac{tresor}{R (C)}
\monstercarac{save_mort_poison}{14}
\monstercarac{save_baguettes}{15}
\monstercarac{save_paralysie_petrification}{16}
\monstercarac{save_souffles}{17}
\monstercarac{save_sorts_sceptres_batons}{18}
\monsterattack{1 × arme }{1d6 ou selon l’arme)}
\end{monster}

\begin{monster}
\monstercarac{name}{Goblours}
\monstercarac{description}{
  Ce grand gobelin velu à la démarche disgracieuse préfère attaquer par
surprise.

}
\monsterdetail{Surprise}{Très discret, surprend sur un résultat de 1-3.
}
\monstercarac{ca}{5 [14]}
\monstercarac{hd}{3+1 (14 pv)}
\monstercarac{taco}{16[+3]}
\monstercarac{moral}{9}
\monstercarac{alignement}{Chaotique}
\monstercarac{xp}{50}
\monstercarac{nombre_donjon}{2d4}
\monstercarac{nombre_exterieur}{5d4}
\monstercarac{tresor}{B}
\monstercarac{mvt}{27 m (9 m)}
\monstercarac{save_mort_poison}{12}
\monstercarac{save_baguettes}{13}
\monstercarac{save_paralysie_petrification}{14}
\monstercarac{save_souffles}{15}
\monstercarac{save_sorts_sceptres_batons}{16}
\monsterattack{1 × arme }{2d4 }
\monsterattack{selon l’arme +1)}{}
\end{monster}

\input{liste/Golem_d’ambre.tex}
\begin{monster}
\monstercarac{name}{Golem de bois}
\monstercarac{description}{
  Être artificiel construit par de puissants magiciens ou clercs à partir
de divers matériaux.

  Humanoïde en bois, haut de 1 m, grossièrement construit.

}
\monsterdetail{Immunité}{Immunisé contre les gaz ; insensible aux sorts charme, paralysie et
sommeil.
}
\monsterdetail{Autres matériaux}{Des golems formés d'autres matériaux sont également possibles.
}
\monsterdetail{Construction}{Long processus très complexe et coûteux.
}
\monsterdetail{Immunité aux dégâts normaux}{Ne peut être blessé que par des attaques magiques.
}
\monsterdetail{Initiative}{Malus de --1 à cause de ses mouvements raides.
}
\monsterdetail{Inflammable}{Malus de --2 à ses jets de sauvegarde contre les attaques de feu ; subit
un point de dégâts supplémentaire par dé.
}
\monstercarac{ca}{7 [12]}
\monstercarac{hd}{2+2 (11 pv)}
\monstercarac{taco}{17 [+2]}
\monstercarac{moral}{12}
\monstercarac{alignement}{Neutre}
\monstercarac{xp}{25}
\monstercarac{nombre_donjon}{1}
\monstercarac{nombre_exterieur}{1}
\monstercarac{tresor}{Aucun}
\monstercarac{mvt}{36 m (12 m)}
\monstercarac{save_mort_poison}{12}
\monstercarac{save_baguettes}{13}
\monstercarac{save_paralysie_petrification}{14}
\monstercarac{save_souffles}{15}
\monstercarac{save_sorts_sceptres_batons}{16}
\monsterattack{1 × poing }{1d8}
\end{monster}

\begin{monster}
\monstercarac{name}{Golem de bronze}
\monstercarac{description}{
  Être artificiel construit par de puissants magiciens ou clercs à partir
de divers matériaux.

  Cette créature artificielle en bronze, possédant une grande chaleur
interne, ressemble à un géant de feu.

}
\monsterdetail{Immunité}{Immunisé contre les gaz ; insensible aux sorts charme, paralysie et
sommeil.
}
\monsterdetail{Autres matériaux}{Des golems formés d'autres matériaux sont également possibles.
}
\monsterdetail{Construction}{Long processus très complexe et coûteux.
}
\monsterdetail{Immunité aux dégâts normaux}{Ne peut être blessé que par des attaques magiques.
}
\monsterdetail{Sang incendiaire}{S'il est endommagé par une arme tranchante, le golem de bronze émet un
jet de feu liquide : l'attaquant subit 2d6 de dégâts (jet de sauvegarde
contre la mort pour l'éviter).
}
\monsterdetail{Immunité au feu}{Immunisé aux dégâts de feu.
}
\monstercarac{ca}{0 [19]}
\monstercarac{hd}{20** (90 pv)}
\monstercarac{taco}{6 [+13]}
\monstercarac{moral}{12}
\monstercarac{alignement}{Neutre}
\monstercarac{xp}{4}
\monstercarac{nombre_donjon}{1}
\monstercarac{nombre_exterieur}{1}
\monstercarac{tresor}{Aucun}
\monstercarac{save_mort_poison}{6}
\monstercarac{save_baguettes}{7}
\monstercarac{save_paralysie_petrification}{8}
\monstercarac{save_souffles}{8}
\monstercarac{save_sorts_sceptres_batons}{10}
\monsterattack{1 × poing }{3d10 + 1d10 chaleur)}
\end{monster}

\input{liste/Golem_d’os.tex}
\begin{monster}
\monstercarac{img}{Gorgon-5e.png}
\monstercarac{name}{Gorgone}
\monstercarac{description}{
  Monstre magique ressemble à un taureau couvert d'écailles. Il habite
dans les plaines ou les collines.

}
\monsterdetail{Charge}{Lorsqu'elle n'est pas encore engagée en mêlée. Nécessite une trajectoire
dégagée d'au moins 20 m entre elle et sa cible. Ses cornes infligent
alors le double de dégâts.
}
\monsterdetail{Souffle pétrifiant}{Nuage de 18 m de long et de 3 m de large. Tout créature prise dedans est
transformée en pierre (Jet de sauvegarde contre la pétrification pour
l'éviter). Pas affectée par son propre souffle.
}
\monstercarac{ca}{2 [17]}
\monstercarac{hd}{8* (36 pv)}
\monstercarac{taco}{12 [+7]}
\monstercarac{moral}{8}
\monstercarac{alignement}{Chaotique}
\monstercarac{xp}{1 200}
\monstercarac{nombre_donjon}{1d2}
\monstercarac{nombre_exterieur}{1d4}
\monstercarac{tresor}{E}
\monstercarac{mvt}{36 m (12 m)}
\monstercarac{save_mort_poison}{8}
\monstercarac{save_baguettes}{9}
\monstercarac{save_paralysie_petrification}{10}
\monstercarac{save_souffles}{10}
\monstercarac{save_sorts_sceptres_batons}{12}
\monsterattack{1 × corne }{2d6}
\monsterattack{1 × souffle }{pétrification}
\end{monster}

\begin{monster}
\monstercarac{img}{ape.jpg}
\monstercarac{name}{Gorille blanc}
\monstercarac{description}{
  Grands singes albinos herbivores vivant dans des cavernes et sortant la
nuit en quête de nourriture.

}
\monsterdetail{ Territorial}{Se comportent d'abord de manière menaçante, et si cela ne suffit pas, de
manière violente pour protéger leur repaire.
}
\monstercarac{ca}{6 [13]}
\monstercarac{hd}{4 (18 pv)}
\monstercarac{taco}{16 [+3]}
\monstercarac{moral}{7}
\monstercarac{alignement}{Neutre}
\monstercarac{xp}{75}
\monstercarac{nombre_donjon}{1d6}
\monstercarac{nombre_exterieur}{2d4}
\monstercarac{tresor}{Aucun}
\monstercarac{mvt}{36 m (12 m)}
\monstercarac{save_mort_poison}{12}
\monstercarac{save_baguettes}{13}
\monstercarac{save_paralysie_petrification}{14}
\monstercarac{save_souffles}{15}
\monstercarac{save_sorts_sceptres_batons}{16}
\monsterattack{2 × griffes }{1d4}
\monsterattack{1x lancer de rocher }{1d6}
\end{monster}

\begin{monster}
\monstercarac{img}{Ghoul_LengGhoul.png}
\monstercarac{name}{Goule}
\monstercarac{description}{
  Humain mort-vivant hideux, bestial et avide de chair vivante.

}
\monsterdetail{Paralysie}{Pour 2d4 tours (jet de sauvegarde contre la paralysie). Les elfes et les
créatures plus grandes que les ogres ne sont pas affectés. Après avoir
paralysé une cible, la goule attaque d'autres adversaires.
}
\monsterdetail{Mort-vivant}{Ne fait aucun bruit jusqu'à ce qu'elle attaque. Immunisée contre les
effets affectant les créatures vivantes (par exemple, le poison).
Immunisée contre les sorts affectant ou lisant l'esprit (par exemple,
charme, paralysie, sommeil).
}
\monstercarac{ca}{6 [13]}
\monstercarac{hd}{2* (9 pv)}
\monstercarac{taco}{18 [+1]}
\monstercarac{moral}{9}
\monstercarac{alignement}{Chaotique}
\monstercarac{xp}{25}
\monstercarac{nombre_donjon}{1d6}
\monstercarac{nombre_exterieur}{2d8}
\monstercarac{tresor}{B}
\monstercarac{mvt}{27 m (9 m)}
\monstercarac{save_mort_poison}{12}
\monstercarac{save_baguettes}{13}
\monstercarac{save_paralysie_petrification}{14}
\monstercarac{save_souffles}{15}
\monstercarac{save_sorts_sceptres_batons}{16}
\monsterattack{2 × griffes }{1d3 + paralysie}
\monsterattack{1 × morsure }{1d3 + paralysie}
\end{monster}

\begin{monster}
\monstercarac{img}{Griffon.png}
\monstercarac{name}{Griffon}
\monstercarac{description}{
  Grand rapace prédateur qui combine les caractéristiques d'un aigle
(tête, ailes, griffes avant) et celles d'un lion. Chasse les chevaux.

}
\monsterdetail{Attaque les chevaux}{Jusqu'à 36 m, à moins de réussir un test de morale.
}
\monsterdetail{Défense du nid}{Attaque si on l'approche.
}
\monsterdetail{Dressage}{Les petits capturés peuvent être dressés comme des montures loyales.
Mais l'entraînement ne fera pas disparaître leur nature féroce : ils
attaqueront toujours instinctivement les chevaux.
}
\monstercarac{ca}{5 [14]}
\monstercarac{hd}{7 (31 pv)}
\monstercarac{taco}{13 [+6]}
\monstercarac{moral}{8}
\monstercarac{alignement}{Neutre}
\monstercarac{xp}{450}
\monstercarac{nombre_donjon}{0}
\monstercarac{nombre_exterieur}{2d8}
\monstercarac{tresor}{E}
\monstercarac{mvt}{36 m (12 m) / 108 m (36 m) vol}
\monstercarac{save_mort_poison}{10}
\monstercarac{save_baguettes}{11}
\monstercarac{save_paralysie_petrification}{12}
\monstercarac{save_souffles}{13}
\monstercarac{save_sorts_sceptres_batons}{14}
\monsterattack{2 × griffes }{1d4}
\monsterattack{1 × morsure }{2d8}
\end{monster}

\input{liste/Géant_des_collines.tex}
\input{liste/Géant_de_feu.tex}
\input{liste/Géant_des_glaces.tex}
\input{liste/Géant_des_nuages.tex}
\input{liste/Géant_de_pierre.tex}
\begin{monster}
\monstercarac{name}{Géant des tempêtes}
\monstercarac{description}{
  Humanoïde, haut de 6,60 m, a la peau aux reflets bronze et les cheveux
d'une couleur criarde (rouges ou blonds).

}
\monsterdetail{Convocation de tempête}{Prend 1 tour.
}
\monsterdetail{Éclairs}{Dans une tempête, le géant peut, une fois tous les 5 rounds, lancer des
éclairs: de 18 m de long, et 1,50 m de large ; inflige des dégâts égaux
à ses points de vie actuels (jet de sauvegarde contre les sorts pour ne
subir que la moitié des dégâts) ; rebondit sur les surfaces dures se
trouvant sur son chemin.
}
\monsterdetail{Immunité à la foudre}{Immunisé contre la foudre. Se complaît dans les tempêtes.
}
\monsterdetail{Château}{Situé sur les sommets de montagne, au-dessus des bancs de nuages
\hspace{0pt}\hspace{0pt}ou dans les profondeurs sous-marines.
}
\monsterdetail{Gardiens}{Accompagné par 2d4 \href{Griffon.md}{griffons}. Sous l'eau : 3d6
\href{Crabe_géant.md}{crabes géants}.
}
\monstercarac{ca}{2 [17]}
\monstercarac{hd}{15 (67 pv)}
\monstercarac{taco}{9 [+10]}
\monstercarac{moral}{10}
\monstercarac{alignement}{Loyal}
\monstercarac{xp}{1 350}
\monstercarac{nombre_donjon}{1}
\monstercarac{nombre_exterieur}{1d3}
\monstercarac{tresor}{E + 5 000 po}
\monstercarac{save_mort_poison}{4}
\monstercarac{save_baguettes}{5}
\monstercarac{save_paralysie_petrification}{6}
\monstercarac{save_souffles}{5}
\monstercarac{save_sorts_sceptres_batons}{8}
\monsterattack{1 × arme }{8d6)}
\monsterattack{1 × éclair}{}
\end{monster}

\input{liste/Génie_des_eaux.tex}
\begin{monster}
\monstercarac{img}{Harpy.png}
\monstercarac{name}{Harpie}
\monstercarac{description}{
  Cette créature ressemble à un aigle géant doté du torse et de la tête
d'une hideuse sorcière. Son chant attire ses victimes vers leur perte.

}
\monsterdetail{Charme}{Quiconque entend le chant d'un groupe de harpies doit effectuer un jet
de sauvegarde contre les sorts. En cas d'échec, la victime est charmée
et doit : se déplacer en direction des harpies (en résistant à ceux qui
essaient de l'en empêcher) ; les défendre ; et obéir à leurs ordres (à
condition de les comprendre). En outre, elle ne peut ni lancer de sorts,
ni utiliser d'objets magiques, ni s'en prendre aux harpies. Un
personnage qui résiste au charme est immunisé pour le reste de la
rencontre. La mort des harpies rompt le charme.
}
\monsterdetail{Résistance à la magie}{Bonus de +2 à tous les jets de sauvegarde.
}
\monstercarac{ca}{7 [12]}
\monstercarac{hd}{3* (13 pv)}
\monstercarac{taco}{17 [+2]}
\monstercarac{moral}{7}
\monstercarac{alignement}{Chaotique}
\monstercarac{xp}{50}
\monstercarac{nombre_donjon}{1d6}
\monstercarac{nombre_exterieur}{2d4}
\monstercarac{tresor}{C}
\monstercarac{mvt}{18 m (6 m) / 45 m (15 m) vol}
\monstercarac{save_mort_poison}{12}
\monstercarac{save_baguettes}{13}
\monstercarac{save_paralysie_petrification}{14}
\monstercarac{save_souffles}{15}
\monstercarac{save_sorts_sceptres_batons}{16}
\monsterattack{2 × serre }{1d4}
\monsterattack{1 × arme }{1d6 (selon l’arme)}
\monsterattack{1 × chant }{charme}
\end{monster}

\begin{monster}
\monstercarac{name}{Hippogriffe}
\monstercarac{description}{
  Créature fantastique qui combine les caractéristiques d'un aigle géant
(tête et partie avant) et d'un cheval. Il construit son nid parmi les
rochers escarpés.

}
\monsterdetail{Haine des pégases}{L'hippogriffe les attaque généralement.
}
\monsterdetail{Monture}{Peut transporter un cavalier de taille humaine.
}
\monsterdetail{Dressage}{Il peut être dressé pour servir de monture.
}
\monstercarac{ca}{5 [14]}
\monstercarac{hd}{3+1 (14 pv)}
\monstercarac{taco}{16 [+3]}
\monstercarac{moral}{8}
\monstercarac{alignement}{Neutre}
\monstercarac{xp}{50}
\monstercarac{nombre_donjon}{0}
\monstercarac{nombre_exterieur}{2d8}
\monstercarac{tresor}{Aucun}
\monstercarac{mvt}{54 m (18 m) / 108 m (36 m) vol}
\monstercarac{save_mort_poison}{12}
\monstercarac{save_baguettes}{13}
\monstercarac{save_paralysie_petrification}{14}
\monstercarac{save_souffles}{15}
\monstercarac{save_sorts_sceptres_batons}{16}
\monsterattack{2 × griffes }{1d6}
\monsterattack{1 × morsure }{1d10}
\end{monster}

\begin{monster}
\monstercarac{name}{Hobgobelin}
\monstercarac{description}{
  Parent plus grand et plus méchant du gobelin. Habite sous terre, mais
cherche souvent ses proies à la surface.

}
\monsterdetail{Roi hobgobelin et gardes du corps}{Un roi de 5 DV (22 pv) et 1d4 gardes du corps de 4DV (3d6 pv) vivent
dans le repaire des hobgobelins. Le roi a un bonus de +2 aux jets de
dégâts. On peut parfois trouver un thoul parmi les gardes du corps du
roi.
}
\monstercarac{ca}{6 [13]}
\monstercarac{hd}{1+1 (5 pv)}
\monstercarac{taco}{18 [+1]}
\monstercarac{moral}{8 (10 avec un roi)}
\monstercarac{alignement}{Chaotique}
\monstercarac{xp}{15 (bodyguard: 75}
\monstercarac{nombre_donjon}{1d6}
\monstercarac{nombre_exterieur}{4d6}
\monstercarac{tresor}{D}
\monstercarac{mvt}{27 m (9 m)}
\monstercarac{save_mort_poison}{12}
\monstercarac{save_baguettes}{13}
\monstercarac{save_paralysie_petrification}{14}
\monstercarac{save_souffles}{15}
\monstercarac{save_sorts_sceptres_batons}{16}
\monsterattack{1 × arme }{1d8 }
\monsterattack{selon l’arme)}{}
\end{monster}

\input{liste/Homme-lézard.tex}
\begin{monster}
\monstercarac{name}{Homme-poisson}
\monstercarac{description}{
  Humanoïde aquatique, avec une queue de poisson à la place des jambes.
Vit dans les eaux côtières. Cultive les algues et pêche les poissons.

}
\monsterdetail{Armes}{Épieu, trident ou dague.
}
\monsterdetail{Chefs}{Par groupe de 10 hommes-poissons, il y a un chef avec 2 DV. Pour chaque
groupe de 50 hommes-poissons, il y a un chef avec 4 DV (jets de
sauvegarde : MP 10 B 11 PP 12 S 13 SSB 14 (4)).
}
\monsterdetail{Villages sous-marins}{Abritent 1d3 × 100 individus.
}
\monsterdetail{Gardiens}{Ils utilisent des créatures ou des monstres marins entraînés pour garder
leur maison.
}
\monstercarac{ca}{6 [13]}
\monstercarac{hd}{1 (4 pv)}
\monstercarac{taco}{19 [0]}
\monstercarac{moral}{8}
\monstercarac{alignement}{Neutre}
\monstercarac{xp}{10 (chef : 20/75)}
\monstercarac{nombre_donjon}{0}
\monstercarac{nombre_exterieur}{1d20}
\monstercarac{tresor}{A}
\monstercarac{mvt}{36 m (12 m)}
\monstercarac{save_mort_poison}{12}
\monstercarac{save_baguettes}{13}
\monstercarac{save_paralysie_petrification}{14}
\monstercarac{save_souffles}{15}
\monstercarac{save_sorts_sceptres_batons}{16}
\monsterattack{1 × arme }{1d6 }
\monsterattack{selon l’arme)}{}
\end{monster}

\begin{monster}
\monstercarac{name}{Humain normal}
\monstercarac{description}{
  Humains non aventuriers qui n'ont pas de classe de personnage. Ils
peuvent être artistes, mendiants, enfants, artisans, fermiers, pêcheurs,
femmes au foyer, érudits, esclaves.

}
\monsterdetail{Gain d’XP}{Après avoir gagné de l'expérience dans une aventure, il doit choisir une
classe de personnage.
}
\monstercarac{ca}{9 [10]}
\monstercarac{hd}{½ (2 pv)}
\monstercarac{taco}{20 [–1]}
\monstercarac{moral}{6}
\monstercarac{alignement}{Tous}
\monstercarac{xp}{5}
\monstercarac{nombre_donjon}{1d4}
\monstercarac{nombre_exterieur}{1d20}
\monstercarac{tresor}{U}
\monstercarac{save_mort_poison}{14}
\monstercarac{save_baguettes}{15}
\monstercarac{save_paralysie_petrification}{16}
\monstercarac{save_souffles}{17}
\monstercarac{save_sorts_sceptres_batons}{18}
\monsterattack{1 × arme }{1d6 ou selon l’arme)}
\end{monster}

\begin{monster}
\monstercarac{name}{Hydre}
\monstercarac{description}{
  Grande créature qui ressemble à un dragon aux multiples têtes
serpentines. L'hydre de mer (adaptée à l'eau, possédant des nageoires)
existe aussi.

}
\monsterdetail{Têtes}{1d8+4 têtes ; 1 DV par tête.
}
\monsterdetail{Perte d’une tête}{Pour chaque tranche de 8 dégâts infligés, l'hydre perd une tête (ne peut
plus attaquer).
}
\monsterdetail{Variantes}{On trouve parfois des hydres spéciales avec du venin, un souffle
enflammé, etc.
}
\monstercarac{ca}{5 [14]}
\monstercarac{hd}{5 à 12 (8 pv par DV)}
\monstercarac{taco}{selon DV (15 [+4] à 10 [+9])}
\monstercarac{moral}{9}
\monstercarac{alignement}{Neutre}
\monstercarac{xp}{175/275/450/50/900/900/1 100/1 100}
\monstercarac{nombre_donjon}{1}
\monstercarac{nombre_exterieur}{1}
\monstercarac{tresor}{B}
\monstercarac{mvt}{36 m (12 m)}
\monstercarac{save_mort_poison}{}
\monstercarac{save_baguettes}{}
\monstercarac{save_paralysie_petrification}{}
\monstercarac{save_souffles}{}
\monstercarac{save_sorts_sceptres_batons}{}
\monsterattack{5 à 12 × morsure }{1d10}
\end{monster}

\begin{monster}
\monstercarac{name}{Kobold}
\monstercarac{description}{
  Petit humanoïde à l'aspect canin et dont la peau écailleuse est couleur
rouille. Cette créature mauvaise et vicieuse vit sous terre.

}
\monsterdetail{Embuscade}{Préfèrent attaquer par surprise.
}
\monsterdetail{Infravision}{27 m.
}
\monsterdetail{Haine des gnomes}{Le kobold les attaque à vue.
}
\monsterdetail{Chef et gardes du corps}{Dans leur repaire vivent un chef de tribu avec 2 DV (9 pv) et 1d6 gardes
du corps avec 1+1 DV (6 pv).
}
\monsterdetail{Butin}{Dans les contrées sauvages ou dans leur repaire, seul le trésor de type
J est présent.
}
\monstercarac{ca}{7 [12]}
\monstercarac{hd}{½ (2 pv)}
\monstercarac{taco}{19 [0]}
\monstercarac{moral}{6 (8 avec un chef de tribu)}
\monstercarac{alignement}{Chaotique}
\monstercarac{xp}{5 (garde du corps : 15, chef de tribu : 20)}
\monstercarac{nombre_donjon}{4d4}
\monstercarac{nombre_exterieur}{6d10}
\monstercarac{tresor}{P (J)}
\monstercarac{mvt}{18 m (6 m)}
\monstercarac{save_mort_poison}{14}
\monstercarac{save_baguettes}{15}
\monstercarac{save_paralysie_petrification}{16}
\monstercarac{save_souffles}{17}
\monstercarac{save_sorts_sceptres_batons}{18}
\monsterattack{1 × arme }{1d4 }
\monsterattack{selon l’arme –1)}{}
\end{monster}

\begin{monster}
\monstercarac{name}{Licorne}
\monstercarac{description}{
  Créature fantastique ressemblant à un cheval élégant, avec une longue
corne. Les licornes sont timides, mais fières et volontaires.

}
\monsterdetail{Empathie avec les jeunes filles}{Une jeune fille au cœur pur peut communiquer avec une licorne et la
monter.
}
\monsterdetail{Téléportation}{Une fois par jour, peut se téléporter avec sa cavalière jusqu'à une
distance de 108 m.
}
\monstercarac{ca}{2 [17]}
\monstercarac{hd}{4* (18 pv)}
\monstercarac{taco}{16 [+3]}
\monstercarac{moral}{7}
\monstercarac{alignement}{Loyal}
\monstercarac{xp}{125}
\monstercarac{nombre_donjon}{1d6}
\monstercarac{nombre_exterieur}{1d8}
\monstercarac{tresor}{Aucun}
\monstercarac{mvt}{72 m (24 m)}
\monstercarac{save_mort_poison}{8}
\monstercarac{save_baguettes}{9}
\monstercarac{save_paralysie_petrification}{10}
\monstercarac{save_souffles}{10}
\monstercarac{save_sorts_sceptres_batons}{12}
\monsterattack{2 × sabot }{1d8}
\monsterattack{1 × corne }{1d8}
\end{monster}

\begin{monster}
\monstercarac{name}{Limon vert}
\monstercarac{description}{
  Gelée verdâtre, dégoulinante, qui grimpe aux murs et au plafond.

}
\monsterdetail{Surprise}{Se laisse tomber sur les personnages surpris.
}
\monsterdetail{Acide}{Lorsqu'en contact avec sa victime, se colle à elle et exsude un acide
détruisant le bois et le métal (y compris les armures) en 6 rounds, mais
n'affectant pas la pierre.
}
\monsterdetail{Dévorer la chair}{Après 6 rounds en contact avec la chair de sa victime, cette dernière
est transformée en limon vert au bout de 1d4 rounds supplémentaires.
}
\monsterdetail{Retirer}{Une fois collé à sa victime, il ne se détache que s'il est détruit par
le feu. Les dégâts sont en ce cas partagés équitablement entre le limon
et la victime.
}
\monsterdetail{Immunité}{Seules les attaques de froid et de feu peuvent blesser le limon.
}
\monstercarac{ca}{aucun jet pour toucher nécessaire}
\monstercarac{hd}{2* (9 pv)}
\monstercarac{taco}{18 [+1]}
\monstercarac{moral}{12}
\monstercarac{alignement}{Neutre}
\monstercarac{xp}{25}
\monstercarac{nombre_donjon}{1}
\monstercarac{nombre_exterieur}{0}
\monstercarac{tresor}{Aucun}
\monstercarac{mvt}{1 m (30 cm)}
\monstercarac{save_mort_poison}{12}
\monstercarac{save_baguettes}{13}
\monstercarac{save_paralysie_petrification}{14}
\monstercarac{save_souffles}{15}
\monstercarac{save_sorts_sceptres_batons}{16}
\monsterattack{1 × toucher }{dévore la chair}
\end{monster}

\begin{monster}
\monstercarac{name}{Loup normal}
\monstercarac{description}{
  Cousin carnivore du chien ; chasse en meute.

  Trouvé le plus souvent dans les contrées sauvages, mais niche parfois
dans des grottes.

}
\monsterdetail{Dressage}{À la discrétion de l'arbitre, les louveteaux capturés peuvent être
dressés comme des chiens. Les loups adultes ne se laissent pas dresser
facilement.
}
\monsterdetail{L’union fait la force}{Les meutes de 4 loups ou plus ont un moral de 8. Ce bonus est perdu si
la meute est réduite de moitié.
}
\monstercarac{ca}{7 [12]}
\monstercarac{hd}{2+2 (11 pv)}
\monstercarac{taco}{17 [+2]}
\monstercarac{moral}{6 (8 pour une meute importante)}
\monstercarac{alignement}{Neutre}
\monstercarac{xp}{25}
\monstercarac{nombre_donjon}{2d6}
\monstercarac{nombre_exterieur}{3d6}
\monstercarac{tresor}{Aucun}
\monstercarac{save_mort_poison}{12}
\monstercarac{save_baguettes}{13}
\monstercarac{save_paralysie_petrification}{14}
\monstercarac{save_souffles}{15}
\monstercarac{save_sorts_sceptres_batons}{16}
\monsterattack{1 × morsure }{1d6)}
\end{monster}

\input{liste/Loup_géant.tex}
\begin{monster}
\monstercarac{name}{Loup-garou}
\monstercarac{description}{
  Changeurs de forme capables d'alterner entre leur apparence humaine et
animale.

  Créature semi-intelligente qui chasse en meute.

}
\monsterdetail{Forme humaine}{Possède des caractéristiques physiques qui rappellent sa nature animale.
}
\monsterdetail{Immunité aux dégâts normaux}{Sous sa forme animale, ne peut être blessé que par les armes en argent
ou la magie.
}
\monsterdetail{Langues}{Sous sa forme humaine, il parle normalement, mais sous forme animale, il
ne peut communiquer qu'avec les animaux du même type.
}
\monsterdetail{Armure}{Les armures gênent la transformation d'un lycanthrope ; il n'en porte
jamais.
}
\monsterdetail{Appel d’animaux}{Peut convoquer 1 ou 2 animaux de son espèce présents dans les alentours
(les rats-garous appellent des rats géants --- voir p.~XX). Ils arrivent
en 1d4 rounds.
}
\monsterdetail{Herbe au loup}{Un lycanthrope touché par de l'herbe au loup doit effectuer un jet de
sauvegarde contre le poison ou s'enfuir, terrorisé.
}
\monsterdetail{Retour}{À sa mort, le lycanthrope reprend sa forme humaine.
}
\monsterdetail{Odeur}{Les chevaux et certains autres animaux sentent les lycanthropes et
peuvent prendre peur.
}
\monsterdetail{Infection}{Un personnage qui perd plus de la moitié de ses points de vie à cause
des attaques naturelles d'un lycanthrope (morsure, griffes\ldots)
contracte la lycanthropie. Les humains deviennent des créatures-garous
du même type et passent sous le contrôle de l'arbitre ; les non-humains
meurent. La maladie se déclare après 2d12 jours, des signes d'infection
étant déjà visibles à la moitié de ce temps.
}
\monsterdetail{Meneur}{Les groupes de cinq individus et plus sont dirigés par un loup-garou
avec 5 DV (30 pv). Le chef reçoit un bonus de +2 à ses jets de dégâts.
}
\monstercarac{ca}{5 [14] (9 [10] sous forme humaine)}
\monstercarac{hd}{4* (18 pv)}
\monstercarac{taco}{16 [+3]}
\monstercarac{moral}{8}
\monstercarac{alignement}{Chaotique}
\monstercarac{xp}{125 (chef : 300)}
\monstercarac{nombre_donjon}{1d6}
\monstercarac{nombre_exterieur}{2d6}
\monstercarac{tresor}{C}
\monstercarac{mvt}{54 m (18 m)}
\monstercarac{save_mort_poison}{10}
\monstercarac{save_baguettes}{11}
\monstercarac{save_paralysie_petrification}{12}
\monstercarac{save_souffles}{13}
\monstercarac{save_sorts_sceptres_batons}{14}
\monsterattack{1 × morsure }{2d4}
\end{monster}

\begin{monster}
\monstercarac{img}{Werebear.png}
\monstercarac{name}{Ours-garou}
\monstercarac{description}{
  Changeurs de forme capables d'alterner entre leur apparence humaine et
animale.

  Très intelligent, même sous sa forme d'ours. Il vit seul ou parmi les
ours normaux.

}
\monsterdetail{Forme humaine}{Possède des caractéristiques physiques qui rappellent sa nature animale.
}
\monsterdetail{Immunité aux dégâts normaux}{Sous sa forme animale, ne peut être blessé que par les armes en argent
ou la magie.
}
\monsterdetail{Langues}{Sous sa forme humaine, il parle normalement, mais sous forme animale, il
ne peut communiquer qu'avec les animaux du même type.
}
\monsterdetail{Armure}{Les armures gênent la transformation d'un lycanthrope ; il n'en porte
jamais.
}
\monsterdetail{Appel d’animaux}{Peut convoquer 1 ou 2 animaux de son espèce présents dans les alentours
(les rats-garous appellent des rats géants --- voir p.~XX). Ils arrivent
en 1d4 rounds.
}
\monsterdetail{Herbe au loup}{Un lycanthrope touché par de l'herbe au loup doit effectuer un jet de
sauvegarde contre le poison ou s'enfuir, terrorisé.
}
\monsterdetail{Retour}{À sa mort, le lycanthrope reprend sa forme humaine.
}
\monsterdetail{Odeur}{Les chevaux et certains autres animaux sentent les lycanthropes et
peuvent prendre peur.
}
\monsterdetail{Infection}{Un personnage qui perd plus de la moitié de ses points de vie à cause
des attaques naturelles d'un lycanthrope (morsure, griffes\ldots)
contracte la lycanthropie. Les humains deviennent des créatures-garous
du même type et passent sous le contrôle de l'arbitre ; les non-humains
meurent. La maladie se déclare après 2d12 jours, des signes d'infection
étant déjà visibles à la moitié de ce temps.
}
\monsterdetail{Aimable}{Peut être amical, si on l'aborde de manière pacifique.
}
\monsterdetail{Étreinte d’ours}{Si une victime est touchée par les deux pattes au cours du même round,
l'ours-garou l'étreint et lui inflige 2d8 dégâts automatiques
supplémentaires.
}
\monstercarac{ca}{2 [17]}
\monstercarac{hd}{6* (27 pv)}
\monstercarac{taco}{14 [+5]}
\monstercarac{moral}{10}
\monstercarac{alignement}{Neutre}
\monstercarac{xp}{500}
\monstercarac{nombre_donjon}{1d4}
\monstercarac{nombre_exterieur}{1d4}
\monstercarac{tresor}{C}
\monstercarac{mvt}{36 m (12 m)}
\monstercarac{save_mort_poison}{10}
\monstercarac{save_baguettes}{11}
\monstercarac{save_paralysie_petrification}{12}
\monstercarac{save_souffles}{13}
\monstercarac{save_sorts_sceptres_batons}{14}
\monsterattack{2 × griffes }{2d4}
\monsterattack{1 × morsure }{2d8}
\end{monster}

\input{liste/Pourceau_maléfique.tex}
\begin{monster}
\monstercarac{name}{Rat-garou}
\monstercarac{description}{
  Changeurs de forme capables d'alterner entre leur apparence humaine et
animale.

  Rat humanoïde intelligent capable de se transformer en humain.

}
\monsterdetail{Forme humaine}{Possède des caractéristiques physiques qui rappellent sa nature animale.
}
\monsterdetail{Immunité aux dégâts normaux}{Sous sa forme animale, ne peut être blessé que par les armes en argent
ou la magie.
}
\monsterdetail{Langues}{Sous sa forme humaine, il parle normalement, mais sous forme animale, il
ne peut communiquer qu'avec les animaux du même type.
}
\monsterdetail{Armure}{Les armures gênent la transformation d'un lycanthrope ; il n'en porte
jamais.
}
\monsterdetail{Appel d’animaux}{Peut convoquer 1 ou 2 animaux de son espèce présents dans les alentours
(les rats-garous appellent des rats géants --- voir p.~XX). Ils arrivent
en 1d4 rounds.
}
\monsterdetail{Herbe au loup}{Un lycanthrope touché par de l'herbe au loup doit effectuer un jet de
sauvegarde contre le poison ou s'enfuir, terrorisé.
}
\monsterdetail{Retour}{À sa mort, le lycanthrope reprend sa forme humaine.
}
\monsterdetail{Odeur}{Les chevaux et certains autres animaux sentent les lycanthropes et
peuvent prendre peur.
}
\monsterdetail{Infection}{Un personnage qui perd plus de la moitié de ses points de vie à cause
des attaques naturelles d'un lycanthrope (morsure, griffes\ldots)
contracte la lycanthropie. Les humains deviennent des créatures-garous
du même type et passent sous le contrôle de l'arbitre ; les non-humains
meurent. La maladie se déclare après 2d12 jours, des signes d'infection
étant déjà visibles à la moitié de ce temps.
}
\monsterdetail{Surprise}{Sur un 1-à-4 ; tendent des embuscades.
}
\monsterdetail{Langues}{Parle le commun, quelle que soit sa forme.
}
\monsterdetail{Armes}{Peut manier des armes également sous sa forme animale.
}
\monstercarac{ca}{}
\monstercarac{hd}{}
\monstercarac{taco}{}
\monstercarac{moral}{8}
\monstercarac{alignement}{Chaotique}
\monstercarac{xp}{50}
\monstercarac{nombre_donjon}{}
\monstercarac{nombre_exterieur}{}
\monstercarac{tresor}{}
\monstercarac{mvt}{}
\monstercarac{save_mort_poison}{}
\monstercarac{save_baguettes}{}
\monstercarac{save_paralysie_petrification}{}
\monstercarac{save_souffles}{}
\monstercarac{save_sorts_sceptres_batons}{}
\monsterattack{1 × morsure }{1d4}
\monsterattack{1 × arme }{1d6 }
\monsterattack{selon l’arme)}{}
\end{monster}

\begin{monster}
\monstercarac{img}{Wereboar-5e.png}
\monstercarac{name}{Sanglier-garou}
\monstercarac{description}{
  Changeurs de forme capables d'alterner entre leur apparence humaine et
animale.

  Semi-intelligent et irascible, on le confond souvent sous sa forme
humaine avec un berserker.

}
\monsterdetail{Forme humaine}{Possède des caractéristiques physiques qui rappellent sa nature animale.
}
\monsterdetail{Immunité aux dégâts normaux}{Sous sa forme animale, ne peut être blessé que par les armes en argent
ou la magie.
}
\monsterdetail{Langues}{Sous sa forme humaine, il parle normalement, mais sous forme animale, il
ne peut communiquer qu'avec les animaux du même type.
}
\monsterdetail{Armure}{Les armures gênent la transformation d'un lycanthrope ; il n'en porte
jamais.
}
\monsterdetail{Appel d’animaux}{Peut convoquer 1 ou 2 animaux de son espèce présents dans les alentours
(les rats-garous appellent des rats géants --- voir p.~XX). Ils arrivent
en 1d4 rounds.
}
\monsterdetail{Herbe au loup}{Un lycanthrope touché par de l'herbe au loup doit effectuer un jet de
sauvegarde contre le poison ou s'enfuir, terrorisé.
}
\monsterdetail{Retour}{À sa mort, le lycanthrope reprend sa forme humaine.
}
\monsterdetail{Odeur}{Les chevaux et certains autres animaux sentent les lycanthropes et
peuvent prendre peur.
}
\monsterdetail{Infection}{Un personnage qui perd plus de la moitié de ses points de vie à cause
des attaques naturelles d'un lycanthrope (morsure, griffes\ldots)
contracte la lycanthropie. Les humains deviennent des créatures-garous
du même type et passent sous le contrôle de l'arbitre ; les non-humains
meurent. La maladie se déclare après 2d12 jours, des signes d'infection
étant déjà visibles à la moitié de ce temps.
}
\monsterdetail{Rage de bataille}{Sous forme humaine, peut entrer dans une rage folle : bonus de +2 pour
toucher, combat à mort. La rage le pousse parfois à attaquer ses alliés.
}
\monstercarac{ca}{4 [15]}
\monstercarac{hd}{4+1* (19 pv)}
\monstercarac{taco}{15 [+4]}
\monstercarac{moral}{9}
\monstercarac{alignement}{Neutre}
\monstercarac{xp}{200}
\monstercarac{nombre_donjon}{1d4}
\monstercarac{nombre_exterieur}{2d4}
\monstercarac{tresor}{C}
\monstercarac{mvt}{45 m (15 m)}
\monstercarac{save_mort_poison}{10}
\monstercarac{save_baguettes}{11}
\monstercarac{save_paralysie_petrification}{12}
\monstercarac{save_souffles}{13}
\monstercarac{save_sorts_sceptres_batons}{14}
\monsterattack{1 × défense / morsure }{2d6}
\end{monster}

\begin{monster}
\monstercarac{img}{Weretiger-5e.png}
\monstercarac{name}{Tigre-garou}
\monstercarac{description}{
  Changeurs de forme capables d'alterner entre leur apparence humaine et
animale.

  Montre un comportement félin : curieux, mais dangereux quand il est
acculé. Ce sont des nageurs et des chasseurs talentueux.

}
\monsterdetail{Forme humaine}{Possède des caractéristiques physiques qui rappellent sa nature animale.
}
\monsterdetail{Immunité aux dégâts normaux}{Sous sa forme animale, ne peut être blessé que par les armes en argent
ou la magie.
}
\monsterdetail{Langues}{Sous sa forme humaine, il parle normalement, mais sous forme animale, il
ne peut communiquer qu'avec les animaux du même type.
}
\monsterdetail{Armure}{Les armures gênent la transformation d'un lycanthrope ; il n'en porte
jamais.
}
\monsterdetail{Appel d’animaux}{Peut convoquer 1 ou 2 animaux de son espèce présents dans les alentours
(les rats-garous appellent des rats géants --- voir p.~XX). Ils arrivent
en 1d4 rounds.
}
\monsterdetail{Herbe au loup}{Un lycanthrope touché par de l'herbe au loup doit effectuer un jet de
sauvegarde contre le poison ou s'enfuir, terrorisé.
}
\monsterdetail{Retour}{À sa mort, le lycanthrope reprend sa forme humaine.
}
\monsterdetail{Odeur}{Les chevaux et certains autres animaux sentent les lycanthropes et
peuvent prendre peur.
}
\monsterdetail{Infection}{Un personnage qui perd plus de la moitié de ses points de vie à cause
des attaques naturelles d'un lycanthrope (morsure, griffes\ldots)
contracte la lycanthropie. Les humains deviennent des créatures-garous
du même type et passent sous le contrôle de l'arbitre ; les non-humains
meurent. La maladie se déclare après 2d12 jours, des signes d'infection
étant déjà visibles à la moitié de ce temps.
}
\monsterdetail{Surprise}{Sur un 1-à-4, en raison de leur discrétion.
}
\monstercarac{ca}{3 [16]}
\monstercarac{hd}{5* (22 pv)}
\monstercarac{taco}{15 [+4]}
\monstercarac{moral}{9}
\monstercarac{alignement}{Neutre}
\monstercarac{xp}{300}
\monstercarac{nombre_donjon}{1d4}
\monstercarac{nombre_exterieur}{1d4}
\monstercarac{tresor}{C}
\monstercarac{mvt}{45 m (15 m)}
\monstercarac{save_mort_poison}{10}
\monstercarac{save_baguettes}{11}
\monstercarac{save_paralysie_petrification}{12}
\monstercarac{save_souffles}{13}
\monstercarac{save_sorts_sceptres_batons}{14}
\monsterattack{2 × griffes }{1d6}
\monsterattack{1 × morsure }{2d6}
\end{monster}

\begin{monster}
\monstercarac{name}{Caméléon cornu}
\monstercarac{description}{
  Lézard long de 2,10 m, dont les écailles aux couleurs changeantes lui
servent de camouflage.

}
\monsterdetail{Surprise}{Sur un 1-à-5, en raison du camouflage.
}
\monsterdetail{Langue collante}{Peut attaquer des cibles jusqu'à 1,50 m de distance. Si l'attaque
réussit, la victime est traînée vers la gueule et mordue (2d4 points de
dégâts).
}
\monsterdetail{Queue}{N'inflige aucun dégât, mais renverse l'adversaire, qui ne peut pas
attaquer ce round.
}
\monstercarac{ca}{2 [17]}
\monstercarac{hd}{5* (22 pv)}
\monstercarac{taco}{15 [+4]}
\monstercarac{moral}{7}
\monstercarac{alignement}{Neutre}
\monstercarac{xp}{300}
\monstercarac{nombre_donjon}{1d3}
\monstercarac{nombre_exterieur}{1d6}
\monstercarac{tresor}{U}
\monstercarac{mvt}{36 m (12 m)}
\monstercarac{save_mort_poison}{12}
\monstercarac{save_baguettes}{13}
\monstercarac{save_paralysie_petrification}{14}
\monstercarac{save_souffles}{15}
\monstercarac{save_sorts_sceptres_batons}{16}
\monsterattack{1 × langue / morsure }{2d4}
\monsterattack{1 × corne }{1d6}
\monsterattack{1 × queue }{renversé}
\end{monster}

\begin{monster}
\monstercarac{name}{Draco}
\monstercarac{description}{
  Lézard carnivore, de 1,80 m de long, qui possède des membranes cutanées
entre les jambes qui lui permettent de planer. Habite généralement en
surface, mais s'abrite parfois dans des grottes. Attaque parfois les
humains.

}
\monstercarac{ca}{5 [14]}
\monstercarac{hd}{4+2 (20 pv)}
\monstercarac{taco}{15 [+4]}
\monstercarac{moral}{7}
\monstercarac{alignement}{Neutre}
\monstercarac{xp}{125}
\monstercarac{nombre_donjon}{1d4}
\monstercarac{nombre_exterieur}{1d8}
\monstercarac{tresor}{U}
\monstercarac{save_mort_poison}{12}
\monstercarac{save_baguettes}{13}
\monstercarac{save_paralysie_petrification}{14}
\monstercarac{save_souffles}{15}
\monstercarac{save_sorts_sceptres_batons}{16}
\monsterattack{1 × morsure }{1d10)}
\end{monster}

\begin{monster}
\monstercarac{name}{Gecko}
\monstercarac{description}{
  Lézard nocturne carnivore, de 1,50 m de long, et recouvert d'écailles
bleu clair avec des taches orange.

}
\monsterdetail{Collant}{Grimpe sur les murs, les arbres, etc. et se laisse tomber sur ses
victimes.
}
\monstercarac{ca}{5 [14]}
\monstercarac{hd}{3+1 (14 pv)}
\monstercarac{taco}{16 [+3]}
\monstercarac{moral}{7}
\monstercarac{alignement}{Neutre}
\monstercarac{xp}{50}
\monstercarac{nombre_donjon}{1d6}
\monstercarac{nombre_exterieur}{1d10}
\monstercarac{tresor}{U}
\monstercarac{mvt}{36 m (12 m)}
\monstercarac{save_mort_poison}{12}
\monstercarac{save_baguettes}{13}
\monstercarac{save_paralysie_petrification}{14}
\monstercarac{save_souffles}{15}
\monstercarac{save_sorts_sceptres_batons}{16}
\monsterattack{1 × morsure }{1d8}
\end{monster}

\begin{monster}
\monstercarac{name}{Tuatara}
\monstercarac{description}{
  Lézard carnivore de 2,40 m de long, semblable à un iguane, avec des
écailles de couleur olive et une crête de pointes blanches le long du
dos. Connu pour attaquer les humains.

}
\monsterdetail{Infravision}{27 m. Grâce à ses membranes oculaires rétractables.
}
\monstercarac{ca}{4 [15]}
\monstercarac{hd}{6 (27 pv)}
\monstercarac{taco}{14 [+5]}
\monstercarac{moral}{6}
\monstercarac{alignement}{Neutre}
\monstercarac{xp}{275}
\monstercarac{nombre_donjon}{1d2}
\monstercarac{nombre_exterieur}{1d4}
\monstercarac{tresor}{V}
\monstercarac{mvt}{27 m (9 m)}
\monstercarac{save_mort_poison}{}
\monstercarac{save_baguettes}{}
\monstercarac{save_paralysie_petrification}{}
\monstercarac{save_souffles}{}
\monstercarac{save_sorts_sceptres_batons}{}
\monsterattack{2 × griffes }{1d4}
\monsterattack{1 × morsure }{2d6}
\end{monster}

\begin{monster}
\monstercarac{name}{Manticore}
\monstercarac{description}{
  Monstruosité avec le visage d'un homme, le corps d'un lion, des ailes de
chauve-souris et une queue hérissée de pointes. Adore dévorer les
humains et habite dans les régions sauvages et montagneuses.

}
\monsterdetail{Queue hérissée de pointes}{Peut les projeter jusqu'à 54 m. 24 au total. En repousse 2 chaque jour.
}
\monsterdetail{Pistage et embuscade}{Elle suit les humains et les attaque avec les pointes de sa queue
lorsqu'ils s'arrêtent pour se reposer.
}
\monstercarac{ca}{4 [15]}
\monstercarac{hd}{6+1 (28 pv)}
\monstercarac{taco}{13 [+6]}
\monstercarac{moral}{9}
\monstercarac{alignement}{Chaotique}
\monstercarac{xp}{350}
\monstercarac{nombre_donjon}{1d2}
\monstercarac{nombre_exterieur}{1d4}
\monstercarac{tresor}{D}
\monstercarac{mvt}{36 m (12 m) / 54 m (18 m) vol}
\monstercarac{save_mort_poison}{10}
\monstercarac{save_baguettes}{11}
\monstercarac{save_paralysie_petrification}{12}
\monstercarac{save_souffles}{13}
\monstercarac{save_sorts_sceptres_batons}{14}
\monsterattack{2 × griffe }{1d4}
\monsterattack{1 × morsure }{2d4}
\monsterattack{6 × pointe de queue }{1d6}
\end{monster}

\begin{monster}
\monstercarac{name}{Marchand}
\monstercarac{description}{
  Commerçants organisés voyageant entre différents villages dans des
caravanes bien armées, achetant et vendant des biens commerciaux (par
exemple, de l'or, des bijoux, de la soie, des épices, du vin, etc.).

}
\monsterdetail{Armes}{Épée et dague.
}
\monsterdetail{Monté}{À cheval, à dos de chameau ou de mule (selon le terrain).
}
\monsterdetail{Chariots}{2 par marchand. Tirés par des chevaux, des mules ou des chameaux (selon
le terrain).
}
\monsterdetail{Gardes de caravanes}{Pour chaque marchand, 4 gardes sont présents (guerriers de 1er niveau).
CA 4 {[}15{]}, avec des arbalètes, des épées, des poignards.
}
\monsterdetail{Lieutenants}{2 lieutenants sont présents par groupe de 5 marchands (guerriers de 2e
ou 3e niveau). CA 4 {[}15{]}. Équipés comme les gardes de caravanes.
}
\monsterdetail{Capitaine}{Les gardes de caravanes sont dirigés par un guerrier de 5e niveau. CA 4
{[}15{]}. Équipé comme les gardes de caravanes.
}
\monsterdetail{Animaux porteurs}{1d12 chevaux, mules ou chameaux supplémentaires.
}
\monsterdetail{Trésor}{Doit être réduit s'il y a moins de 10 marchands dans le groupe.
}
\monstercarac{ca}{5 [14]}
\monstercarac{hd}{1 (4 pv)}
\monstercarac{taco}{19 [0]}
\monstercarac{moral}{variable}
\monstercarac{alignement}{Neutre}
\monstercarac{xp}{10}
\monstercarac{nombre_donjon}{0}
\monstercarac{nombre_exterieur}{1d20}
\monstercarac{tresor}{A}
\monstercarac{mvt}{27 m (9 m)}
\monstercarac{save_mort_poison}{12}
\monstercarac{save_baguettes}{13}
\monstercarac{save_paralysie_petrification}{14}
\monstercarac{save_souffles}{15}
\monstercarac{save_sorts_sceptres_batons}{16}
\monsterattack{1 × arme }{1d6 }
\monsterattack{selon l’arme)}{}
\end{monster}

\begin{monster}
\monstercarac{name}{Mastodonte}
\monstercarac{description}{
  Éléphant à long poil avec de grandes défenses. Vit dans les régions
glacées ou celles du Monde perdu.

}
\monsterdetail{Charge}{Au premier round de combat, lorsqu'il n'est pas encore engagé dans la
mêlée. Nécessite une trajectoire dégagée d'au moins 20 m. Les défenses
infligent alors double dégâts.
}
\monsterdetail{Piétinement}{3-sur-4 chances de piétiner à chaque round. Bonus de +4 pour toucher les
créatures de taille humaine ou plus petites.
}
\monsterdetail{Ivoire}{2d4 × 100 po par défense.
}
\monstercarac{ca}{3 [16]}
\monstercarac{hd}{15 (67 pv)}
\monstercarac{taco}{9 [+10]}
\monstercarac{moral}{8}
\monstercarac{alignement}{Neutre}
\monstercarac{xp}{1 350}
\monstercarac{nombre_donjon}{0}
\monstercarac{nombre_exterieur}{2d8}
\monstercarac{tresor}{Défenses}
\monstercarac{mvt}{36 m (12 m)}
\monstercarac{save_mort_poison}{8}
\monstercarac{save_baguettes}{9}
\monstercarac{save_paralysie_petrification}{10}
\monstercarac{save_souffles}{10}
\monstercarac{save_sorts_sceptres_batons}{12}
\monsterattack{2 × défense }{2d6}
\monsterattack{1 × piétinement }{4d8}
\end{monster}

\input{liste/Mille-pattes_géant.tex}
\begin{monster}
\monstercarac{img}{Minotaur.jpg}
\monstercarac{name}{Minotaure}
\monstercarac{description}{
  Ce grand homme brutal à tête de taureau a soif de chair humaine et
habite dans des dédales et des labyrinthes.

}
\monsterdetail{Armes}{Préfère les haches, les gourdins ou les lances.
}
\monsterdetail{Agressif}{Attaque à vue les créatures de même taille ou plus petites. Il les
poursuit jusqu'à ce qu'elles soient hors de vue.
}
\monstercarac{ca}{6 [13]}
\monstercarac{hd}{6 (27 pv)}
\monstercarac{taco}{14 [+5]}
\monstercarac{moral}{12}
\monstercarac{alignement}{Chaotique}
\monstercarac{xp}{275}
\monstercarac{nombre_donjon}{1d6}
\monstercarac{nombre_exterieur}{1d8}
\monstercarac{tresor}{C}
\monstercarac{mvt}{36 m (12 m)}
\monstercarac{save_mort_poison}{10}
\monstercarac{save_baguettes}{11}
\monstercarac{save_paralysie_petrification}{12}
\monstercarac{save_souffles}{13}
\monstercarac{save_sorts_sceptres_batons}{14}
\monsterattack{1 × corne }{1d6}
\monsterattack{1 × morsure }{1d6}
\monsterattack{1 × arme }{1d6+2 ou selon l’arme +2}
\end{monster}

\begin{monster}
\monstercarac{name}{Moisissure jaune}
\monstercarac{description}{
}
\monsterdetail{Superficie}{Chaque surface de 1 m² (par exemple, 1m × 1m) recouvert de moisissure
jaune est traité comme un « individu ». Une zone de 3 m × 3 m comprendra
donc 9 moisissures individuelles.
}
\monsterdetail{Immunité}{Immunisée à toutes les attaques hormis le feu (une torche allumée lui
inflige 1d4 dégâts).
}
\monsterdetail{Nuage de spores}{50 \% de chances de produire cette attaque si elle est touchée (ou subit
des dégâts). Elle libère alors un nuage de spores affectant tout ce qui
se trouve dans une zone de 3 m3.
}
\monsterdetail{Étouffement}{Jet de sauvegarde contre la mort ou mort dans les 6 rounds.
}
\monsterdetail{Corrosion}{Le bois ou le cuir en contact avec la moisissure sont rongés.
}
\monstercarac{ca}{Aucun jet requis}
\monstercarac{hd}{2* (9 pv)}
\monstercarac{taco}{18 [+1]}
\monstercarac{moral}{12}
\monstercarac{alignement}{Neutre}
\monstercarac{xp}{25}
\monstercarac{nombre_donjon}{1d8}
\monstercarac{nombre_exterieur}{1d4}
\monstercarac{tresor}{Aucun}
\monstercarac{save_mort_poison}{12}
\monstercarac{save_baguettes}{13}
\monstercarac{save_paralysie_petrification}{14}
\monstercarac{save_souffles}{15}
\monstercarac{save_sorts_sceptres_batons}{16}
\monsterattack{1 × spore }{1d6 + étouffement)}
\end{monster}

\begin{monster}
\monstercarac{name}{Molosse satanique}
\monstercarac{description}{
  Molosse aussi intelligent que vicieux, de la taille d'un petit poney.
Crache du feu et adore la chaleur. On le rencontre dans les donjons ou
près des volcans.

}
\monsterdetail{Souffle}{À chaque round, un molosse satanique a 2-sur-6 chances de souffler du
feu. Une seule cible. Jet de sauvegarde contre les souffles pour ne
subir que la moitié des dégâts.
}
\monsterdetail{Immunité au feu}{Immunisé au feu non magique.
}
\monsterdetail{Détection de l’invisible}{75 \% de chances par round, à 18 m alentour.
}
\monsterdetail{Adoption}{On les rencontre parfois en compagnie d'autres créatures liées au feu.
}
\monstercarac{ca}{4 [15]}
\monstercarac{hd}{3 to 7* (13/18/22/27/ 31 pv)}
\monstercarac{taco}{17 [+2] / 13 [+6]}
\monstercarac{moral}{9}
\monstercarac{alignement}{Chaotique}
\monstercarac{xp}{50/125/300/500/850}
\monstercarac{nombre_donjon}{2d4}
\monstercarac{nombre_exterieur}{2d4}
\monstercarac{tresor}{C}
\monstercarac{mvt}{36 m (12 m)}
\monstercarac{save_mort_poison}{}
\monstercarac{save_baguettes}{}
\monstercarac{save_paralysie_petrification}{}
\monstercarac{save_souffles}{}
\monstercarac{save_sorts_sceptres_batons}{}
\monsterattack{1 × morsure }{1d6}
\monsterattack{1 × souffle }{1d6 par DV}
\end{monster}

\begin{monster}
\monstercarac{name}{Momie}
\monstercarac{description}{
  Humanoïde mort-vivant enveloppé de bandages funéraires. On le rencontre
dans les ruines hantées et les tombeaux.

}
\monsterdetail{Paralyser par la terreur}{Quiconque voit une momie doit effectuer un jet de sauvegarde contre la
paralysie ou être paralysé par la terreur. La paralysie est rompue si la
momie attaque ou disparaît de la vue.
}
\monsterdetail{Maladie}{Toute personne touchée contracte une horrible maladie putrescente. La
guérison magique des blessures est inefficace ; la guérison naturelle
prend dix fois plus de temps. La maladie ne peut être éliminée que par
magie.
}
\monsterdetail{Immunité aux dégâts normaux}{Ne peut être blessée que par le feu ou la magie. Tous les dégâts sont
réduits de moitié.
}
\monsterdetail{Mort-vivant}{Ne fait aucun bruit jusqu'à ce qu'elle attaque. Insensible aux effets
affectant les créatures vivantes (par exemple, le poison). Immunisée
contre les sorts affectant ou lisant l'esprit (ex. : charme, paralysie,
sommeil).
}
\monstercarac{ca}{3 [16]}
\monstercarac{hd}{5+1* (23 pv)}
\monstercarac{taco}{14 [+5]}
\monstercarac{moral}{12}
\monstercarac{alignement}{Chaotique}
\monstercarac{xp}{400}
\monstercarac{nombre_donjon}{1d4}
\monstercarac{nombre_exterieur}{1d12}
\monstercarac{tresor}{D}
\monstercarac{save_mort_poison}{10}
\monstercarac{save_baguettes}{11}
\monstercarac{save_paralysie_petrification}{12}
\monstercarac{save_souffles}{13}
\monstercarac{save_sorts_sceptres_batons}{14}
\monsterattack{1 × toucher }{1d12 + maladie)}
\end{monster}

\begin{monster}
\monstercarac{img}{rust-monster-six.jpg}
\monstercarac{name}{Monstre rouilleur}
\monstercarac{description}{
  Créature magique qui ressemble à un tatou avec une longue queue et deux
longues antennes. Se nourrit de métal rouillé.

}
\monsterdetail{Rouille}{Le métal touché par un monstre rouilleur (par exemple, les armes qui le
touchent ou l'armure frappée par une antenne) s'effondre instantanément,
touché par la rouille. Les objets magiques ont 10\% de chances par 
"plus" de ne pas être affectés à chaque coup réussi. Chaque fois qu'un
objet magique est affecté, il perd un "plus".
}
\monsterdetail{Immunité aux dégâts normaux}{Ne peut être blessé que par des attaques magiques.
}
\monsterdetail{Sentir le métal}{Attiré par l'odeur du métal.
}
\monstercarac{ca}{2 [17]}
\monstercarac{hd}{5 (22 pv)}
\monstercarac{taco}{15 [+4]}
\monstercarac{moral}{7}
\monstercarac{alignement}{Neutre}
\monstercarac{xp}{175}
\monstercarac{nombre_donjon}{1d4}
\monstercarac{nombre_exterieur}{1d4}
\monstercarac{tresor}{Aucun}
\monstercarac{mvt}{36 m (12 m)}
\monstercarac{save_mort_poison}{12}
\monstercarac{save_baguettes}{13}
\monstercarac{save_paralysie_petrification}{14}
\monstercarac{save_souffles}{15}
\monstercarac{save_sorts_sceptres_batons}{16}
\monsterattack{1 × antenne }{rouille}
\end{monster}

\begin{monster}
\monstercarac{name}{Mouche voleuse}
\monstercarac{description}{
  Mouche carnivore aux rayures jaunes et noires, longue de 1 m. Ressemble
à l'abeille tueuse, qu'elle chasse d'ailleurs. Peut attaquer les
humains.

}
\monsterdetail{Surprise}{Chasse patiemment en attendant ses proies dans l'ombre et les surprend
4-sur-6 fois.
}
\monsterdetail{Immunité aux poisons}{Insensible au poison des abeilles tueuses.
}
\monsterdetail{Bond}{Elle peut sauter jusqu'à 9 m de haut et attaquer.
}
\monstercarac{ca}{6 [13]}
\monstercarac{hd}{2 (9 pv)}
\monstercarac{taco}{18 [+1]}
\monstercarac{moral}{8}
\monstercarac{alignement}{Neutre}
\monstercarac{xp}{20}
\monstercarac{nombre_donjon}{1d6}
\monstercarac{nombre_exterieur}{2d6}
\monstercarac{tresor}{U}
\monstercarac{mvt}{27 m (9 m) / 54 m (18 m) vol}
\monstercarac{save_mort_poison}{12}
\monstercarac{save_baguettes}{13}
\monstercarac{save_paralysie_petrification}{14}
\monstercarac{save_souffles}{15}
\monstercarac{save_sorts_sceptres_batons}{16}
\monsterattack{1 × morsure }{1d8}
\end{monster}

\begin{monster}
\monstercarac{name}{Mule}
\monstercarac{description}{
  Rejeton d'une jument et d'un âne connu pour son caractère borné, qui
sert souvent de bête de somme.

}
\monsterdetail{Tenace}{Si l'arbitre l'autorise, la mule peut être menée sous terre.
}
\monsterdetail{Défensive}{Peut attaquer si elle est menacée, mais on ne peut pas lui apprendre à
attaquer sur commande.
}
\monsterdetail{Bête de somme}{Peut transporter jusqu'à 2 000 pièces en charge normale et jusqu'à 4 000
pièces surchargée et à la moitié de sa vitesse.
}
\monstercarac{ca}{7 [12]}
\monstercarac{hd}{2 (9 pv)}
\monstercarac{taco}{18 [+1]}
\monstercarac{moral}{8}
\monstercarac{alignement}{Neutre}
\monstercarac{xp}{20}
\monstercarac{nombre_donjon}{1d8}
\monstercarac{nombre_exterieur}{2d6}
\monstercarac{tresor}{Aucun}
\monstercarac{save_mort_poison}{14}
\monstercarac{save_baguettes}{15}
\monstercarac{save_paralysie_petrification}{16}
\monstercarac{save_souffles}{17}
\monstercarac{save_sorts_sceptres_batons}{18}
\monsterattack{1 × ruade }{1d4) ou 1 × morsure}
\end{monster}

\input{liste/Musaraigne_géante.tex}
\input{liste/Médium.tex}
\input{liste/Méduse.tex}
\begin{monster}
\monstercarac{name}{Nain (Monstre)}
\monstercarac{description}{
  Ce semi-humain petit, trapu et barbu vit dans les montagnes et les
royaumes souterrains.

}
\monsterdetail{Chef}{Par groupe de 20 nains, on trouve un chef de niveau 1d6+2. Il peut avoir
des objets magiques : 5 \% de chances par niveau pour chaque tableau
d'objets magiques (à l'exception des parchemins et des baguettes /
bâtons / sceptres --- voir \href{/Objets_magiques_(généralités)}{Objets
magiques}).
}
\monsterdetail{Haine des gobeli}{: Les attaquent normalement à vue.
}
\monstercarac{ca}{4 [15]}
\monstercarac{hd}{1 (4 pv)}
\monstercarac{taco}{19 [0]}
\monstercarac{moral}{8 (10 avec un chef)}
\monstercarac{alignement}{Loyal ou Neutre}
\monstercarac{xp}{10}
\monstercarac{nombre_donjon}{1d6}
\monstercarac{nombre_exterieur}{5d8}
\monstercarac{tresor}{G}
\monstercarac{mvt}{18 m (6 m)}
\monstercarac{save_mort_poison}{8}
\monstercarac{save_baguettes}{9}
\monstercarac{save_paralysie_petrification}{10}
\monstercarac{save_souffles}{13}
\monstercarac{save_sorts_sceptres_batons}{12}
\monsterattack{1 × arme }{1d8 }
\monsterattack{selon l’arme)}{}
\end{monster}

\begin{monster}
\monstercarac{name}{Noble}
\monstercarac{description}{
  Puissant humain avec des titres de noblesse (par exemple, comte, duc,
chevalier, etc.) et qui habite dans un château.

}
\monsterdetail{Classe}{Généralement considéré comme un guerrier de 3e niveau, mais peut être en
fait de n'importe quelle classe et niveau.
}
\monsterdetail{Écuyer et suivants}{Accompagné d'un guerrier de 2e niveau (écuyer) et jusqu'à dix guerriers
de 1er niveau (suivants).
}
\monstercarac{ca}{2 [17]}
\monstercarac{hd}{3 (13 pv)}
\monstercarac{taco}{17 [+2]}
\monstercarac{moral}{8}
\monstercarac{alignement}{Tous}
\monstercarac{xp}{35}
\monstercarac{nombre_donjon}{2d6}
\monstercarac{nombre_exterieur}{2d6}
\monstercarac{tresor}{V × 3}
\monstercarac{save_mort_poison}{12}
\monstercarac{save_baguettes}{13}
\monstercarac{save_paralysie_petrification}{14}
\monstercarac{save_souffles}{15}
\monstercarac{save_sorts_sceptres_batons}{16}
\monsterattack{1 × arme }{1d8 ou selon l’arme)}
\end{monster}

\begin{monster}
\monstercarac{name}{Nomade}
\monstercarac{description}{
  Tribus superstitieuses qui errent dans les steppes et les régions
désertiques. Elles vivent dans des tentes ou des huttes temporaires.
Leur comportement dépend de la tribu : certaines sont belliqueuses,
d'autres pacifiques.

}
\monsterdetail{Monté}{À \href{Cheval.md\#Cheval-de-selle}{cheval de selle} ou à dos de
\href{Chameau.md}{chameau} (dans le désert).
}
\monsterdetail{Armes (désert)}{50 \% du groupe est équipé avec : une armure de cuir, un bouclier, une
lance ; 30 \% est équipé avec : une cotte de mailles, un bouclier, une
lance ; 20 \% est équipé avec : une armure de cuir, un arc court.
}
\monsterdetail{Armes (steppes)}{50 \% du groupe est équipé avec : une armure de cuir, un arc court ; 20
\% est équipé avec : une armure de cuir, un bouclier, une lance ; 20 \%
est équipé avec : une cotte de mailles, un arc court ; 10 \% est équipé
avec : une cotte de mailles, un bouclier, une lance et peut être monté
sur un cheval de guerre.
}
\monsterdetail{Chefs}{Par groupe de 25 nomades, on trouve un guerrier de 2e niveau. Par groupe
de 40 nomades, un guerrier de 4e niveau.
}
\monsterdetail{Camps}{Les groupes de chasseurs / cueilleurs s'unissent et vivent généralement
dans le cadre d'une tribu comptabilisant jusqu'à 300 guerriers nomades.
}
\monsterdetail{Chefs de camp}{Par groupe de 100 nomades, on trouve un chef de tribu (guerrier de 8e
niveau), plus un guerrier de 5e niveau. 50 \% de chances qu'il y ait un
clerc (9e niveau) ; 25 \% de chances qu'il y ait un magicien (8e
niveau).
}
\monsterdetail{Butin}{Ne possèdent qu'un trésor de type A lorsqu'on les rencontre dans le
camp.
}
\monsterdetail{Commerçants}{Ils acceptent souvent de raconter des récits de pays lointains dont ils
ont entendu parler sur les routes commerciales.
}
\monstercarac{ca}{7 [12] à 4}
\monstercarac{hd}{1 (4 pv)}
\monstercarac{taco}{19 [0]}
\monstercarac{moral}{8}
\monstercarac{alignement}{Tous}
\monstercarac{xp}{10}
\monstercarac{nombre_donjon}{0}
\monstercarac{nombre_exterieur}{1d4 × 10}
\monstercarac{tresor}{A}
\monstercarac{mvt}{36 m (12 m)}
\monstercarac{save_mort_poison}{12}
\monstercarac{save_baguettes}{13}
\monstercarac{save_paralysie_petrification}{14}
\monstercarac{save_souffles}{15}
\monstercarac{save_sorts_sceptres_batons}{16}
\monsterattack{1 × arme }{1d6 }
\monsterattack{selon l’arme)}{}
\end{monster}

\input{liste/Néandertalien_(homme_des_cavernes).tex}
\input{liste/Nécrophage.tex}
\begin{monster}
\monstercarac{name}{Ogre}
\monstercarac{description}{
  Humanoïde effrayant mesure entre 2,40 m et 3 m de haut. Vêtu de peaux de
bêtes, il vit souvent dans des grottes.

}
\monsterdetail{Sac}{À l'extérieur de son repaire, un ogre transporte un sac contenant 1d6 ×
100 po.
}
\monsterdetail{Haine des néandertaliens}{Les attaque à vue.
}
\monstercarac{ca}{5 [14]}
\monstercarac{hd}{4+1 (19 pv)}
\monstercarac{taco}{15 [+4]}
\monstercarac{moral}{10}
\monstercarac{alignement}{Chaotique}
\monstercarac{xp}{125}
\monstercarac{nombre_donjon}{1d6}
\monstercarac{nombre_exterieur}{2d6}
\monstercarac{tresor}{C + 1}
\monstercarac{mvt}{27 m (9 m)}
\monstercarac{save_mort_poison}{10}
\monstercarac{save_baguettes}{11}
\monstercarac{save_paralysie_petrification}{12}
\monstercarac{save_souffles}{13}
\monstercarac{save_sorts_sceptres_batons}{14}
\monsterattack{1 × gourdin }{1d10}
\end{monster}

\begin{monster}
\monstercarac{name}{Ombre}
\monstercarac{description}{
  Créature incorporelle et intelligente, qui n'est pas morte-vivante,
ressemble à une ombre. Sont capables de légèrement changer leur forme
pour se camoufler.

}
\monsterdetail{Surprise}{Sur un 1-à-5.
}
\monsterdetail{Absorption de Force}{Les victimes perdent 1 point de FOR à chaque attaque réussie. Ces points
sont récupérés après 8 tours. Si la FOR est réduite à 0, la victime
devient une ombre.
}
\monsterdetail{Immunité aux dégâts normaux}{Ne peut être blessée que par des attaques magiques.
}
\monsterdetail{Immunité aux sorts}{Immunisée contre les sorts charme et sommeil.
}
\monstercarac{ca}{7 [12]}
\monstercarac{hd}{2+2* (11 pv)}
\monstercarac{taco}{17 [+2]}
\monstercarac{moral}{12}
\monstercarac{alignement}{Chaotic}
\monstercarac{xp}{35}
\monstercarac{nombre_donjon}{1d8}
\monstercarac{nombre_exterieur}{1d12}
\monstercarac{tresor}{F}
\monstercarac{mvt}{27 m (9 m)}
\monstercarac{save_mort_poison}{12}
\monstercarac{save_baguettes}{13}
\monstercarac{save_paralysie_petrification}{14}
\monstercarac{save_souffles}{15}
\monstercarac{save_sorts_sceptres_batons}{16}
\monsterattack{1 × toucher }{1d4 + absorption de Force}
\end{monster}

\begin{monster}
\monstercarac{img}{Orc-5e.png}
\monstercarac{name}{Orque}
\monstercarac{description}{
  Humanoïde hideux aux traits animaliers et au tempérament colérique qui
vit sous terre et est actif la nuit. Brute sadique qui hait les autres
créatures et les met à mort avec délectation.

}
\monsterdetail{Détestation du soleil}{Malus de --1 à l'attaque en plein jour.
}
\monsterdetail{Armes}{Privilégient les haches, les gourdins, les épieux et les épées. Seuls
les chefs savent se servir des armes mécaniques comme les arbalètes et
les catapultes.
}
\monsterdetail{Lâche}{Ont peur des créatures plus grandes ou plus fortes qu'eux. Leurs chefs
peuvent cependant les forcer à les combattre.
}
\monsterdetail{Chef}{Les groupes sont menés par un orque avec 8 points de vie. Celui-ci
reçoit un bonus de +1 à ses jets de dégâts. Un orque devient chef après
avoir vaincu d'autres orques en combat.
}
\monsterdetail{Chef de tribu}{Une tribu orque est menée par un chef de tribu avec 4 DV. Celui-ci
reçoit un bonus de +2 à ses jets de dégâts.
}
\monsterdetail{Alliés géants}{Chaque groupe de 20 orques a 1-sur-6 chance d'être accompagné par un
\href{Ogre.md}{Ogre} . 1-sur-10 chance qu'un \href{Troll.md}{Troll} vive
dans leur repaire.
}
\monsterdetail{Tribal}{Les orques de tribus différentes se battent parfois entre eux, à moins
que leurs chefs les en empêchent. Chaque tribu a son propre repaire,
abritant autant de femelles que de mâles, et deux petits pour deux
adultes.
}
\monsterdetail{Mercenaire}{Peuvent être enrôlés dans les armées chaotiques. Adorent le pillage et
les massacres.
}
\monstercarac{ca}{6 [13]}
\monstercarac{hd}{1 (4 pv)}
\monstercarac{taco}{19 [0]}
\monstercarac{moral}{6 (8 avec un chef)}
\monstercarac{alignement}{Chaotique}
\monstercarac{xp}{10 (chef : 10, chef de tribu : 75)}
\monstercarac{nombre_donjon}{2d4}
\monstercarac{nombre_exterieur}{1d6 × 10}
\monstercarac{tresor}{D}
\monstercarac{mvt}{36 m (12 m)}
\monstercarac{save_mort_poison}{12}
\monstercarac{save_baguettes}{13}
\monstercarac{save_paralysie_petrification}{14}
\monstercarac{save_souffles}{15}
\monstercarac{save_sorts_sceptres_batons}{16}
\monsterattack{1 × arme }{1d6 selon l’arme}
\end{monster}

\begin{monster}
\monstercarac{img}{Owl Bear.png}
\monstercarac{name}{Ours-hibou}
\monstercarac{description}{
  Cette énorme créature (2,40 m de haut et 700 kg), souvent irascible,
ressemble à un ours carnivore au visage de hibou. On rencontre
l'ours-hiboux dans les forêts denses et les souterrains.

}
\monsterdetail{Étreinte d’ours}{Si une victime est touchée par les deux pattes au cours du même round,
l'ours-hibou l'étreint et lui inflige 2d8 dégâts automatiques
supplémentaires.
}
\monstercarac{ca}{5 [14]}
\monstercarac{hd}{5 (22 pv)}
\monstercarac{taco}{15 [+4]}
\monstercarac{moral}{9}
\monstercarac{alignement}{Neutre}
\monstercarac{xp}{175}
\monstercarac{nombre_donjon}{1d4}
\monstercarac{nombre_exterieur}{1d4}
\monstercarac{tresor}{C}
\monstercarac{mvt}{36 m (12 m)}
\monstercarac{save_mort_poison}{12}
\monstercarac{save_baguettes}{13}
\monstercarac{save_paralysie_petrification}{14}
\monstercarac{save_souffles}{15}
\monstercarac{save_sorts_sceptres_batons}{16}
\monsterattack{2 × griffe }{1d8}
\monsterattack{1 × morsure }{1d8}
\end{monster}

\begin{monster}
\monstercarac{name}{Grizzly}
\monstercarac{description}{
  Agressif, cet ours de 2,70 m de haut, possède une fourrure brune ou
rousse, aux extrémités grises, et vit dans les forêts et les montagnes.
Apprécie la viande.

}
\monsterdetail{Étreinte d’ours}{Si un adversaire est touché par les deux pattes d'un ours le même round,
l'étreinte de l'ours lui inflige 2d8 points de dégâts automatiques
supplémentaires.
}
\monstercarac{ca}{6 [13]}
\monstercarac{hd}{5 (22 pv)}
\monstercarac{taco}{15 [+4]}
\monstercarac{moral}{8}
\monstercarac{alignement}{Neutre}
\monstercarac{xp}{175}
\monstercarac{nombre_donjon}{1}
\monstercarac{nombre_exterieur}{1d4}
\monstercarac{tresor}{U}
\monstercarac{mvt}{36 m (12 m)}
\monstercarac{save_mort_poison}{12}
\monstercarac{save_baguettes}{13}
\monstercarac{save_paralysie_petrification}{14}
\monstercarac{save_souffles}{15}
\monstercarac{save_sorts_sceptres_batons}{16}
\monsterattack{2 × griffes }{1d4}
\monsterattack{1 × morsure }{1d8}
\end{monster}

\begin{monster}
\monstercarac{name}{Ours brun}
\monstercarac{description}{
  Cet ours haut de 1,80 m apprécie les baies et les racines.

}
\monsterdetail{Étreinte d’ours}{Si un adversaire est touché par les deux pattes d'un ours le même round,
l'étreinte de l'ours lui inflige 2d8 points de dégâts automatiques
supplémentaires.
}
\monsterdetail{Défensif}{Les adultes protègent leurs petits au péril de leur vie. En dehors de
cela, l'ours brun n'attaque que s'il est acculé.
}
\monsterdetail{Attaque de campements}{Attaque parfois des campements pour se nourrir --- particulièrement s'il
s'agit de poissons ou de friandises.
}
\monstercarac{ca}{6 [13]}
\monstercarac{hd}{4 (18 pv)}
\monstercarac{taco}{16 [+3]}
\monstercarac{moral}{7}
\monstercarac{alignement}{Neutre}
\monstercarac{xp}{75}
\monstercarac{nombre_donjon}{1d4}
\monstercarac{nombre_exterieur}{1d4}
\monstercarac{tresor}{U}
\monstercarac{mvt}{36 m (12 m)}
\monstercarac{save_mort_poison}{12}
\monstercarac{save_baguettes}{13}
\monstercarac{save_paralysie_petrification}{14}
\monstercarac{save_souffles}{15}
\monstercarac{save_sorts_sceptres_batons}{16}
\monsterattack{2 × griffes }{1d3}
\monsterattack{1 × morsure }{1d6}
\end{monster}

\begin{monster}
\monstercarac{img}{Bear-cave-image.jpg}
\monstercarac{name}{Ours des cavernes}
\monstercarac{description}{
  On peut rencontrer ce grizzly féroce de 4,50 m de haut dans les cavernes
ou dans le Monde perdu. Omnivore, il préfère se nourrir de viande, y
compris humaine.

}
\monsterdetail{Étreinte d’ours}{Si un adversaire est touché par les deux pattes d'un ours le même round,
l'étreinte de l'ours lui inflige 2d8 points de dégâts automatiques
supplémentaires.
}
\monsterdetail{Odorat développé}{Possède une pauvre vue, mais un odorat aiguisé. Quand il est affamé, il
peut suivre une piste sanglante grâce à ce flair.
}
\monstercarac{ca}{5 [14]}
\monstercarac{hd}{7 (31 pv)}
\monstercarac{taco}{13 [+6]}
\monstercarac{moral}{9}
\monstercarac{alignement}{Neutre}
\monstercarac{xp}{450}
\monstercarac{nombre_donjon}{1d2}
\monstercarac{nombre_exterieur}{1d2}
\monstercarac{tresor}{V}
\monstercarac{mvt}{36 m (12 m)}
\monstercarac{save_mort_poison}{12}
\monstercarac{save_baguettes}{13}
\monstercarac{save_paralysie_petrification}{14}
\monstercarac{save_souffles}{15}
\monstercarac{save_sorts_sceptres_batons}{16}
\monsterattack{2 × griffes }{1d8}
\monsterattack{1 × morsure }{2d6}
\end{monster}

\begin{monster}
\monstercarac{name}{Ours polaire}
\monstercarac{description}{
  Agressif, cet ours blanc peut atteindre les 3,40 m de hauteur et vi dans
les régions froides, appréciant le poisson.

}
\monsterdetail{Étreinte d’ours}{Si un adversaire est touché par les deux pattes d'un ours le même round,
l'étreinte de l'ours lui inflige 2d8 points de dégâts automatiques
supplémentaires.
}
\monsterdetail{Nageur}{Excellent nageur.
}
\monsterdetail{Marche dans la neige}{Ses grandes pattes sont faites pour courir dans la neige sans s'y
enfoncer.
}
\monstercarac{ca}{6 [13]}
\monstercarac{hd}{6 (27 pv)}
\monstercarac{taco}{14 [+5]}
\monstercarac{moral}{8}
\monstercarac{alignement}{Neutre}
\monstercarac{xp}{275}
\monstercarac{nombre_donjon}{1}
\monstercarac{nombre_exterieur}{1d2}
\monstercarac{tresor}{U}
\monstercarac{mvt}{36 m (12 m)}
\monstercarac{save_mort_poison}{12}
\monstercarac{save_baguettes}{13}
\monstercarac{save_paralysie_petrification}{14}
\monstercarac{save_souffles}{15}
\monstercarac{save_sorts_sceptres_batons}{16}
\monsterattack{2 × griffes }{1d6}
\monsterattack{1 × morsure }{1d10}
\end{monster}

\input{liste/Pieuvre_géante.tex}
\begin{monster}
\monstercarac{img}{Pirate.jpg}
\monstercarac{name}{Pirate}
\monstercarac{description}{
  Ces marins gagnent leur vie en attaquant les villages côtiers, en volant
d'autres navires et en faisant commerce illégal d'esclaves. On les
croise généralement en haute mer. Réputés pour leur conduite mauvaise,
ils sont impitoyables.

}
\monsterdetail{Navires et équipage}{Ils dépendent de l'endroit où on rencontre les pirates. Rivières ou lacs
: 1d8 bateaux fluviaux avec chacun 1d2 × 10 pirates ; eaux côtières :
1d6 petites galères avec chacune 1d3+1 × 10 pirates ; tous : drakkars
1d4 avec chacun 1d3+2 × 10 pirates ; océan : 1d3 petits navires de
guerre avec chacun 1d5+3 × 10 pirates.
}
\monsterdetail{Armes}{50 \% du groupe est équipé avec : une armure de cuir et une épée ; 35 \%
est équipé avec : une armure de cuir, une épée et une arbalète ; 15 \%
est équipé avec : une cotte de mailles et une épée.
}
\monsterdetail{Chefs}{Par groupe de 30 pirates, on trouve un guerrier de 4e niveau. Par groupe
de 50 pirates et pour chaque navire, un guerrier de 5e niveau. Par
groupe de 100 pirates et par flotte, un guerrier de 8e niveau.
}
\monsterdetail{Commandant de flotte}{Les flottes de 300 pirates ou plus sont dirigées par un seigneur pirate
(guerrier de 11eniveau). 75 \% de chances qu'il y ait un magicien de
niveau 1d2+8.
}
\monsterdetail{Traître}{Les pirates n'hésitent pas à attaquer d'autres pirates, s'ils y trouvent
leur profit.
}
\monsterdetail{Prisonniers}{Il y a 25 \% de chances de demandes de rançons pour 1d3 de leurs
prisonniers.
}
\monsterdetail{Trésor}{Réparti entre les vaisseaux de la flotte. Au lieu de l'avoir à bord, ils
peuvent avoir une carte de l'endroit où il est enterré.
}
\monsterdetail{Havres}{Les villes côtières fortifiées où aucune loi n'a cours peuvent
constituer un refuge pour les pirates.
}
\monstercarac{ca}{7 [12] or 5}
\monstercarac{hd}{1 (4 pv)}
\monstercarac{taco}{19 [0]}
\monstercarac{moral}{7}
\monstercarac{alignement}{Chaotique}
\monstercarac{xp}{10}
\monstercarac{nombre_donjon}{0}
\monstercarac{nombre_exterieur}{voir ci-dessous}
\monstercarac{tresor}{A}
\monstercarac{mvt}{36 m (12 m)}
\monstercarac{save_mort_poison}{12}
\monstercarac{save_baguettes}{13}
\monstercarac{save_paralysie_petrification}{14}
\monstercarac{save_souffles}{15}
\monstercarac{save_sorts_sceptres_batons}{16}
\monsterattack{1 × arme }{1d6 selon l’arme}
\end{monster}

\begin{monster}
\monstercarac{name}{Pixie}
\monstercarac{description}{
  Petit humanoïde (de 30 à 60 cm) aux ailes d'insectes, lointain cousin
des elfes.

}
\monsterdetail{Invisibilité}{Naturellement invisibles, peuvent cependant choisir de se révéler aux
autres. Peuvent rester invisibles lorsqu'ils attaquent : on ne peut pas
les attaquer au premier round et, par la suite, les attaques se font
avec un malus de -- 2 pour toucher (localisable en suivant les ombres
translucides et les mouvements d'air).
}
\monsterdetail{Surprise}{S'ils sont invisibles, ils surprennent automatiquement.
}
\monsterdetail{Vol limité}{Ils ne peuvent voler que pendant 3 tours et doivent ensuite reposer
leurs petites ailes pendant un tour.
}
\monstercarac{ca}{3 [16]}
\monstercarac{hd}{1* (4 pv)}
\monstercarac{taco}{19 [0]}
\monstercarac{moral}{7}
\monstercarac{alignement}{Neutral}
\monstercarac{xp}{13}
\monstercarac{nombre_donjon}{2d4}
\monstercarac{nombre_exterieur}{1d4 × 10}
\monstercarac{tresor}{R + S}
\monstercarac{save_mort_poison}{12}
\monstercarac{save_baguettes}{13}
\monstercarac{save_paralysie_petrification}{13}
\monstercarac{save_souffles}{15}
\monstercarac{save_sorts_sceptres_batons}{15}
\monsterattack{1 × dague }{1d4)}
\end{monster}

\input{liste/Carpe_géante.tex}
\input{liste/Esturgeon_géant.tex}
\input{liste/Piranha_géant.tex}
\input{liste/Rascasse_géante.tex}
\input{liste/Silure_géant.tex}
\input{liste/Ptéranodon.tex}
\input{liste/Ptérodactyle.tex}
\begin{monster}
\monstercarac{name}{Pudding noir}
\monstercarac{description}{
  Sorte d'énorme blob stupide et aveugle de gelée noire (long de 1,50 à 9
m), poussé par une faim vorace.

}
\monsterdetail{Immunité}{Ne peut être blessé que par les attaques basées sur le feu.
}
\monsterdetail{Division}{Les attaques qui ne sont pas liées au feu (y compris les sorts)
provoquent la division du pudding. Chaque coup crée un nouveau pudding
de 2 DV qui inflige 1d8 points de dégâts.
}
\monsterdetail{Corrode le bois et le métal}{Peut dissoudre le bois ou le métal en un tour.
}
\monsterdetail{Collant}{Peut se déplacer sur les murs et les plafonds.
}
\monsterdetail{Suintement}{Peut s'infiltrer à travers les petits trous et fissures.
}
\monstercarac{ca}{6 [13]}
\monstercarac{hd}{10* (45 pv)}
\monstercarac{taco}{11 [+8]}
\monstercarac{moral}{12}
\monstercarac{alignement}{Neutre}
\monstercarac{xp}{1 600}
\monstercarac{nombre_donjon}{1}
\monstercarac{nombre_exterieur}{0}
\monstercarac{tresor}{Aucun}
\monstercarac{mvt}{18 m (6 m)}
\monstercarac{save_mort_poison}{10}
\monstercarac{save_baguettes}{11}
\monstercarac{save_paralysie_petrification}{12}
\monstercarac{save_souffles}{13}
\monstercarac{save_sorts_sceptres_batons}{14}
\monsterattack{1 × toucher }{3d8}
\end{monster}

\input{liste/Pégase.tex}
\begin{monster}
\monstercarac{img}{rat.jpg}
\monstercarac{name}{Rat normal}
\monstercarac{description}{
  Rongeurs innombrables et perclus de maladies, qui dévorent tout. Ils
évitent le contact avec les humains, mais peuvent s'attaquer à eux pour
défendre leur nid ou s'ils sont convoqués et commandés par magie (voir
\href{Lycanthrope.md}{Rat-garou}).

  Meute de rat grouillants de 15 à 60 cm de long, à la fourrure brune ou
grise.

}
\monsterdetail{Maladie}{Une morsure a 1-sur-20 chance d'infecter la cible (jet de sauvegarde
contre le poison). La maladie a 1-sur-4 chance d'être mortelle (mort en
1d6 jours). Sinon, la victime est malade et alitée pendant un mois.
}
\monsterdetail{Peur du feu}{Fuient le feu, à moins d'être forcés de se battre par la personne qui
les a convoqués.
}
\monsterdetail{Attaque dans l’eau}{Excellents nageurs, attaquent sans malus.
}
\monsterdetail{Meute}{Chaque groupe de 5 à 10 rats attaque en meute. Chaque meute n'effectue
qu'une seule attaque contre une créature donnée.
}
\monsterdetail{Engloutissement}{La créature attaquée doit effectuer un jet de sauvegarde contre la mort
ou tomber à terre et être dans l'incapacité d'attaquer jusqu'à ce
qu'elle se soit relevée.
}
\monstercarac{ca}{9 [10]}
\monstercarac{hd}{1 pv}
\monstercarac{taco}{19 [0]}
\monstercarac{moral}{5}
\monstercarac{alignement}{Neutre}
\monstercarac{xp}{5}
\monstercarac{nombre_donjon}{5d10}
\monstercarac{nombre_exterieur}{2d10}
\monstercarac{tresor}{L}
\monstercarac{mvt}{18 m (6 m) / 9 m (3 m) nage}
\monstercarac{save_mort_poison}{14}
\monstercarac{save_baguettes}{15}
\monstercarac{save_paralysie_petrification}{16}
\monstercarac{save_souffles}{17}
\monstercarac{save_sorts_sceptres_batons}{18}
\monsterattack{1 × morsure par meute }{1d6 + maladie}
\end{monster}

\input{liste/Rat_géant.tex}
\begin{monster}
\monstercarac{name}{Grand requin blanc}
\monstercarac{description}{
  Prédateur agressif, de faible intelligence et au comportement
imprévisible. Vit dans les eaux salées.

  Long de 9 m (ou plus !), de coloration grise et un ventre blanc, il
attaque parfois de petits bateaux.

}
\monsterdetail{Goût du sang}{Peut détecter du sang dans l'eau jusqu'à 90 m de distance.
}
\monsterdetail{Frénésie dévorante}{Déclenchée par le goût du sang : attaque toujours ; aucun jet de moral.
}
\monstercarac{ca}{4 [15]}
\monstercarac{hd}{8 (36 pv)}
\monstercarac{taco}{12 [+7]}
\monstercarac{moral}{7}
\monstercarac{alignement}{Neutre}
\monstercarac{xp}{650}
\monstercarac{nombre_donjon}{0}
\monstercarac{nombre_exterieur}{1d4}
\monstercarac{tresor}{Aucun}
\monstercarac{mvt}{54 m (18 m)}
\monstercarac{save_mort_poison}{10}
\monstercarac{save_baguettes}{11}
\monstercarac{save_paralysie_petrification}{12}
\monstercarac{save_souffles}{13}
\monstercarac{save_sorts_sceptres_batons}{14}
\monsterattack{1 × morsure }{2d10}
\end{monster}

\begin{monster}
\monstercarac{img}{Bull Shark.jpg}
\monstercarac{name}{Requin-bouledogue}
\monstercarac{description}{
  Prédateur agressif, de faible intelligence et au comportement
imprévisible. Vit dans les eaux salées.

  Long de 2,40 m, de couleur foncée, il attaque en heurtant sa proie pour
la mordre ensuite alors qu'elle est étourdie.

}
\monsterdetail{Goût du sang}{Peut détecter du sang dans l'eau jusqu'à 90 m de distance.
}
\monsterdetail{Frénésie dévorante}{Déclenchée par le goût du sang : attaque toujours ; aucun jet de moral.
}
\monsterdetail{Percussion}{Jet de sauvegarde contre la paralysie ou être étourdie pendant 3 rounds.
}
\monstercarac{ca}{4 [15]}
\monstercarac{hd}{2 (9 pv)}
\monstercarac{taco}{18 [+1]}
\monstercarac{moral}{7}
\monstercarac{alignement}{Neutre}
\monstercarac{xp}{20}
\monstercarac{nombre_donjon}{0}
\monstercarac{nombre_exterieur}{3d6}
\monstercarac{tresor}{Aucun}
\monstercarac{mvt}{54 m (18 m)}
\monstercarac{save_mort_poison}{12}
\monstercarac{save_baguettes}{13}
\monstercarac{save_paralysie_petrification}{14}
\monstercarac{save_souffles}{15}
\monstercarac{save_sorts_sceptres_batons}{16}
\monsterattack{1 × morsure }{2d4}
\monsterattack{1 × percussion }{étourdissement}
\end{monster}

\begin{monster}
\monstercarac{name}{Requin-taupe bleu}
\monstercarac{description}{
  Prédateur agressif, de faible intelligence et au comportement
imprévisible. Vit dans les eaux salées.

  Long de 4,50 m et de couleur gris bleuté ou beige.

}
\monsterdetail{Goût du sang}{Peut détecter du sang dans l'eau jusqu'à 90 m de distance.
}
\monsterdetail{Frénésie dévorante}{Déclenchée par le goût du sang : attaque toujours ; aucun jet de moral.
}
\monsterdetail{Imprévisible}{Peut ignorer des créatures à proximité, pour les attaquer quelques
instants plus tard.
}
\monstercarac{ca}{4 [15]}
\monstercarac{hd}{4 (18 pv)}
\monstercarac{taco}{16 [+3]}
\monstercarac{moral}{7}
\monstercarac{alignement}{Neutre}
\monstercarac{xp}{75}
\monstercarac{nombre_donjon}{0}
\monstercarac{nombre_exterieur}{2d6}
\monstercarac{tresor}{Aucun}
\monstercarac{save_mort_poison}{12}
\monstercarac{save_baguettes}{13}
\monstercarac{save_paralysie_petrification}{14}
\monstercarac{save_souffles}{15}
\monstercarac{save_sorts_sceptres_batons}{16}
\monsterattack{1 × morsure }{2d6)}
\end{monster}

\begin{monster}
\monstercarac{name}{Rhagodessa}
\monstercarac{description}{
  Énorme arachnide nocturne poilu, avec une grande tête et de grandes
mandibules, et muni de 10 pattes. Sa tête et son abdomen sont jaunes,
son thorax brun foncé. Habite dans des grottes, chasse avec voracité et
est carnivore.

}
\monsterdetail{Ventouses}{Pattes avant sont équipées de ventouses lui permettant de saisir ses
proies.
}
\monsterdetail{Étreinte}{Lorsqu'elle est frappée avec une ventouse, la victime reste collée et
est automatiquement mordue au tour suivant.
}
\monsterdetail{Collant}{Peut marcher sur les murs.
}
\monstercarac{ca}{5 [14]}
\monstercarac{hd}{4+2 (20 pv)}
\monstercarac{taco}{15 [+4]}
\monstercarac{moral}{9}
\monstercarac{alignement}{Neutre}
\monstercarac{xp}{125}
\monstercarac{nombre_donjon}{1d4}
\monstercarac{nombre_exterieur}{1d6}
\monstercarac{tresor}{U}
\monstercarac{save_mort_poison}{12}
\monstercarac{save_baguettes}{13}
\monstercarac{save_paralysie_petrification}{14}
\monstercarac{save_souffles}{15}
\monstercarac{save_sorts_sceptres_batons}{16}
\monsterattack{1 × ventouse }{étreinte)}
\monsterattack{1 × morsure }{2d8)}
\end{monster}

\input{liste/Rhinocéros_normal.tex}
\input{liste/Rhinocéros_laineux.tex}
\begin{monster}
\monstercarac{img}{Roc.png}
\monstercarac{name}{Petit Roc}
\monstercarac{description}{
  Ce gigantesque rapace niche dans les plus hauts sommets de chaînes de
montagnes isolées. Attaque les intrus si on l'approche négligemment.

}
\monsterdetail{Réaction d’alignement}{Malus de --1 aux jets de réaction contre les personnages neutres ; malus
de --2 contre les personnages chaotiques.
}
\monsterdetail{Descente en piqué}{Peut plonger sur les victimes visibles en contrebas. Si la victime est
surprise, l'attaque lui inflige doubles dégâts. Sur un jet d'attaque de
18 ou plus, la victime peut être emportée (à condition qu'elle soit de
taille appropriée).
}
\monsterdetail{Œufs}{50 \% des nids contiennent 1d6 œufs ou poussins.
}
\monsterdetail{Dressage}{Peut être dressé s'il est capturé jeune.
}
\monsterdetail{Réaction d’alignement}{Malus de --1 aux jets de réaction contre les personnages neutres ; malus
de --2 contre les personnages chaotiques.
}
\monsterdetail{Descente en piqué}{Peut plonger sur les victimes visibles en contrebas. Si la victime est
surprise, l'attaque lui inflige doubles dégâts. Sur un jet d'attaque de
18 ou plus, la victime peut être emportée (à condition qu'elle soit de
taille appropriée).
}
\monstercarac{ca}{4 [15]}
\monstercarac{hd}{6 (27 pv)}
\monstercarac{taco}{14 [+5]}
\monstercarac{moral}{8 (12 dans son repaire)}
\monstercarac{alignement}{Loyal}
\monstercarac{xp}{275}
\monstercarac{nombre_donjon}{0}
\monstercarac{nombre_exterieur}{1d12}
\monstercarac{tresor}{I}
\monstercarac{mvt}{18 m (6 m) / 144 m (48 m) vol}
\monstercarac{save_mort_poison}{12}
\monstercarac{save_baguettes}{13}
\monstercarac{save_paralysie_petrification}{14}
\monstercarac{save_souffles}{15}
\monstercarac{save_sorts_sceptres_batons}{16}
\monsterattack{2 × serre }{1d4+1}
\monsterattack{1 × morsure }{2d6}
\end{monster}

\begin{monster}
\monstercarac{name}{Grand Roc}
\monstercarac{description}{
  Ce gigantesque rapace niche dans les plus hauts sommets de chaînes de
montagnes isolées. Attaque les intrus si on l'approche négligemment.

}
\monsterdetail{Réaction d’alignement}{Malus de --1 aux jets de réaction contre les personnages neutres ; malus
de --2 contre les personnages chaotiques.
}
\monsterdetail{Descente en piqué}{Peut plonger sur les victimes visibles en contrebas. Si la victime est
surprise, l'attaque lui inflige doubles dégâts. Sur un jet d'attaque de
18 ou plus, la victime peut être emportée (à condition qu'elle soit de
taille appropriée).
}
\monsterdetail{Œufs}{50 \% des nids contiennent 1d6 œufs ou poussins.
}
\monsterdetail{Dressage}{Peut être dressé s'il est capturé jeune.
}
\monsterdetail{Réaction d’alignement}{Malus de --1 aux jets de réaction contre les personnages neutres ; malus
de --2 contre les personnages chaotiques.
}
\monsterdetail{Descente en piqué}{Peut plonger sur les victimes visibles en contrebas. Si la victime est
surprise, l'attaque lui inflige doubles dégâts. Sur un jet d'attaque de
18 ou plus, la victime peut être emportée (à condition qu'elle soit de
taille appropriée).
}
\monstercarac{ca}{2 [17]}
\monstercarac{hd}{12 (54 pv)}
\monstercarac{taco}{10 [+9]}
\monstercarac{moral}{9 (12 dans son repaire)}
\monstercarac{alignement}{Loyal}
\monstercarac{xp}{1 100}
\monstercarac{nombre_donjon}{0}
\monstercarac{nombre_exterieur}{1d8}
\monstercarac{tresor}{I}
\monstercarac{mvt}{18 m (6 m) / 144 m (48 m) vol}
\monstercarac{save_mort_poison}{10}
\monstercarac{save_baguettes}{11}
\monstercarac{save_paralysie_petrification}{12}
\monstercarac{save_souffles}{13}
\monstercarac{save_sorts_sceptres_batons}{14}
\monsterattack{2 × serre }{1d8}
\monsterattack{1 × morsure }{2d10}
\end{monster}

\input{liste/Roc_géant.tex}
\begin{monster}
\monstercarac{name}{Salamandre de feu}
\monstercarac{description}{
  Ce reptile magique géant a une attirance pour les températures extrêmes
(chaud ou froid).

  Serpent intelligent, de 3,60 à 4,80 m de long, avec une tête et des
pattes semblables aux lézards. Écailles orange, jaune ou rouge vif.
Provient du plan du feu élémentaire, mais vit également dans les volcans
et les déserts brûlants.

}
\monsterdetail{Aura de chaleur}{Toutes les créatures à moins de 6 m subissent 1d8 points de dégâts par
round.
}
\monsterdetail{Immunité aux dégâts normaux}{Ne peut être blessée que par des attaques magiques.
}
\monsterdetail{Immunité au feu}{Immunisée contre les dégâts de feu.
}
\monsterdetail{Haine des salamandres des glaces}{Les attaque à vue.
}
\monstercarac{ca}{2 [17]}
\monstercarac{hd}{8* (36 pv)}
\monstercarac{taco}{12 [+7]}
\monstercarac{moral}{8}
\monstercarac{alignement}{Neutre}
\monstercarac{xp}{1 200}
\monstercarac{nombre_donjon}{1d4+1}
\monstercarac{nombre_exterieur}{2d4}
\monstercarac{tresor}{F}
\monstercarac{save_mort_poison}{8}
\monstercarac{save_baguettes}{9}
\monstercarac{save_paralysie_petrification}{10}
\monstercarac{save_souffles}{10}
\monstercarac{save_sorts_sceptres_batons}{12}
\monsterattack{2 × griffes }{1d4)}
\monsterattack{1 × morsure }{1d8)}
\monsterattack{1 × aura de chaleur }{1d8)}
\end{monster}

\begin{monster}
\monstercarac{name}{Salamandre des glaces}
\monstercarac{description}{
  Ce reptile magique géant a une attirance pour les températures extrêmes
(chaud ou froid).

  Lézard géant à 6 pattes et aux écailles bleues ou blanches. Habite dans
les régions glaciales des contrées sauvages.

}
\monsterdetail{Aura de froid}{Toutes les créatures à moins de 6 m subissent 1d8 points de dégâts par
round.
}
\monsterdetail{Immunité aux dégâts normaux}{Ne peut être blessée que par des attaques magiques.
}
\monsterdetail{Immunité au froid}{Immunisée contre les dégâts de froid.
}
\monsterdetail{Haine des salamandres de feu}{Les attaque à vue.
}
\monstercarac{ca}{3 [16]}
\monstercarac{hd}{12* (54 pv)}
\monstercarac{taco}{10 [+9]}
\monstercarac{moral}{9}
\monstercarac{alignement}{Chaotique}
\monstercarac{xp}{1 900}
\monstercarac{nombre_donjon}{1d3}
\monstercarac{nombre_exterieur}{1d3}
\monstercarac{tresor}{E}
\monstercarac{save_mort_poison}{6}
\monstercarac{save_baguettes}{7}
\monstercarac{save_paralysie_petrification}{8}
\monstercarac{save_souffles}{8}
\monstercarac{save_sorts_sceptres_batons}{10}
\monsterattack{4 × griffes }{1d6)}
\monsterattack{1 × morsure }{2d6)}
\monsterattack{1 × aura de froid }{1d8)}
\end{monster}

\begin{monster}
\monstercarac{name}{Sanglier}
\monstercarac{description}{
  Ce sanglier sauvage omnivore vit essentiellement en forêt. Facilement
irritable et devient dangereux s'il est dérangé.

}
\monstercarac{ca}{7 [12]}
\monstercarac{hd}{3 (13 pv)}
\monstercarac{taco}{17 [+2]}
\monstercarac{moral}{9}
\monstercarac{alignement}{Neutre}
\monstercarac{xp}{35}
\monstercarac{nombre_donjon}{1d6}
\monstercarac{nombre_exterieur}{1d6}
\monstercarac{tresor}{Aucun}
\monstercarac{mvt}{45 m (15 m)}
\monstercarac{save_mort_poison}{12}
\monstercarac{save_baguettes}{13}
\monstercarac{save_paralysie_petrification}{14}
\monstercarac{save_souffles}{15}
\monstercarac{save_sorts_sceptres_batons}{16}
\monsterattack{1 × défense }{2d4}
\end{monster}

\input{liste/Sangsue_géante.tex}
\input{liste/Sauterelle_géante.tex}
\input{liste/Scarabée_de_feu.tex}
\begin{monster}
\monstercarac{name}{Scarabée à huile}
\monstercarac{description}{
  On peut rencontrer sous terre ce coléoptère fouisseur de 1 m de long.

}
\monsterdetail{Crachat huileux}{En réaction à une attaque, cible un adversaire à 1,50 m de lui. Une
attaque réussie provoque des cloques sur la peau infligeant un malus de
-- 2 aux jets d'attaques pendant 24 heures. Le sort Guérison des
blessures légères peut être utilisé pour soigner ces cloques, les
faisant disparaître au lieu de rendre des points de vie.
}
\monstercarac{ca}{4 [15]}
\monstercarac{hd}{2* (9 pv)}
\monstercarac{taco}{18 [+1]}
\monstercarac{moral}{8}
\monstercarac{alignement}{Neutre}
\monstercarac{xp}{25}
\monstercarac{nombre_donjon}{1d8}
\monstercarac{nombre_exterieur}{2d6}
\monstercarac{tresor}{Aucun}
\monstercarac{save_mort_poison}{12}
\monstercarac{save_baguettes}{13}
\monstercarac{save_paralysie_petrification}{14}
\monstercarac{save_souffles}{15}
\monstercarac{save_sorts_sceptres_batons}{16}
\monsterattack{1 × morsure }{1d6)}
\monsterattack{1 × crachat huileux }{cloques)}
\end{monster}

\begin{monster}
\monstercarac{name}{Scarabée tigré}
\monstercarac{description}{
  Coléoptère carnivore, long de 1,20 m, à la carapace tigrée et muni de
puissantes mandibules. Chasse les asilidés, mais peut parfois s'attaquer
à des humains.

}
\monstercarac{ca}{3 [16]}
\monstercarac{hd}{3+1 (14 pv)}
\monstercarac{taco}{16 [+3]}
\monstercarac{moral}{9}
\monstercarac{alignement}{Neutre}
\monstercarac{xp}{50}
\monstercarac{nombre_donjon}{1d6}
\monstercarac{nombre_exterieur}{2d4}
\monstercarac{tresor}{U}
\monstercarac{mvt}{45 m (15 m)}
\monstercarac{save_mort_poison}{12}
\monstercarac{save_baguettes}{13}
\monstercarac{save_paralysie_petrification}{14}
\monstercarac{save_souffles}{15}
\monstercarac{save_sorts_sceptres_batons}{16}
\monsterattack{1 × morsure }{2d6}
\end{monster}

\input{liste/Scorpion_géant.tex}
\begin{monster}
\monstercarac{name}{Cobra cracheur}
\monstercarac{description}{
  Existe dans tous les climats, sauf les plus extrêmes. N'attaque
généralement que s'il est acculé ou surpris.

  Serpent de 1 m de long, avec des écailles grises ou blanches. Préfère
attaquer à distance avec son crachat.

}
\monsterdetail{Crachat aveuglant}{Portée : 1,80 m. Provoque une cécité permanente (jet de sauvegarde
contre le poison).
}
\monsterdetail{Poison}{Provoque la mort en 1d10 tours (jet de sauvegarde contre le poison).
}
\monstercarac{ca}{7 [12]}
\monstercarac{hd}{1* (4 pv)}
\monstercarac{taco}{19 [0]}
\monstercarac{moral}{7}
\monstercarac{alignement}{Neutre}
\monstercarac{xp}{13}
\monstercarac{nombre_donjon}{1d6}
\monstercarac{nombre_exterieur}{1d6}
\monstercarac{tresor}{Aucun}
\monstercarac{mvt}{27 m (9 m)}
\monstercarac{save_mort_poison}{12}
\monstercarac{save_baguettes}{13}
\monstercarac{save_paralysie_petrification}{14}
\monstercarac{save_souffles}{15}
\monstercarac{save_sorts_sceptres_batons}{16}
\monsterattack{1 × crachat }{cécité}
\monsterattack{1 × morsure }{1d3 + poison}
\end{monster}

\begin{monster}
\monstercarac{img}{Pit Viper.jpg}
\monstercarac{name}{Crotale}
\monstercarac{description}{
  Existe dans tous les climats, sauf les plus extrêmes. N'attaque
généralement que s'il est acculé ou surpris.

  Serpent de 1,50 m de long, avec des écailles grises ou vertes.

}
\monsterdetail{Infravision}{Sur 18 m. Les orifices sur la tête lui permettent de ressentir la
chaleur.
}
\monsterdetail{Initiative}{En raison de sens spéciaux, gagne toujours l'initiative (pas de jet).
}
\monsterdetail{Poison}{Provoque la mort (jet de sauvegarde contre le poison).
}
\monstercarac{ca}{6 [13]}
\monstercarac{hd}{2* (9 pv)}
\monstercarac{taco}{18 [+1]}
\monstercarac{moral}{7}
\monstercarac{alignement}{Neutre}
\monstercarac{xp}{25}
\monstercarac{nombre_donjon}{1d8}
\monstercarac{nombre_exterieur}{1d8}
\monstercarac{tresor}{Aucun}
\monstercarac{mvt}{27 m (9 m)}
\monstercarac{save_mort_poison}{12}
\monstercarac{save_baguettes}{13}
\monstercarac{save_paralysie_petrification}{14}
\monstercarac{save_souffles}{15}
\monstercarac{save_sorts_sceptres_batons}{16}
\monsterattack{1 × morsure }{1d4 + poison}
\end{monster}

\begin{monster}
\monstercarac{img}{Rock Python.png}
\monstercarac{name}{Python des rochers}
\monstercarac{description}{
  Existe dans tous les climats, sauf les plus extrêmes. N'attaque
généralement que s'il est acculé ou surpris.

  Serpent de 6 m de long, avec des motifs en spirale marron et jaune sur
ses écailles.

}
\monsterdetail{Constriction}{Lorsqu'une attaque par morsure réussit, le python s'enroule autour de sa
victime et commence à l'enserrer, lui infligeant immédiatement 2d4
dégâts automatiques, puis à chaque round ultérieur.
}
\monstercarac{ca}{6 [13]}
\monstercarac{hd}{5* (22 pv)}
\monstercarac{taco}{15 [+4]}
\monstercarac{moral}{8}
\monstercarac{alignement}{Neutre}
\monstercarac{xp}{300}
\monstercarac{nombre_donjon}{1d3}
\monstercarac{nombre_exterieur}{1d3}
\monstercarac{tresor}{U}
\monstercarac{mvt}{27 m (9 m)}
\monstercarac{save_mort_poison}{12}
\monstercarac{save_baguettes}{13}
\monstercarac{save_paralysie_petrification}{14}
\monstercarac{save_souffles}{15}
\monstercarac{save_sorts_sceptres_batons}{16}
\monsterattack{1 × morsure }{1d4 + constriction}
\end{monster}

\begin{monster}
\monstercarac{name}{Serpent de mer}
\monstercarac{description}{
  Existe dans tous les climats, sauf les plus extrêmes. N'attaque
généralement que s'il est acculé ou surpris.

  Serpent de 1,80 m de long, vivant sous l'eau. Ne remonte pour respirer
qu'une fois par heure et peut attaquer les humains.

}
\monsterdetail{Morsure minuscule}{50 \% de chances de passer inaperçue.
}
\monsterdetail{Poison : Action lente}{les effets sont ressentis après 1d4+2 tours. Jet de sauvegarde contre le
poison ou mort 1 tour plus tard. À ce stade, le sort contre-poison a 25
\% de chances de ne pas fonctionner.
}
\monsterdetail{Grands individus}{Il est possible de rencontrer des serpents de mer avec plus de 3 DV. Ils
mesurent alors 1,80 m de long pour chaque tranche de 3 DV.
}
\monstercarac{ca}{6 [13]}
\monstercarac{hd}{3* (13 pv)}
\monstercarac{taco}{17 [+2]}
\monstercarac{moral}{7}
\monstercarac{alignement}{Neutre}
\monstercarac{xp}{50}
\monstercarac{nombre_donjon}{1d8}
\monstercarac{nombre_exterieur}{1d8}
\monstercarac{tresor}{Aucun}
\monstercarac{mvt}{27 m (9 m)}
\monstercarac{save_mort_poison}{12}
\monstercarac{save_baguettes}{13}
\monstercarac{save_paralysie_petrification}{14}
\monstercarac{save_souffles}{15}
\monstercarac{save_sorts_sceptres_batons}{16}
\monsterattack{1 × morsure }{1 pv + poison}
\end{monster}

\begin{monster}
\monstercarac{img}{Giant Rattler.png}
\monstercarac{name}{Serpent à sonnette géant}
\monstercarac{description}{
  Existe dans tous les climats, sauf les plus extrêmes. N'attaque
généralement que s'il est acculé ou surpris.

  Long serpent de 3 m de long, avec des écailles brunes et blanches en
forme de diamant, et une cascabelle d'écailles râpeuses sur sa queue.

}
\monsterdetail{Cascabelle}{Il la secoue pour avertir les créatures qu'il ne souhaite pas attaquer.
}
\monsterdetail{Poison}{Provoque la mort en 1d6 tours (jet de sauvegarde contre le poison).
}
\monsterdetail{Vitesse}{Attaque deux fois par round. La deuxième attaque a lieu à la fin de
chaque round.
}
\monstercarac{ca}{5 [14]}
\monstercarac{hd}{4* (18 pv)}
\monstercarac{taco}{16 [+3]}
\monstercarac{moral}{8}
\monstercarac{alignement}{Neutre}
\monstercarac{xp}{125}
\monstercarac{nombre_donjon}{1d4}
\monstercarac{nombre_exterieur}{1d4}
\monstercarac{tresor}{U}
\monstercarac{mvt}{36 m (12 m)}
\monstercarac{save_mort_poison}{12}
\monstercarac{save_baguettes}{13}
\monstercarac{save_paralysie_petrification}{14}
\monstercarac{save_souffles}{15}
\monstercarac{save_sorts_sceptres_batons}{16}
\monsterattack{2 × morsure }{1d4 + poison}
\end{monster}

\begin{monster}
\monstercarac{img}{Sea Serpent.jpg}
\monstercarac{name}{Serpent de mer (mineur)}
\monstercarac{description}{
  Ce monstre marin, entre 6 et 9 m de long, ressemble à un serpent avec
des rangées de nombreuses nageoires.

}
\monsterdetail{Bond en avant}{Peut effectuer une attaque en bondissant de l'eau jusqu'à 6 m de hauteur
pour utiliser son attaque de morsure.
}
\monsterdetail{Constriction}{S'enroule autour d'un navire de même taille ou plus petit, et l'écrase.
}
\monstercarac{ca}{5 [14]}
\monstercarac{hd}{6 (27 pv)}
\monstercarac{taco}{14 [+5]}
\monstercarac{moral}{8}
\monstercarac{alignement}{Neutre}
\monstercarac{xp}{275}
\monstercarac{nombre_donjon}{0}
\monstercarac{nombre_exterieur}{2d6}
\monstercarac{tresor}{Aucun}
\monstercarac{mvt}{45 m (15 m)}
\monstercarac{save_mort_poison}{12}
\monstercarac{save_baguettes}{13}
\monstercarac{save_paralysie_petrification}{14}
\monstercarac{save_souffles}{15}
\monstercarac{save_sorts_sceptres_batons}{16}
\monsterattack{1 × morsure }{2d6}
\monsterattack{1 × constriction }{1d10 points de dégâts de structure}
\end{monster}

\begin{monster}
\monstercarac{img}{Spectre.jpg}
\monstercarac{name}{Spectre}
\monstercarac{description}{
  Fantôme incorporel --- l'un des monstres morts-vivants les plus
puissants qui soient.

}
\monsterdetail{Mort-vivant}{Ne fait aucun bruit, jusqu'à ce qu'il attaque. Immunisé contre les
effets affectant les créatures vivantes (par exemple, le poison).
Immunisé contre les sorts affectant ou lisant l'esprit (ex. : charme,
paralysie, sommeil).
}
\monsterdetail{Immunité aux dégâts normaux}{Ne peut être blessé que par des attaques magiques.
}
\monsterdetail{Absorption d’énergie}{Une cible touchée avec succès perd deux niveaux d'expérience (ou Dés de
vie) de manière permanente. Ceci entraîne la perte de points de vie
correspondant aux DV de perdus, ainsi que les autres capacités dues aux
niveaux perdus (sorts, jets de sauvegarde, etc.). L'XP du personnage est
réduite au minimum du nouveau niveau. Une personne qui perd tous ses
niveaux devient un spectre la nuit suivante, sous le contrôle du spectre
qui l'a tuée.
}
\monstercarac{ca}{2 [17]}
\monstercarac{hd}{6** (27 pv)}
\monstercarac{taco}{14 [+5]}
\monstercarac{moral}{11}
\monstercarac{alignement}{Chaotique}
\monstercarac{xp}{725}
\monstercarac{nombre_donjon}{1d4}
\monstercarac{nombre_exterieur}{1d8}
\monstercarac{tresor}{E}
\monstercarac{mvt}{45 m (15 m) / 90 m (30 m) vol}
\monstercarac{save_mort_poison}{10}
\monstercarac{save_baguettes}{11}
\monstercarac{save_paralysie_petrification}{12}
\monstercarac{save_souffles}{13}
\monstercarac{save_sorts_sceptres_batons}{14}
\monsterattack{1 × toucher }{1d8 + absorption d’énergie}
\end{monster}

\begin{monster}
\monstercarac{name}{Squelette}
\monstercarac{description}{
  Restes squelettiques d'humanoïdes, réanimés sous la forme de gardiens
par de puissants magiciens ou clercs. On rencontre souvent les
squelettes dans les cimetières, les cryptes ou autres lieux délaissés.

}
\monsterdetail{Mort-vivant}{Ne fait aucun bruit, jusqu'à ce qu'il attaque. Immunisé contre les
effets affectant les créatures vivantes (par exemple, le poison).
Immunisé contre les sorts affectant ou lisant l'esprit (ex. : charme,
paralysie, sommeil).
}
\monstercarac{ca}{7 [12]}
\monstercarac{hd}{1 (4 pv)}
\monstercarac{taco}{19 [0]}
\monstercarac{moral}{12}
\monstercarac{alignement}{Chaotique}
\monstercarac{xp}{10}
\monstercarac{nombre_donjon}{3d4}
\monstercarac{nombre_exterieur}{3d10}
\monstercarac{tresor}{Aucun}
\monstercarac{mvt}{18 m (6 m)}
\monstercarac{save_mort_poison}{12}
\monstercarac{save_baguettes}{13}
\monstercarac{save_paralysie_petrification}{14}
\monstercarac{save_souffles}{15}
\monstercarac{save_sorts_sceptres_batons}{16}
\monsterattack{1 × arme }{1d6 }
\monsterattack{selon l’arme)}{}
\end{monster}

\begin{monster}
\monstercarac{name}{Statue vivante de cristal}
\monstercarac{description}{
  Statues animées de toutes tailles et matériaux (trois types sont décrits
ci-dessous, mais d'autres peuvent être inventés par l'arbitre).

  Formée de cristaux. Souvent de forme humaine.

}
\monsterdetail{Immunité}{Insensible aux sorts de sommeil.
}
\monsterdetail{Origines magiques}{Créées par de puissants magiciens.
}
\monsterdetail{Immunité}{Insensible aux sorts de sommeil.
}
\monsterdetail{Attaques}{Elles dépendent de la forme de la statue (les statues humanoïdes peuvent
utiliser des armes, les statues animales utilisent les griffes, etc.).
}
\monstercarac{ca}{4 [15]}
\monstercarac{hd}{3 (13 pv)}
\monstercarac{taco}{17 [+2]}
\monstercarac{moral}{11}
\monstercarac{alignement}{Loyal}
\monstercarac{xp}{35}
\monstercarac{nombre_donjon}{1d6}
\monstercarac{nombre_exterieur}{1d6}
\monstercarac{tresor}{Aucun}
\monstercarac{save_mort_poison}{12}
\monstercarac{save_baguettes}{13}
\monstercarac{save_paralysie_petrification}{14}
\monstercarac{save_souffles}{15}
\monstercarac{save_sorts_sceptres_batons}{16}
\monsterattack{2 × coup }{1d6)}
\end{monster}

\begin{monster}
\monstercarac{name}{Statue vivante de fer}
\monstercarac{description}{
  Statues animées de toutes tailles et matériaux (trois types sont décrits
ci-dessous, mais d'autres peuvent être inventés par l'arbitre).

}
\monsterdetail{Immunité}{Insensible aux sorts de sommeil.
}
\monsterdetail{Origines magiques}{Créées par de puissants magiciens.
}
\monsterdetail{Immunité}{Insensible aux sorts de sommeil.
}
\monsterdetail{Absorption des métaux}{Les coups donnés avec des armes en métal non magiques infligent leurs
dégâts, mais elles peuvent rester coincées dans la statue (jet de
sauvegarde contre les sorts). Les armes coincées peuvent être retirées
si la statue est tuée.
}
\monsterdetail{Attaques}{Dépendent de la forme de la statue (les statues humanoïdes peuvent
utiliser des armes, les statues animales utilisent les griffes, etc.).
}
\monstercarac{ca}{2 [17]}
\monstercarac{hd}{4 (18 pv)}
\monstercarac{taco}{16 [+3]}
\monstercarac{moral}{11}
\monstercarac{alignement}{Neutre}
\monstercarac{xp}{75}
\monstercarac{nombre_donjon}{1d4}
\monstercarac{nombre_exterieur}{1d4}
\monstercarac{tresor}{Aucun}
\monstercarac{mvt}{9 m (3 m)}
\monstercarac{save_mort_poison}{10}
\monstercarac{save_baguettes}{11}
\monstercarac{save_paralysie_petrification}{12}
\monstercarac{save_souffles}{13}
\monstercarac{save_sorts_sceptres_batons}{14}
\monsterattack{2 × coup }{1d8}
\end{monster}

\begin{monster}
\monstercarac{name}{Statue vivante de pierre}
\monstercarac{description}{
  Statues animées de toutes tailles et matériaux (trois types sont décrits
ci-dessous, mais d'autres peuvent être inventés par l'arbitre).

  Sous sa surface pierreuse, elle est remplie de magma qu'elle peut
projeter du bout de ses doigts.

}
\monsterdetail{Immunité}{Insensible aux sorts de sommeil.
}
\monsterdetail{Origines magiques}{Créées par de puissants magiciens.
}
\monsterdetail{Immunité}{Insensible aux sorts de sommeil.
}
\monstercarac{ca}{4 [15]}
\monstercarac{hd}{5** (22 pv)}
\monstercarac{taco}{15 [+4]}
\monstercarac{moral}{11}
\monstercarac{alignement}{Chaotique}
\monstercarac{xp}{425}
\monstercarac{nombre_donjon}{1d3}
\monstercarac{nombre_exterieur}{1d3}
\monstercarac{tresor}{Aucun}
\monstercarac{mvt}{18 m (6 m)}
\monstercarac{save_mort_poison}{10}
\monstercarac{save_baguettes}{11}
\monstercarac{save_paralysie_petrification}{12}
\monstercarac{save_souffles}{13}
\monstercarac{save_sorts_sceptres_batons}{14}
\monsterattack{2 × jet de magma }{2d6}
\end{monster}

\begin{monster}
\monstercarac{name}{Strige}
\monstercarac{description}{
  Créature à plumes rappelant des oiseaux, dont le bec est long et acéré.

}
\monsterdetail{Attaque en piqué}{La première attaque se fait avec un bonus de +2 pour toucher.
}
\monsterdetail{Absorption de sang}{Après une attaque réussie, la strige se fixe sur sa cible afin de lui
boire le sang : 1d3 points de dégâts automatiques par round.
}
\monsterdetail{Détachement}{Uniquement à la mort de la strige ou de la victime.
}
\monstercarac{ca}{7 [12]}
\monstercarac{hd}{1* (4 pv)}
\monstercarac{taco}{19 [0]}
\monstercarac{moral}{9}
\monstercarac{alignement}{Neutre}
\monstercarac{xp}{13}
\monstercarac{nombre_donjon}{1d10}
\monstercarac{nombre_exterieur}{3d12}
\monstercarac{tresor}{L}
\monstercarac{save_mort_poison}{12}
\monstercarac{save_baguettes}{13}
\monstercarac{save_paralysie_petrification}{14}
\monstercarac{save_souffles}{15}
\monstercarac{save_sorts_sceptres_batons}{16}
\monsterattack{1 × bec }{1d3 + absorption de sang)}
\end{monster}

\input{liste/Stégosaure.tex}
\begin{monster}
\monstercarac{name}{Sylvanien}
\monstercarac{description}{
  Cet humanoïde géant, haut de 5,40 m, ressemble à un arbre. Habite dans
les forêts et ne s'occupe que de protéger les plantes de son lieu de
vie. Parle sa propre langue tortueuse et bavarde.

}
\monsterdetail{Peur du feu}{Et de ceux qui le manient.
}
\monsterdetail{Surprise}{Sur un 1-à-3, dans une forêt, grâce à sa capacité à être confondu avec
un arbre. La rencontre se produit à 30 m ou moins.
}
\monstercarac{ca}{2 [17]}
\monstercarac{hd}{8 (36 pv)}
\monstercarac{taco}{12 [+7]}
\monstercarac{moral}{9}
\monstercarac{alignement}{Loyal}
\monstercarac{xp}{650}
\monstercarac{nombre_donjon}{0}
\monstercarac{nombre_exterieur}{1d8}
\monstercarac{tresor}{C}
\monstercarac{save_mort_poison}{8}
\monstercarac{save_baguettes}{9}
\monstercarac{save_paralysie_petrification}{10}
\monstercarac{save_souffles}{10}
\monstercarac{save_sorts_sceptres_batons}{12}
\monsterattack{2 × poing }{2d6)}
\end{monster}

\input{liste/Termite_d’eau_douce.tex}
\begin{monster}
\monstercarac{img}{placeholder.jpg}
\monstercarac{name}{Termite d’eau salée}
\monstercarac{description}{
  Insecte aquatique géant, faisant entre 30 cm à 1,50 m de longueur, qui
se nourrit de bois et muni d'une poche leur permettant d'absorber et de
projeter de l'eau. Aime se nourrir des navires qui passent, et
n'attaquent des créatures que s'ils sont acculés.

}
\monsterdetail{Projection irritante}{Utilisable en surface ; maximum une fois par tour. Peut projeter sur une
cible, qui doit effectuer un jet de sauvegarde contre le poison ou être
étourdie pendant 1 tour.
}
\monsterdetail{Nuage d’encre}{Utilisable sous l'eau et au maximum une fois par tour. En s'échappant,
le termite aquatique peut émettre un nuage d'encre noire afin de
troubler ses ennemis.
}
\monsterdetail{Mange les navires}{Elles s'accrochent sous la coque. Chaque individu inflige 1d3 points de
dégâts de structure, puis se décolle.
}
\monsterdetail{Remarquer les dégâts sur un navire}{50 \% de chances par round de détecter des voies d'eau.
}
\monstercarac{ca}{5 [14]}
\monstercarac{hd}{4 (18 pv)}
\monstercarac{taco}{16 [+3]}
\monstercarac{moral}{11}
\monstercarac{alignement}{Neutre}
\monstercarac{xp}{75}
\monstercarac{nombre_donjon}{0}
\monstercarac{nombre_exterieur}{1d6+1}
\monstercarac{tresor}{Aucun}
\monstercarac{mvt}{54 m (18 m)}
\monstercarac{save_mort_poison}{12}
\monstercarac{save_baguettes}{13}
\monstercarac{save_paralysie_petrification}{14}
\monstercarac{save_souffles}{15}
\monstercarac{save_sorts_sceptres_batons}{16}
\monsterattack{1 × morsure }{1d6}
\end{monster}

\begin{monster}
\monstercarac{img}{placeholder.jpg}
\monstercarac{name}{Termite des marais}
\monstercarac{description}{
  Insecte aquatique géant, faisant entre 30 cm à 1,50 m de longueur, qui
se nourrit de bois et muni d'une poche leur permettant d'absorber et de
projeter de l'eau. Aime se nourrir des navires qui passent, et
n'attaquent des créatures que s'ils sont acculés.

}
\monsterdetail{Projection irritante}{Utilisable en surface ; maximum une fois par tour. Peut projeter sur une
cible, qui doit effectuer un jet de sauvegarde contre le poison ou être
étourdie pendant 1 tour.
}
\monsterdetail{Nuage d’encre}{Utilisable sous l'eau et au maximum une fois par tour. En s'échappant,
le termite aquatique peut émettre un nuage d'encre noire afin de
troubler ses ennemis.
}
\monsterdetail{Mange les navires}{Elles s'accrochent sous la coque. Chaque individu inflige 1d3 points de
dégâts de structure, puis se décolle.
}
\monsterdetail{Remarquer les dégâts sur un navire}{50 \% de chances par round de détecter des voies d'eau.
}
\monstercarac{ca}{4 [15]}
\monstercarac{hd}{1+1 (5 pv)}
\monstercarac{taco}{18 [+1]}
\monstercarac{moral}{10}
\monstercarac{alignement}{Neutre}
\monstercarac{xp}{15}
\monstercarac{nombre_donjon}{0}
\monstercarac{nombre_exterieur}{1d4}
\monstercarac{tresor}{Aucun}
\monstercarac{mvt}{27 m (9 m)}
\monstercarac{save_mort_poison}{12}
\monstercarac{save_baguettes}{13}
\monstercarac{save_paralysie_petrification}{14}
\monstercarac{save_souffles}{15}
\monstercarac{save_sorts_sceptres_batons}{16}
\monsterattack{1 × morsure }{1d3}
\end{monster}

\begin{monster}
\monstercarac{name}{Thoul}
\monstercarac{description}{
  Abomination magique qui ressemble à un hobgobelin (sauf en cas d'examen
attentif), mais possède en même temps des capacités héritées des trolls
et des goules. Vivent parfois avec les hobgobelins.

}
\monsterdetail{Paralysie}{Une attaque réussie paralyse la victime pour 2d4 tours (un jet de
sauvegarde contre la paralysie annule l'effet). Les elfes et les
créatures plus grandes qu'un ogre sont immunisés. Après avoir paralysé
un adversaire, un thoul passe à un autre.
}
\monsterdetail{Régénération}{Un thoul blessé récupère un point de vie au début de chaque round aussi
longtemps qu'il est en vie.
}
\monstercarac{ca}{6 [13]}
\monstercarac{hd}{3** (13 pv)}
\monstercarac{taco}{17 [+2]}
\monstercarac{moral}{10}
\monstercarac{alignement}{Chaotique}
\monstercarac{xp}{65}
\monstercarac{nombre_donjon}{1d6}
\monstercarac{nombre_exterieur}{1d10}
\monstercarac{tresor}{C}
\monstercarac{save_mort_poison}{12}
\monstercarac{save_baguettes}{13}
\monstercarac{save_paralysie_petrification}{14}
\monstercarac{save_souffles}{15}
\monstercarac{save_sorts_sceptres_batons}{16}
\monsterattack{2 × griffes }{1d3 + paralysie) ou 1 × arme}
\end{monster}

\begin{monster}
\monstercarac{name}{Tinigens (Monstre)}
\monstercarac{description}{
  Ces semi-humains de petite taille et aux pieds velus habitent dans de
petits villages (entre 30 et 300 habitants).

}
\monsterdetail{Chef et milice}{Les villages sont dirigés par un tinigens de niveau 1d6+1. Une milice
est composée de 5d4 gardes ayant chacun 2 DV.
}
\monsterdetail{Trésor}{Ne possède qu'un trésor de type B que lorsqu'on les rencontre dans les
contrées sauvages.
}
\monstercarac{ca}{7 [12]}
\monstercarac{hd}{1–1 (3 pv)}
\monstercarac{taco}{19 [0]}
\monstercarac{moral}{7}
\monstercarac{alignement}{Loyal}
\monstercarac{xp}{5 (guard: 20)}
\monstercarac{nombre_donjon}{3d6}
\monstercarac{nombre_exterieur}{5d8}
\monstercarac{tresor}{V (B)}
\monstercarac{mvt}{27 m (9 m)}
\monstercarac{save_mort_poison}{8}
\monstercarac{save_baguettes}{9}
\monstercarac{save_paralysie_petrification}{10}
\monstercarac{save_souffles}{13}
\monstercarac{save_sorts_sceptres_batons}{12}
\monsterattack{1 × arme }{1d6 }
\monsterattack{selon l’arme)}{}
\end{monster}

\input{liste/Titanothère.tex}
\input{liste/Tricératops.tex}
\begin{monster}
\monstercarac{name}{Troglodyte}
\monstercarac{description}{
  Reptile humanoïde intelligent, reconnaissable à ses mains agiles, ses
longues jambes et sa courte queue, et doté également d'une crête
d'épines sur la tête et les bras.

}
\monsterdetail{Haineux}{Essaie de tuer toute créature qu'il rencontre.
}
\monsterdetail{Surprise}{Sur un 1-à-4, dû à sa capacité de changer de couleur afin de
correspondre à son environnement. Aime se dissimuler le long des parois
de pierre en attendant leurs victimes.
}
\monsterdetail{Odeur nauséabonde}{L'odeur des huiles sécrétées par sa peau rend malades les humains et les
semi-humains : un jet de sauvegarde contre le poison raté entraîne un
malus de --2 pour toucher lorsque le PJ combat en mêlée avec des
troglodytes.
}
\monstercarac{ca}{5 [14]}
\monstercarac{hd}{2* (9 pv)}
\monstercarac{taco}{18 [+1]}
\monstercarac{moral}{9}
\monstercarac{alignement}{Chaotique}
\monstercarac{xp}{25}
\monstercarac{nombre_donjon}{1d8}
\monstercarac{nombre_exterieur}{5d8}
\monstercarac{tresor}{A}
\monstercarac{mvt}{36 m (12 m)}
\monstercarac{save_mort_poison}{12}
\monstercarac{save_baguettes}{13}
\monstercarac{save_paralysie_petrification}{14}
\monstercarac{save_souffles}{15}
\monstercarac{save_sorts_sceptres_batons}{16}
\monsterattack{2 × griffes }{1d4}
\monsterattack{1 × morsure }{1d4}
\end{monster}

\begin{monster}
\monstercarac{name}{Troll}
\monstercarac{description}{
  Humanoïde tout à la fois intelligent et méchant, faisant 2,40 m de haut,
avec un corps maigre et caoutchouteux. Se nourrit de la chair d'autres
humanoïdes et vit sous terre, dans les contrées sauvages et les maisons
en ruines ayant appartenu à d'anciennes victimes.

}
\monsterdetail{Régénération}{3 rounds après avoir été blessé, le troll regagne 3 pv par round. Cette
capacité peut lui permettre de rattacher des membres sectionnés.
}
\monsterdetail{Retour à la vie}{S'il est tué (0 pv), il se régénère et combat de nouveau en 2d6 rounds.
}
\monsterdetail{Feu et acide}{Ne peut pas régénérer les dégâts provenant de ces sources. C'est la
seule façon de tuer définitivement un troll.
}
\monsterdetail{Peur du feu}{Moral 8 en cas d'attaque par le feu ou l'acide.
}
\monstercarac{ca}{4 [15]}
\monstercarac{hd}{6+3* (30 pv)}
\monstercarac{taco}{13 [+6]}
\monstercarac{moral}{10 (8 peur du feu)}
\monstercarac{alignement}{Chaotique}
\monstercarac{xp}{650}
\monstercarac{nombre_donjon}{1d8}
\monstercarac{nombre_exterieur}{1d8}
\monstercarac{tresor}{D}
\monstercarac{mvt}{36 m (12 m)}
\monstercarac{save_mort_poison}{10}
\monstercarac{save_baguettes}{11}
\monstercarac{save_paralysie_petrification}{12}
\monstercarac{save_souffles}{13}
\monstercarac{save_sorts_sceptres_batons}{14}
\monsterattack{2 × griffes }{1d6}
\monsterattack{1 × morsure }{1d10}
\end{monster}

\begin{monster}
\monstercarac{img}{placeholder.jpg}
\monstercarac{name}{Petit troupeau}
\monstercarac{description}{
  Ces animaux sauvages vivent dans les pâturages en grands troupeaux. Le
type exact dépend du terrain.

  Par exemple : antilope, cerf, chèvre.

}
\monsterdetail{Débandade}{Les troupeaux de 20 individus ou plus peuvent piétiner ceux qui se
trouvent sur leur passage. 3-sur-4 chances à chaque round. Bonus de +4
pour toucher les créatures de taille humaine ou plus petites. 1d20
points de dégâts.
}
\monsterdetail{Mâles}{Dans les groupes de 3 individus ou plus, un quart seulement sont des
mâles. Ils ont 1d4 points de vie supplémentaires et protègent le
troupeau.
}
\monsterdetail{Femelles et petits}{Ils fuient le danger. Les femelles n'ont pas d'attaque de percussion.
Les petits n'ont que la moitié des points de vie.
}
\monsterdetail{Piétinement}{Les troupeaux de 20 individus ou plus peuvent piétiner ceux qui se
trouvent sur leur passage. 3-sur-4 chances à chaque round. Bonus de +4
pour toucher les créatures de taille humaine ou plus petites. 1d20
points de dégâts.
}
\monstercarac{ca}{7 [12]}
\monstercarac{hd}{1 à 2 (4/9 pv)}
\monstercarac{taco}{19 [0]/18 [+1]}
\monstercarac{moral}{5}
\monstercarac{alignement}{Neutre}
\monstercarac{xp}{10/20}
\monstercarac{nombre_donjon}{0}
\monstercarac{nombre_exterieur}{3d10}
\monstercarac{tresor}{Aucun}
\monstercarac{mvt}{72 m (24 m)}
\monstercarac{save_mort_poison}{12}
\monstercarac{save_baguettes}{13}
\monstercarac{save_paralysie_petrification}{14}
\monstercarac{save_souffles}{15}
\monstercarac{save_sorts_sceptres_batons}{16}
\monsterattack{1 × percussion }{1d4}
\end{monster}

\begin{monster}
\monstercarac{name}{Troupeau moyen}
\monstercarac{description}{
  Ces animaux sauvages vivent dans les pâturages en grands troupeaux. Le
type exact dépend du terrain.

  Par exemple : caribou ou bovidé.

}
\monsterdetail{Débandade}{Les troupeaux de 20 individus ou plus peuvent piétiner ceux qui se
trouvent sur leur passage. 3-sur-4 chances à chaque round. Bonus de +4
pour toucher les créatures de taille humaine ou plus petites. 1d20
points de dégâts.
}
\monsterdetail{Mâles}{Dans les groupes de 3 individus ou plus, un quart seulement sont des
mâles. Ils ont 1d4 points de vie supplémentaires et protègent le
troupeau.
}
\monsterdetail{Femelles et petits}{Ils fuient le danger. Les femelles n'ont pas d'attaque de percussion.
Les petits n'ont que la moitié des points de vie.
}
\monsterdetail{Piétinement}{Les troupeaux de 20 individus ou plus peuvent piétiner ceux qui se
trouvent sur leur passage. 3-sur-4 chances à chaque round. Bonus de +4
pour toucher les créatures de taille humaine ou plus petites. 1d20
points de dégâts.
}
\monstercarac{ca}{7 [12]}
\monstercarac{hd}{3 (13 pv)}
\monstercarac{taco}{17 [+2]}
\monstercarac{moral}{5}
\monstercarac{alignement}{Neutre}
\monstercarac{xp}{35}
\monstercarac{nombre_donjon}{0}
\monstercarac{nombre_exterieur}{3d10}
\monstercarac{tresor}{Aucun}
\monstercarac{mvt}{72 m (24 m)}
\monstercarac{save_mort_poison}{12}
\monstercarac{save_baguettes}{13}
\monstercarac{save_paralysie_petrification}{14}
\monstercarac{save_souffles}{15}
\monstercarac{save_sorts_sceptres_batons}{16}
\monsterattack{1 × percussion }{1d6}
\end{monster}

\begin{monster}
\monstercarac{name}{Grand troupeau}
\monstercarac{description}{
  Ces animaux sauvages vivent dans les pâturages en grands troupeaux. Le
type exact dépend du terrain.

  Par exemple : wapiti ou élan.

}
\monsterdetail{Débandade}{Les troupeaux de 20 individus ou plus peuvent piétiner ceux qui se
trouvent sur leur passage. 3-sur-4 chances à chaque round. Bonus de +4
pour toucher les créatures de taille humaine ou plus petites. 1d20
points de dégâts.
}
\monsterdetail{Mâles}{Dans les groupes de 3 individus ou plus, un quart seulement sont des
mâles. Ils ont 1d4 points de vie supplémentaires et protègent le
troupeau.
}
\monsterdetail{Femelles et petits}{Ils fuient le danger. Les femelles n'ont pas d'attaque de percussion.
Les petits n'ont que la moitié des points de vie.
}
\monsterdetail{Piétinement}{Les troupeaux de 20 individus ou plus peuvent piétiner ceux qui se
trouvent sur leur passage. 3-sur-4 chances à chaque round. Bonus de +4
pour toucher les créatures de taille humaine ou plus petites. 1d20
points de dégâts.
}
\monstercarac{ca}{7 [12]}
\monstercarac{hd}{4 (18 pv)}
\monstercarac{taco}{16 [+3]}
\monstercarac{moral}{5}
\monstercarac{alignement}{Neutre}
\monstercarac{xp}{75}
\monstercarac{nombre_donjon}{0}
\monstercarac{nombre_exterieur}{3d10}
\monstercarac{tresor}{Aucun}
\monstercarac{mvt}{72 m (24 m)}
\monstercarac{save_mort_poison}{12}
\monstercarac{save_baguettes}{13}
\monstercarac{save_paralysie_petrification}{14}
\monstercarac{save_souffles}{15}
\monstercarac{save_sorts_sceptres_batons}{16}
\monsterattack{1 × percussion }{1d8}
\end{monster}

\begin{monster}
\monstercarac{img}{Tyrannosaurus Rex.png}
\monstercarac{name}{Tyrannosaurus Rex}
\monstercarac{description}{
  Grand dinosaure prédateur à deux pattes et plus de 6 m de hauteur, avec
d'énormes mâchoires. Chasse des proies de taille humaine ou plus grande.
Parcourt les régions du Monde perdu.

}
\monsterdetail{Grosse proie}{Attaque d'abord la plus grande cible.
}
\monstercarac{ca}{3 [16]}
\monstercarac{hd}{20 (90 pv)}
\monstercarac{taco}{6 [+13]}
\monstercarac{moral}{11}
\monstercarac{alignement}{Neutre}
\monstercarac{xp}{2 000}
\monstercarac{nombre_donjon}{0}
\monstercarac{nombre_exterieur}{1}
\monstercarac{tresor}{V × 3}
\monstercarac{mvt}{36 m (12 m)}
\monstercarac{save_mort_poison}{6}
\monstercarac{save_baguettes}{7}
\monstercarac{save_paralysie_petrification}{8}
\monstercarac{save_souffles}{8}
\monstercarac{save_sorts_sceptres_batons}{10}
\monsterattack{1 × morsure }{6d6}
\end{monster}

\begin{monster}
\monstercarac{name}{Vampire}
\monstercarac{description}{
  Monstre mort-vivant très redouté qui vit en buvant le sang des mortels.
Habite dans les ruines, les tombes et les lieux désertés.

}
\monsterdetail{Mort-vivant}{Ne fait aucun bruit jusqu'à ce qu'il attaque. Immunisé contre les effets
affectant les créatures vivantes (par exemple, le poison). Immunisé
contre les sorts affectant ou lisant l'esprit (par exemple, charme,
paralysie, sommeil).
}
\monsterdetail{Immunité aux dégâts normaux}{Ne peut être blessé que par des attaques magiques.
}
\monsterdetail{Absorption d’énergie}{Une cible touchée avec succès perd deux niveaux d'expérience (ou Dés de
vie) de manière permanente. Ceci entraîne la perte de points de vie
correspondant aux DV de perdus, ainsi que les autres capacités dues aux
niveaux perdus (sorts, jets de sauvegarde, etc.). L'XP du personnage est
réduite au minimum du nouveau niveau. Une personne qui perd tous ses
niveaux devient un vampire en 3 jours.
}
\monsterdetail{Regard (charme)}{Quiconque croise le regard du vampire doit effectuer un jet de
sauvegarde contre les sorts avec un malus de --2. En cas d'échec, la
victime est charmée et doit : se déplacer en direction du vampire (en
résistant à ceux qui essaient de l'en empêcher) ; le défendre ; et obéir
à ses ordres (à condition de les comprendre). Incapable de lancer de
sorts, ou d'utiliser d'objets magiques, ou s'en prendre au vampire. La
mort du vampire rompt le charme.
}
\monsterdetail{Régénération}{Un vampire ayant subi des dégâts regagne 3 pv au début de chaque round,
tant qu'il est vivant.
}
\monsterdetail{À 0 pv}{Il passe en forme gazeuse et fuit en direction de son cercueil.
}
\monsterdetail{Changeur de forme}{À volonté ; prend 1 round :

\begin{enumerate}
\def\labelenumi{\arabic{enumi}.}
\tightlist
\item
  \textbf{Humanoïde :} Forme standard.
\item
  \textbf{Loup géant :} Att 1 × morsure (2d4), DP 45 m (15 m), CA, DV,
  Ml : Comme ceux du vampire.
\item
  \textbf{Chauve-souris géante :} Att 1 × morsure (1d4), DP 9 m (3 m) /
  54 m (18 m) vol, CA, DV, Ml : Comme ceux du vampire.
\item
  \textbf{Nuage gazeux :} DP 54 m (18 m) vol.~Immunisé aux dégâts
  normaux. Ne peut pas attaquer.
\end{enumerate}
}
\monsterdetail{Convocation de bêtes}{Sous sa forme humaine uniquement, le vampire peut faire venir à lui des
créatures des environs :

\begin{itemize}
\tightlist
\item
  1d10 × 10 \href{Rat.md\#Rat-normal}{rats},
\item
  5d4 \href{Rat.md\#Rat-géant}{rats géants},
\item
  1d10 × 10
  \href{Chauve-souris.md\#Chauve-souris-normale}{chauves-souris},
\item
  3d6 \href{Chauve-souris.md\#Chauve-souris-géante}{chauves-souris
  géantes},
\item
  3d6 \href{Loup.md\#Loup-normal}{loups communs}, ou
\item
  2d4 \href{Loup.md\#Loup-géant}{loups géants}.
\end{itemize}
}
\monsterdetail{Cercueils}{Doit se reposer durant la journée dans un cercueil ou perdre 2d6 pv
(uniquement régénérés en se reposant une journée complète). Ne peut pas
se reposer dans un cercueil béni. Garde toujours plusieurs cercueils
dans des endroits cachés.
}
\monsterdetail{Vulnérabilités}{\begin{enumerate}
\def\labelenumi{\arabic{enumi}.}
\tightlist
\item
  \textbf{Ail :} L'odeur le repousse : jet de sauvegarde contre le
  poison ou incapacité à attaquer ce round.
\item
  \textbf{Symboles sacrés :} S'ils sont présentés, ils gardent le
  vampire à distance (3 m). Peut attaquer le porteur à partir d'une
  autre direction.
\item
  \textbf{Eau vive :} Ne peut la traverser (sous aucune forme), sauf au
  moyen d'un pont ou s'il est transporté à l'intérieur d'un cercueil.
\item
  \textbf{Miroirs :} Les évite ; on n'y voit pas son reflet.
\item
  \textbf{Lumière éternelle :} Partiellement aveuglé par la lumière de
  ce sort (malus de --4 aux attaques).
\end{enumerate}
}
\monsterdetail{Destruction}{\begin{enumerate}
\def\labelenumi{\arabic{enumi}.}
\tightlist
\item
  \textbf{Lumière du soleil :} Doit effectuer un jet de sauvegarde
  contre la mort à chaque round où il y est exposé ou est désintégré.
\item
  \textbf{Pieu dans le cœur :} Tue le vampire définitivement.
\item
  \textbf{Immersion dans l'eau :} Pendant 1 tour, le tue définitivement.
\item
  \textbf{Détruire les cercueils :} Le vampire est tué de façon
  permanente si tous ses points de vie sont perdus et qu'il est
  incapable de se reposer (voir cercueils).
\end{enumerate}
}
\monstercarac{ca}{2 [17]}
\monstercarac{hd}{7 à 9** (31/36/40 pv)}
\monstercarac{taco}{13[+6] / 12[+7] / 12[+7]}
\monstercarac{moral}{11}
\monstercarac{alignement}{Chaotique}
\monstercarac{xp}{1 250 / 1 750 / 2 300}
\monstercarac{nombre_donjon}{1d4}
\monstercarac{nombre_exterieur}{1d6}
\monstercarac{tresor}{F}
\monstercarac{mvt}{36 m (12 m)}
\monstercarac{save_mort_poison}{8}
\monstercarac{save_baguettes}{9}
\monstercarac{save_paralysie_petrification}{10}
\monstercarac{save_souffles}{10}
\monstercarac{save_sorts_sceptres_batons}{12}
\monsterattack{1 × toucher }{1d10 + absorption d’énergie}
\monsterattack{1 × regard }{charme}
\end{monster}

\begin{monster}
\monstercarac{name}{Vase grise}
\monstercarac{description}{
  Horreur visqueuse qui se cache sur les surfaces en pierre ou parmi les
rochers.

}
\monsterdetail{Se confond avec la pierre}{Difficile de la distinguer de la pierre humide.
}
\monsterdetail{Acide}{Après une attaque réussie, elle se colle à la victime et exsude de
l'acide, qui détruit immédiatement toute armure normale et inflige 2d8
dégâts par round. Une armure magique est dissoute en un tour.
}
\monsterdetail{Immunité naturelle}{Immunisée au froid ou au feu.
}
\monstercarac{ca}{8 [11]}
\monstercarac{hd}{3* (13 pv)}
\monstercarac{taco}{17 [+2]}
\monstercarac{moral}{12}
\monstercarac{alignement}{Neutre}
\monstercarac{xp}{50}
\monstercarac{nombre_donjon}{1}
\monstercarac{nombre_exterieur}{1}
\monstercarac{tresor}{Aucun}
\monstercarac{mvt}{3 m (1 m)}
\monstercarac{save_mort_poison}{12}
\monstercarac{save_baguettes}{13}
\monstercarac{save_paralysie_petrification}{14}
\monstercarac{save_souffles}{15}
\monstercarac{save_sorts_sceptres_batons}{16}
\monsterattack{1 × toucher }{2d8}
\end{monster}

\begin{monster}
\monstercarac{name}{Ver charognard}
\monstercarac{description}{
  Ver segmenté long de 3 m et haut de 1 m, muni de plusieurs pattes, avec
des tentacules longs de 60 cm autour de la bouche.

}
\monsterdetail{Paralysie}{L'attaque réussie d'un tentacule provoque une paralysie pour 2d4 tours
(jet de sauvegarde contre la paralysie). Les victimes paralysées sont
dévorées si le charognard est laissé tranquille.
}
\monsterdetail{Collant}{Peut marcher sur les murs et les plafonds.
}
\monstercarac{ca}{7 [12]}
\monstercarac{hd}{3+1* (14 pv)}
\monstercarac{taco}{16 [+3]}
\monstercarac{moral}{9}
\monstercarac{alignement}{Neutre}
\monstercarac{xp}{75}
\monstercarac{nombre_donjon}{1d3}
\monstercarac{nombre_exterieur}{1d3}
\monstercarac{tresor}{B}
\monstercarac{mvt}{36 m (12 m)}
\monstercarac{save_mort_poison}{12}
\monstercarac{save_baguettes}{13}
\monstercarac{save_paralysie_petrification}{14}
\monstercarac{save_souffles}{15}
\monstercarac{save_sorts_sceptres_batons}{16}
\monsterattack{8 × tentacule }{paralysie}
\end{monster}

\begin{monster}
\monstercarac{name}{Ver violet}
\monstercarac{description}{
  Ver gigantesque et visqueux, faisant 30 m de long et un diamètre entre
2,40 m et 3 m. Creuse des tunnels sous terre et émerge à la surface pour
manger d'autres créatures.

}
\monsterdetail{Gobe entièrement}{Sur un jet d'attaque de morsure de 20, ou une marge de 4 ou plus que le
nombre requis, la cible est avalée si elle est de taille humaine ou plus
petite. Une fois gobée, la cible subit 3d6 dégâts par round (jusqu'à ce
que le ver meure) ; la cible peut attaquer avec des armes tranchantes en
subissant un malus de --4 ; le cadavre est digéré en 6 tours après la
mort.
}
\monsterdetail{Poison}{Provoque la mort (jet de sauvegarde contre le poison).
}
\monsterdetail{Dans les espaces restreints}{Ne peut pas toujours mordre et piquer en même temps.
}
\monstercarac{ca}{6 [13]}
\monstercarac{hd}{15* (67 pv)}
\monstercarac{taco}{9 [+10]}
\monstercarac{moral}{10}
\monstercarac{alignement}{Neutre}
\monstercarac{xp}{2 300}
\monstercarac{nombre_donjon}{1d2}
\monstercarac{nombre_exterieur}{1d4}
\monstercarac{tresor}{D}
\monstercarac{save_mort_poison}{8}
\monstercarac{save_baguettes}{9}
\monstercarac{save_paralysie_petrification}{10}
\monstercarac{save_souffles}{10}
\monstercarac{save_sorts_sceptres_batons}{12}
\monsterattack{1 × morsure }{2d8)}
\monsterattack{1 × dard }{1d8 + poison)}
\end{monster}

\begin{monster}
\monstercarac{name}{Vétéran}
\monstercarac{description}{
  Guerrier de bas niveau, souvent en route vers une guerre ou en revenant.

}
\monsterdetail{Niveau et alignement}{Un groupe peut être composé d'individus du même niveau et du même
alignement, ou ceux-ci peuvent être déterminés au hasard, par individu.
}
\monstercarac{ca}{2 [17]}
\monstercarac{hd}{1 à 3 (4/9/13 pv)}
\monstercarac{taco}{19 [0]}
\monstercarac{moral}{9}
\monstercarac{alignement}{Tous}
\monstercarac{xp}{10/20/35}
\monstercarac{nombre_donjon}{2d4}
\monstercarac{nombre_exterieur}{2d6}
\monstercarac{tresor}{V}
\monstercarac{save_mort_poison}{12}
\monstercarac{save_baguettes}{13}
\monstercarac{save_paralysie_petrification}{14}
\monstercarac{save_souffles}{15}
\monstercarac{save_sorts_sceptres_batons}{16}
\monsterattack{1 × arme }{1d8 ou selon l’arme)}
\end{monster}

\begin{monster}
\monstercarac{name}{Wiverne}
\monstercarac{description}{
  Ce monstre ailé, à deux pattes, ressemble à un dragon, avec une longue
queue dotée d'un aiguillon venimeux. Habite dans n'importe quel type de
terrain, mais privilégie les falaises et les forêts.

}
\monsterdetail{Poison}{Provoque la mort (jet de sauvegarde contre le poison).
}
\monstercarac{ca}{3 [16]}
\monstercarac{hd}{7* (31 pv)}
\monstercarac{taco}{13 [+6]}
\monstercarac{moral}{9}
\monstercarac{alignement}{Chaotique}
\monstercarac{xp}{850}
\monstercarac{nombre_donjon}{1d2}
\monstercarac{nombre_exterieur}{1d6}
\monstercarac{tresor}{E}
\monstercarac{save_mort_poison}{10}
\monstercarac{save_baguettes}{11}
\monstercarac{save_paralysie_petrification}{12}
\monstercarac{save_souffles}{13}
\monstercarac{save_sorts_sceptres_batons}{14}
\monsterattack{1 × morsure }{2d8)}
\monsterattack{1 × dard }{1d6 + poison)}
\end{monster}

\begin{monster}
\monstercarac{img}{Zombie.jpg}
\monstercarac{name}{Zombie}
\monstercarac{description}{
  Cadavre d'humanoïde apathique ayant été réanimé par un magicien ou un
clerc expérimenté pour faire office de gardien.

}
\monsterdetail{Gardien}{Attaque toujours à vue.
}
\monsterdetail{Initiative}{Perd toujours l'initiative (pas de jet nécessaire).
}
\monsterdetail{Mort-vivant}{Ne fait aucun bruit jusqu'à ce qu'il attaque. Immunisé contre les effets
affectant les créatures vivantes (par exemple, le poison). Immunisé
contre les sorts affectant ou lisant l'esprit (par exemple, charme,
paralysie, sommeil).
}
\monstercarac{ca}{8 [11]}
\monstercarac{hd}{2 (9 pv)}
\monstercarac{taco}{18 [+1]}
\monstercarac{moral}{12}
\monstercarac{alignement}{Chaotique}
\monstercarac{xp}{20}
\monstercarac{nombre_donjon}{2d4}
\monstercarac{nombre_exterieur}{4d6}
\monstercarac{tresor}{Aucun}
\monstercarac{mvt}{18 m (9 m)}
\monstercarac{save_mort_poison}{12}
\monstercarac{save_baguettes}{13}
\monstercarac{save_paralysie_petrification}{14}
\monstercarac{save_souffles}{15}
\monstercarac{save_sorts_sceptres_batons}{16}
\monsterattack{1 × arme }{1d8 selon l’arme}
\end{monster}


%\begin{monster}
\monstercarac{img}{Killer_Bee.jpg}
\monstercarac{name}{Abeille tueuse}
\monstercarac{description}{
  Abeilles géantes, longues de 30 cm, au tempérament agressif. Installent
leurs ruches dans les souterrains.

}
\monsterdetail{Agressive}{Attaque en général à vue. Attaque toujours un intrus se trouvant à moins
de 9 m de la ruche.
}
\monsterdetail{Meurt après l’attaque}{Sur une attaque de dard réussie, l'abeille tueuse meurt.
}
\monsterdetail{Poison}{Provoque la mort (jet de sauvegarde contre le poison).
}
\monsterdetail{Dard fiché}{Reçoit 1 dégât par round à cause du dard qui pénètre plus profondément
dans la blessure. Le retirer prend un round.
}
\monsterdetail{Reine}{Une reine avec 2 DV vit dans la ruche. Elle ne meurt pas après avoir
piqué.
}
\monsterdetail{Gardes}{Au moins 10 abeilles (dont 4 ou plus ont 1 DV) sont toujours présentes
dans ou à proximité de la ruche pour veiller sur la reine.
}
\monsterdetail{Miel}{La ruche contient environ un litre de miel magique. Quiconque mange
cette quantité regagne 1d4 points de vie.
}
\monstercarac{ca}{7 [12]}
\monstercarac{hd}{½* (2 pv)}
\monstercarac{taco}{19 [0]}
\monstercarac{moral}{9}
\monstercarac{alignement}{Neutre}
\monstercarac{xp}{6 (garde : 13, reine : 25)}
\monstercarac{nombre_donjon}{1d6}
\monstercarac{nombre_exterieur}{5d6}
\monstercarac{tresor}{Miel}
\monstercarac{mvt}{45 m (15 m) vol}
\monstercarac{save_mort_poison}{12}
\monstercarac{save_baguettes}{13}
\monstercarac{save_paralysie_petrification}{14}
\monstercarac{save_souffles}{15}
\monstercarac{save_sorts_sceptres_batons}{16}
\monsterattack{1 × dard }{1d3 + poison + dard fiché}
\end{monster}

	
\part{Crédits}	

Ce document est simplement une mise en forme du contenu du SRD "Open School Essential", disponible à l'adresse :
\url{https://oldschoolessentials.necroticgnome.com/fr/srd}.

L'image de couverture a été généré via Midjourney (\url{https://midjourney.com/}) et par conséquent est sous licence "Creative Commons Noncommercial 4.0 Attribution International License" (CC BY-NC 4.0).
Le texte complet de la licence est disponible sous \url{https://creativecommons.org/licenses/by-nc/4.0/legalcode}.

\section*{Open Game Licence}
\input{OGL.txt}
	
\end{document}